\documentclass[letterpaper]{book}
\title{\huge{201-SN1 - Probability \& Statistics}}
\author{B. Cordy}
\date{Winter 2026}

\usepackage{theorem}
\usepackage{amsmath}
\usepackage{amssymb}
\usepackage{pgf}
\usepackage{pgfplots}
\usepackage{pgfplotstable}
\usepackage{pgfmath}
\usepgfplotslibrary{fillbetween}
\usepgfplotslibrary{statistics}
\usepackage{imakeidx}
\usepackage{xcolor}
\usepackage{tikz}
\usepackage{fontawesome5}
\usepackage{enumerate}
\usepackage{enumitem}
\usepackage{textcomp}
\usepackage{latexsym}
\usepackage{cancel}
\usepackage{enumitem}
\usepackage[most]{tcolorbox}
\usetikzlibrary{patterns}
\usetikzlibrary{patterns.meta}
\usetikzlibrary{trees}
\usetikzlibrary{arrows.meta}
\setlength{\parskip}{1em plus 0.2em minus 0.2em}
\setlength{\textheight}{8.1in}
\setlength{\textwidth}{6.25in}
\setlength{\topmargin}{.25in}
\setlength{\oddsidemargin}{.1in}
\setlength{\evensidemargin}{.1in}

\setlength{\theorempreskipamount}{0.5em}   % space above theorem/definition
\setlength{\theorempostskipamount}{0.5em} 

% Set the overall layout of tikz trees
\tikzstyle{level 1}=[level distance=2.5cm, sibling distance=2.25cm]
\tikzstyle{level 2}=[level distance=2.5cm, sibling distance=1.5cm]
\tikzstyle{level 3}=[level distance=2.5cm, sibling distance=1.5cm]

% Define styles for bags and leafs
\tikzstyle{head} = [inner sep=0pt]
\tikzstyle{bag} = [text width=2em, text centered]
\tikzstyle{bigbag} = [text width=6em, text centered]
\tikzstyle{megabag} = [text width=10em, text centered]
\tikzstyle{end} = [minimum width=3pt, inner sep=0pt]

\makeatletter
\long\def\ifnodedefined#1#2#3{%
    \@ifundefined{pgf@sh@ns@#1}{#3}{#2}%
}

\def\aeundefinenode#1{%%
  \ifnodedefined{#1}{\global\expandafter\let\csname pgf@sh@ns@#1\endcsname\relax}{}%
}

\pgfplotsset{
    xtick parsed/.code={
        \c@pgf@counta 0\relax
        \foreach \x in {#1} {
            \pgfmathparse{\x}
            \ifnum\c@pgf@counta=0
                \xdef\pgfplots@xtick{\pgfmathresult}
            \else
                \xdef\pgfplots@xtick{\pgfplots@xtick,\pgfmathresult}
            \fi
            \global\advance\c@pgf@counta 1\relax
        }
    }
 } 

\pgfplotsset{
    cdf init/.code={\aeundefinenode{marker}},
    cdf/.style={
    scatter,
    scatter/@pre marker code/.code={
			\ifnodedefined{marker}{
            \pgfpointdiff{\pgfpointanchor{marker}{center}}%
             {\pgfpoint{0}{0}}%
             \ifdim\pgf@y>0pt
                \tikzset{options/.style={mark=*}}
                \draw plot [mark=*,mark options={fill=white}] coordinates {(marker-|0,0)};
             \else
                \ifdim\pgf@y<0pt
                    \tikzset{options/.style={mark=*,fill=white}}
                    \draw plot [mark=*] coordinates {(marker-|0,0)};
                \else
                    \tikzset{options/.style={mark=none}}
                \fi
             \fi
        }
				{
            \tikzset{options/.style={mark=none}}        
        }
        \coordinate (marker) at (0,0);
        \begin{scope}[options]
    },
    scatter/@post marker code/.code={\end{scope}}
    }
}

\usepackage{fancyhdr}
\pagestyle{fancy}
\lhead{}
\chead{}
\rhead{\thepage}
\lfoot{}
\cfoot{}
\rfoot{}

% Global defaults for itemize and enumerate
\setlist[itemize]{%
    topsep=0.5em,      % space above and below the list
    itemsep=1em,    % space between items
    parsep=0pt,        % extra space between paragraphs inside an item
    partopsep=0pt,     % extra space added when list starts a new paragraph
    leftmargin=*       % align with normal text margin
}

\setlist[enumerate]{%
    topsep=0.5em,
    itemsep=1em,
    parsep=0pt,
    partopsep=0pt,
    leftmargin=*
}

\newenvironment{example}
{\noindent\begin{examp}}
{\hfill \end{examp}}

\newenvironment{definition}
{\noindent\begin{defn}}
{\hfill \end{defn}}

\newtcolorbox{warningbox}{
  enhanced,
  parbox=false,
  colback=red!0,
  colframe=red!0,
  boxrule=0.4pt,
  arc=2pt,
  left=6pt,
  right=6pt,
  top=6pt,
  bottom=3pt,
  before skip=1.5em,
  after skip=1.5em,
  overlay={
    % Erase part of the frame at top-left
    \begin{scope}
      \clip (frame.north west) rectangle ([xshift=2cm,yshift=-1ex]frame.north west);
      \path[fill=red!0, draw=none] (frame.north west) rectangle ([xshift=2cm,yshift=-0.5ex]frame.north west);
    \end{scope}
    \draw[line width=2pt, red] 
      ([xshift=2pt]frame.south west) -- 
      ([xshift=2pt]frame.north west);
    \draw[line width=2pt, red] 
      ([xshift=2pt]frame.south east) -- 
      ([xshift=2pt]frame.north east);
    % Add "Remark" label
    \node[anchor=west, font=\bfseries] at ([xshift=1pt,yshift=0ex]frame.north west) {Warning \faExclamationTriangle};
  }
}

\newenvironment{warning}
{\begin{warningbox}}
{\end{warningbox}}

\newtcolorbox{keypointbox}{
  enhanced,
  parbox=false,
  colback=blue!0,
  colframe=blue!0,
  boxrule=0.4pt,
  arc=2pt,
  left=6pt,
  right=6pt,
  top=6pt,
  bottom=6pt,
  before skip=1.5em,
  after skip=1.5em,
  overlay={
    % Erase part of the frame at top-left
    \begin{scope}
      \clip (frame.north west) rectangle ([xshift=2cm,yshift=-1ex]frame.north west);
      \path[fill=blue!0, draw=none] (frame.north west) rectangle ([xshift=2cm,yshift=-0.5ex]frame.north west);
    \end{scope}
    \draw[line width=2pt, black] 
      ([xshift=2pt]frame.south west) -- 
      ([xshift=2pt]frame.north west);
    \draw[line width=2pt, black] 
      ([xshift=2pt]frame.south east) -- 
      ([xshift=2pt]frame.north east);
    % Add "Remark" label
    \node[anchor=west, font=\bfseries] at ([xshift=1pt,yshift=0ex]frame.north west) {Key Point \faKey};
  }
}

\newenvironment{keypoint}
{\begin{keypointbox}}
{\end{keypointbox}}

\newtcolorbox{remarkbox}{
  enhanced,
  parbox=false,
  colback=blue!0,
  colframe=blue!0,
  boxrule=0.4pt,
  arc=2pt,
  left=6pt,
  right=6pt,
  top=6pt,
  bottom=6pt,
  before skip=1.5em,
  after skip=1.5em,
  overlay={
    % Erase part of the frame at top-left
    \begin{scope}
      \clip (frame.north west) rectangle ([xshift=2cm,yshift=-1ex]frame.north west);
      \path[fill=blue!0, draw=none] (frame.north west) rectangle ([xshift=2cm,yshift=-0.5ex]frame.north west);
    \end{scope}
    \draw[line width=2pt, black] 
      ([xshift=2pt]frame.south west) -- 
      ([xshift=2pt]frame.north west);
    \draw[line width=2pt, black] 
      ([xshift=2pt]frame.south east) -- 
      ([xshift=2pt]frame.north east);
    % Add "Remark" label
    \node[anchor=west, font=\bfseries] at ([xshift=1pt,yshift=0ex]frame.north west) {Remark \faHandPointUp};
  }
}

\newenvironment{remark}
{\begin{remarkbox}}
{\end{remarkbox}}

\newenvironment{note}
{\begin{remarkbox}}
{\end{remarkbox}}

\newenvironment{theorem}
{\noindent\begin{thm}}
{\hfill \end{thm}}

\newenvironment{corollary}
{\noindent\begin{cor}}
{\hfill \end{cor}}

\newenvironment{proof}
{\vspace*{-7pt}\noindent\begin{pf}}
{\hfill $\Box$ \end{pf}}

\newenvironment{proofnobox}
{\vspace*{-7pt}\noindent\begin{pf}}
{\hfill \end{pf}}

\newenvironment{lemma}
{\noindent\begin{lem}}
{\hfill \end{lem}}

\newenvironment{proposition}
{\noindent\begin{prop}}
{\hfill \end{prop}}
%
\newtheorem{examp}{Example}[chapter]
\newtheorem{defn}{Definition}[chapter]
\newtheorem{thm}{Theorem}[chapter]
\newtheorem{cor}{Corollary}[chapter]
\newtheorem{prop}{Proposition}[chapter]
\newtheorem{lem}{Lemma}[chapter]
\newcommand{\newterm}[1]{{\bf\emph{#1}}}
\newcommand{\exercises}{\subsection*{Exercises}}
\newcommand{\pf}{\noindent{\bf Pf.}}
\newcommand{\vsp}{\vspace{0.1in}}
\newcommand{\emx}[1]{{\em{#1}\/}}
\newcommand{\mat}[1]{{\mbox{\bf#1}\/}}
\newcommand{\inbox}[1]{{\begin{center}\begin{minipage}{0.8\textwidth}\hrule \vspace{5pt}  #1 \vspace{5pt} \hrule \end{minipage}\end{center}}}
\newcommand{\eqnsbiggap}[1]{{$$\begin{aligned} #1 \end{aligned}\vspace*{5pt}$$}}
\newcommand{\eqns}[1]{{$$\begin{aligned} #1 \end{aligned}\vspace*{-10pt}$$}}
\newcommand{\eqnsgap}[1]{{$$\begin{aligned} #1 \end{aligned}$$}}
\newcommand{\eqnspar}[1]{{$$\begin{aligned} #1 \end{aligned}\vspace*{-18pt}$$}}
\newcommand{\given}{\,|\,}
\newcommand{\biimp}{\ \Leftrightarrow\ }   % changed from \leftrightarrow
\newcommand{\xbar}{{\overline{X}}}
\newcommand{\littlexbar}{{\overline{x}}}
\newcommand{\muxbar}{\mu_{\overline{X}}}
\newcommand{\sigmaxbar}{\sigma_{\overline{X}}}

\DeclareMathOperator{\Var}{Var}

\DeclareMathOperator{\Uniform}{Uniform}
\DeclareMathOperator{\Bernoulli}{Bernoulli}
\DeclareMathOperator{\Binomial}{Binomial}
\DeclareMathOperator{\Geometric}{Geometric}
\DeclareMathOperator{\Poisson}{Poisson}
\DeclareMathOperator{\Exponential}{Exponential}
\DeclareMathOperator{\Normal}{Normal}
\DeclareMathOperator{\Gaussian}{Gaussian}



\makeatletter
\renewcommand\chapter{\if@openright\cleardoublepage\else\clearpage\fi
                    \thispagestyle{fancy}%
                    \global\@topnum\z@
                    \@afterindentfalse
                    \secdef\@chapter\@schapter}
\makeatother


\makeindex


% --- hyperref ---
% make all (internal and external) links clickable, adjust colors as wanted,
% this also creates a Table of Contents in the final pdf ("bookmarks" in pdf-parlance)
\usepackage[colorlinks=true, urlcolor=black, linkcolor=black, citecolor=black, bookmarksopen, bookmarksopenlevel=2, bookmarksnumbered]{hyperref}

\usepackage{idxlayout}  % make sure the index doesn't use a different pagestyle (also allows other customization)


\begin{document}
\addtolength{\normalbaselineskip}{1pt}
\normalbaselines

\frontmatter
\include{front}

\mainmatter
% Modified 31 Oct 2005:  Conditioning fallacy alluded to.
% This chapter has been modified on 6-4-05.
% There are two \choice
\pagestyle{headings}
\setcounter{chapter}{0}
\chapter{Descriptive Statistics} \label{chp 1}
\setcounter{page}{1}

\section{Sampling}

\begin{definition}
Given a research question to be addressed by a study or experiment, the group that the question concerns is called a \newterm{population}\index{Population}. Our goal is to try to answer the question using data about some smaller subset of the population, known as a \newterm{sample}\index{Sample}.
\end{definition}

\begin{example}
We are interested in whether a new treatment for pollen allergies is more effective than an existing treatment. To try to answer this question, we can take a large group of people with pollen allergies, give half the new treatment, give the other half the existing treatment, and have them report their results so we can compare. 

We're interested in studying individuals with pollen allergies, in general, that's the population in this scenario. Only those who took one of the two treatments and reported their results back are members of the sample.
\end{example}

\begin{keypoint}
Someone who took one of the treatments, but did not report their results, is not in our sample. The sample is the group we have data about.
\end{keypoint}

Perhaps the simplest way to illustrate the distinction is to think of the population as the group we \emph{want to know about}, and the sample as the group we \emph{do know about}.

\begin{example}
A semiconductor fabrication plant wants to estimate the proportion of defective chips among their total production. To do this, they test every twentieth chip that leaves the production line.

Since they're interested in what proportion of all chips they produce are defective, the population here consists of all chips produced at their plant. The sample includes only those that are tested.
\end{example}

\subsection*{Statistics \& Parameters}

\begin{definition}
A numerical property of a population is called a \newterm{parameter}\index{Parameter}, while a numerical property of sample is called a \newterm{statistic}\index{Statistic}.
\end{definition}

\begin{example}
In a forest where $41.3\%$ of trees are cedar, a biologist (who is unaware of this fact) estimates the proportion of cedar trees by selecting twenty different trees in the forest at random. She notes that eleven of the twenty trees in her sample are cedar, so estimates that $\frac{11}{20} = 55\%$ of trees in the forest are cedar. In this context the value of $55\%$ she computes is a statistic, and the true value of $41.3\%$ is a parameter.
\end{example}

In some contexts, the distinction is more subtle. The point is to consider how a quantity relates to the question that it's being used to address. Does the value give a definitive answer based on complete information, with no uncertainty, or is it based on partial information or subject to some degree of uncertainty?

\begin{example}
LeBron James' free throw percentage in 2024 was $56.6\%$. If this figure is being used to answer the question `what was LeBron James' free throw percentage in 2024', then it's a parameter, but if it's being used to answer the question `how likely is LeBron James to make his next free throw' then it's a statistic.
\end{example}

\begin{remark}
Some quantities colloquially called statistics are not, in fact, statistics. If a college concludes that 23\% of its current students are enrolled in the science program by examining their complete records of student enrollment, they haven't computed a statistic, they've computed a parameter.
\end{remark}

The discipline of statistics is about making decisions using sample data. There are so many contexts where sample data is the only kind of information or the highest quality kind of information we have access to, which is why the applications of statistics are so varied.

Given a research question, and relevant sample data, the first question to ask is: how was the sample obtained? Our ability to do statistics well depends on one factor more than any other: do we have an accurate model of the sampling process? In other words, do we understand the process that generated the data we're working with?

\subsection*{Three Kinds of Samples}

\begin{definition}
If we take as our sample those members of the population that are easiest to access, we obtain a \newterm{convenience sample}\index{Convenience Sample}. \\ If we order our population in some way and periodically select members at regular intervals, we obtain a \newterm{systematic sample}\index{Systematic Sample}. \\ If we select members of the population at random, with all members having an equal chance of appearing on every selection, we obtain a \newterm{simple random sample}\index{Simple Random Sample}.
\end{definition}

\begin{remark}
Sampling can be done \newterm{with replacement}\index{Sampling!with replacement}, which means the same individual may appear many times in the sample, or \newterm{without replacement}\index{Sampling!without replacement}, which means that each individual can appear at most once in the sample.

Think about this process for generating a simple random sample: we put the names of every individual in our population (supposing these are unique) into a hat and draw names. When sampling with replacement, after drawing a name we write it down, put it back in the hat, and shake the hat. When sampling without replacement, the names that are drawn are not put back into the hat.
\end{remark}

In the rest of the course, whenever we refer to a random sample, this will be shorthand for a simple random sample taken with replacement. This means the sampling process is equivalent to repeatedly rolling a die labelled with the names of the members of our population, with the outcome of any one roll having no effect on the outcome of any other. This will allow us, later on in the course, to provide precise justifications and error bounds for our estimates and predictions by applying the theory of probability.

\begin{warning}
In colloquial language, the term random sample could be interpreted as a sample derived from an unknown or not well-understood process. In statistics, a random sample (a simple random sample taken with replacement) results from a completely specified process, namely, drawing names from a hat in manner where each name is equally likely to appear on each selection, regardless of which names are drawn before or after, or repeatedly rolling a die where each side corresponds to a different individual in the population.
\end{warning}

\begin{example}
A semiconductor fabrication plant wants to estimate the proportion of defective chips among their total production. To do this, they test every twentieth chip that leaves the production line.

In this context, the sample obtained will be a systematic sample, not a random sample. If they select a random number between zero and twenty and start testing every twentieth chips after that many chips have been produced, they will still not obtain a random sample, since a sample which contains two consecutively produced chips cannot occur.
\end{example}

\begin{example}
In her statistics class, Anne is doing a project where she wants to estimate how many hours per week students in her age group spend commuting to and from school. She gathers data by asking ten of her friends about their commuting routine. 

In this case, Anne will obtain a convenience sample. Conclusions she draws about students in her age group are dubious, since she's only sampling students she happens to be close with.
\end{example}

Notice that in practice, it's often very difficult to obtain a random sample. It requires a complete list of all members of the population under study, and if these members are people, a way to guarantee they will volunteer the information the experimenters are interested in knowing. If you read any corporate software license agreement, you will almost certainly see a section which affirms all your user data is accessible for studies carried out by the company that developed the software. This ensures the company can obtain high quality samples from the population of its users.

An interesting sampling case study is the COVID-19 outbreak on the MV Diamond Princess, which was quarantined in February 2020. Under normal circumstances, there's no way to obtain a sample of people who are known to have been exposed to a disease (and may or may not have any symptoms), as we cannot forcibly expose people for ethical reasons, and subjects cannot self-select because those who are asymptomatic are often unaware they've been exposed at all. Because no one was allowed to leave the ship, researchers were able to derive accurate estimates of parameters like the proportion of COVID-19 cases which present as asymptomatic.

\section{Variables \& Levels of Measurement}

\begin{definition} 
The members of the population are referred to as \newterm{individuals}\index{Individuals}. Any property of these individuals which can take different values for different individuals is called a \newterm{variable}\index{Variable}.
\end{definition}

\begin{center}
\begin{tabular}{c|c|c|c|c}
Model & Year & Colour & Mileage & Condition \\ \hline
Toyota Yaris & 2015 & Blue & 85\,000 & B \\
Honda CRV & 2018 & Black & 140\,000 & C \\
Ford F-150 & 2022 & Red & 46\,000 & A \\
Subaru Outback & 2019 & Blue & 77\,000 & B \\
\end{tabular}
\end{center}

Typically, if we represent our data in a table, as above, each individual corresponds to a row of the table (in this case the individuals are used cars in some car seller's inventory), and each variable corresponds to a column. Variables can describe any aspect of the individuals in the data set. It's often useful to categorize them based on which operations we can meaningfully apply.

\begin{remark}
In statistics, the letter $n$ is reserved for the sample size, that is, the number of individuals in the dataset, and will appear frequently in formulas and explanations. In the dataset above, $n=4$.
\end{remark}

\begin{definition}
As we answer the three questions below, each affirmative answer moves the variable one level up in the hierarchy of \newterm{levels of measurement}\index{Levels of Measurement}.
$$\textbf{Categorical} \ \to \ \textbf{Ordinal} \ \to \ \textbf{Interval} \ \to \ \textbf{Ratio}$$
\begin{itemize}
\item Can we naturally order the values? (The $<$ and $>$ operations are meaningful)
\item Can we subtract and average the values? (The $+$ and $-$ operations are meaningful)
\item Can we multiply and divide the values? (The $\cdot$ and $\div$ operations are meaningful)
\end{itemize}
\end{definition}

\noindent\textbf{\emph{Ratio}}: Variables like height, distance, or anything else you can measure with a ruler, are at the ratio level. They are naturally ordered and we can perform arithmetic with them. In particular, ratios are meaningful. One distance can be twice as far as another, or two-thirds as far.

\noindent\textbf{\emph{Interval}}: Variables like shoe size or temperature (in\,$\,^{\circ} C$) are at the interval level. Ratios of values are not meaningful. The shoe size 8 is not twice as large as 4, and 10$\,^{\circ} C$ is not twice as hot as 5$\,^{\circ} C$. Values can be put a scale with equal increments, so we can form differences, say a difference of 5$\,^{\circ} C$ between two temperatures, and also compute averages.

\noindent\textbf{\emph{Ordinal}}: Variables like letter grade or level of satisfaction (very unsatisfied, unsatisfied, neutral, satisfied, or very satisfied) are at the ordinal level. None of the four basic arithmetic operations ($+$, $-$, $\cdot$, $\div$) are meaningful, but there's a clear natural ordering.

\noindent\textbf{\emph{Categorical}}: Variables like name, location, or colour are at the categorical level. The values of the variable have no natural ordering, they are simply labels that can be used to group individuals into categories.

In the case of our used car data in the table above, Model and Colour are at the categorical level, Condition is at the ordinal level, Year is at the interval level, and Mileage is at the ratio level.

\begin{note}
This system of classification of variables into four groups is widespread, but not universal. There are alternative systems, but the one described above is widely known and over time has become part of standard statistical vocabulary.
\end{note}

%Examples

\section{Histograms}

Any variable, regardless of its level of measurement, has some collection of values it may take, and each of these values occurs some number of times in the data. This information, taken all together, constitutes the \newterm{distribution} of the variable. The most intuitive way to understand the distribution of a variable is often through a \newterm{histogram}\index{Histogram}.

\begin{center}
\begin{tikzpicture}
    \begin{axis}[
        width=11cm,
        height=7cm,
        xmin=0, xmax=40,
        ymin=0, ymax=9,
        xtick distance=4,
        ytick = {2,4,6,8},
        area style,
    ]
    \addplot+[ybar interval] plot coordinates { (0,0) (4,1) (8,0) (12,2) (16,6) (20,5) (24,7) (28,4) (32,5) (36,3) (40,1)};
\end{axis}
\end{tikzpicture}
\end{center}

The horizontal axis, which represents the variable's value, is divided into segments of equal width, known as \newterm{bins} or \newterm{classes}, and the height of each bar is the number of individuals whose values fall into that particular bin. Each bin has a \newterm{lower limit} and an \newterm{upper limit}. The distance between consecutive lower limits is called the \newterm{bin width} or \newterm{class width}.

The above discussion only applies to variables at the interval or ratio levels of measurement. When we don't have a scale which can be divided into segments of equal width, we can instead create one bin for each possible value of the variable, and the result is usually called a \newterm{bar graph}.

\begin{example}\label{FirstHistogram}
Create a histogram with five bins, to visualize the distribution of ages in a room of kids whose ages are given, in order, below.
$$2\ \ 3\ \ 3\ \ 4\ \ 5\ \ 5\ \ 6\ \ 7\ \ 7\ \ 7\ \ 7\ \ 7\ \ 7\ \ 8\ \ 8\ \ 9\ \ 9\ \ 10\ \  11\ \ 11\ \ 11\ \ 12 \ \ 14$$
We need to partition these 23 values into five bins of equal width. One way to do this is to compute the \newterm{range} of the data (the difference between the smallest and largest value), then add one and divide that by the number of bins, in this case, five.

This calculation yields $\frac{14-2+1}{5} = 2.6$, which we'll round up to 3, and take as the bin width. Adding one to the range and always rounding up will ensure the largest value doesn't exceed the upper limit of the last bin. Taking the smallest data value as the lower limit of the first bin, and repeatedly adding the bin width to obtain the lower limits of the other bins gives the table and histogram below. 

\begin{center}
\begin{tabular}{|c|c|c|c|c|c|}
\hline
Bins & \,$2 - 4$\, & \,$5 - 7$\, & \,$8 - 10$ & $11 - 13$ & $14 - 16$ \\
\hline
Counts & $4$ & $9$ & $5$ & $4$ & $1$ \\
\hline
\end{tabular}
\end{center}

\begin{center}
\begin{tikzpicture}[scale=0.8]
    \begin{axis}[
        width=11cm,
        height=7cm,
        xmin=2, xmax=17,
        ymin=0, ymax=11,
        xtick = {2,5,8,11,14,17},
        ytick = {2,4,6,8,10},
        area style,
    ]
    \addplot+[ybar interval] plot coordinates { (2,4) (5,9) (8,5) (11,4) (14,1) (17,0) };
\end{axis}
\end{tikzpicture}
\end{center}
\end{example}

The precise details of the process that creates the bins is not important, as long as the horizontal axis is divided into consecutive segments of equal width in a way that gives a useful visualization of the distribution of the variable.

\begin{example}
Consider the first eleven positive integers. The histogram below on the left was created from these values by using 2.2 for the bin width, and 0.2 for the lower limit of the first bin. The histogram below on the right was created from the same values by using 1.9 for the bin width, and 1 for the lower limit of the first bin.
\begin{center}
\begin{minipage}{0.47\textwidth}
\centering
\begin{tikzpicture}[scale=0.65]
    \begin{axis}[
        width=11cm,
        height=7cm,
        xmin=0.2, xmax=11.2,
        ymin=0, ymax=4,
        xtick = {0.2,2.4,4.6,6.8,9,11.2},
        ytick = {2,4,6,8,10},
        area style,
    ]
    \addplot+[ybar interval] plot coordinates { (0.2,2) (2.4,2) (4.6,2) (6.8,2) (9,3) (11.2,0)};
\end{axis}
\end{tikzpicture}
\end{minipage}
\begin{minipage}{0.47\textwidth}
\centering
\begin{tikzpicture}[scale=0.65]
    \begin{axis}[
        width=11cm,
        height=7cm,
        xmin=1, xmax=12.4,
        ymin=0, ymax=4,
        xtick = {1,2.9,4.8,6.7,8.6,10.5,12.4},
        ytick = {2,4,6,8,10},
        area style,
    ]
    \addplot+[ybar interval] plot coordinates { (1,2) (2.9,2) (4.8,2) (6.7,2) (8.6,2) (10.5,1) (12.4,0) };
\end{axis}
\end{tikzpicture}
\end{minipage}
\end{center}

There is nothing incorrect about either of the two histograms, it's simply a quirk of the way the axis was divided into bins that results in the difference in appearance. This illustrates how it's sometimes possible to build intentional bias into a histogram.
\end{example}

As mentioned above, we won't dwell on the details of the process of creating bins, but suffice to say it is good practice to have some consistent, explicit process (usually the one built-in to whatever software package is being used to generate the histogram) and not tweak the details until the result is deemed satisfactory, which one might call `bin hacking'.

%\begin{note}
%If a data value lies exactly on the boundary of two bins, it has be put into one or the other. This decision should always be made the same way (either all values on the boundary are put in the smaller bin, or all values on the boundary are put in the larger bin) when creating a histogram.
%\end{note}

%Idea of a distribution

%Examples of Histograms

%Example where choosing different boundaries on categories gives a different impression

\section{Measures of Center}\label{MeasuresOfCenterSec}

It's often useful to give a description of where the middle of a distribution lies by producing a single `typical' or `central' value. The mean, median, and mode are the most common methods of producing such a value.

\begin{definition}
Let $x_1, x_2,\,\dots\,, x_n$ be values of a statistical variable that occur across the $n$ individuals in a dataset.
\begin{itemize}
\item The \newterm{mean}\index{Mean} value of the variable is given by $\frac{1}{n}\sum_{i} x_i = \frac{1}{n}(x_1 + x_2 + \dots + x_n)$.
\item The \newterm{median}\index{Median} value of the variable is obtained by ordering $x_1, x_2, \dots\,, x_n$ smallest to largest, then taking the value in the middle of the resulting list if $n$ is odd, and the mean of the two values nearest the middle if $n$ is even.
\item The \newterm{mode}\index{Mode} is the value of the variable which occurs most frequently in the data. If no value occurs more than once, we say the variable has no mode. Note that there may be more than one mode.
\end{itemize}
\end{definition}

\begin{example}
Consider the set of values $1,1,2,3,5$. We can compute the mean $\frac{1}{5}(1+1+2+3+5) = 2.4$, and since the values are already given in sorted order, we can see the median is $2$ and the mode is $1$.
\end{example}

Note that these three measures of centre correspond to three levels or measurement. The mean is defined at the interval level and above, the median is defined at the the ordinal level and above, and the mode is defined even at the categorical level.

\begin{warning}
In colloquial language, the average almost always refers to the mean, but there are many situations where it doesn't. The average letter grade cannot refer to the mean, since letter grades are ordinal data, and income and wealth averages quoted in the media are often medians, not means.
\end{warning}

Here's a scenario that illustrates the relevance of each measure: In a guessing game, five numbered balls are put into an urn. The distribution of numbers matches our data in the last example (two balls labelled with 1, one with 2, one with 3, and one with 5). A ball is drawn at random from the urn, and the player's task is to guess the number the ball will be labelled with.

If there is a \$10 reward for guessing correctly, and no payout for guessing incorrectly, clearly the best strategy is to guess the mode, $1$. What's not so immediate is that if instead the player is rewarded \$10 minus the error in their guess (so guessing 3 then drawing a 5 will yield a reward of $10 - 2 = 8$ dollars), the best strategy is to guess the median, 2, and if the player is rewarded \$10 minus the squared error in their guess (so guessing 3 then drawing a 5 will yield a reward of $10 - 2^2 = 6$ dollars), then the best strategy is to guess the mean, 2.4. 

This is not a quirk that happens because of the particular choice of labels on the balls. In general, regardless of the distribution of values, \emph{the median minimizes the average absolute error} and \emph{the mean minimizes the average squared error}. These results can be stated formally and proven in the language of random variables (which we'll get to in Chapter 4) with techniques from differential calculus.

\begin{example}
In the preceding section, we constructed the histogram shown below to summarize the distribution of the variable whose values were as follows.
$$2\ \ 3\ \ 3\ \ 4\ \ 5\ \ 5\ \ 6\ \ 7\ \ 7\ \ 7\ \ 7\ \ 7\ \ 7\ \ 8\ \ 8\ \ 9\ \ 9\ \ 10\ \  11\ \ 11\ \ 11\ \ 12 \ \ 14$$
\begin{center}
\begin{tikzpicture}[scale=0.8]
    \begin{axis}[
        width=11cm,
        height=7cm,
        xmin=2, xmax=17,
        ymin=0, ymax=11,
        xtick = {2,5,8,11,14,17},
        ytick = {2,4,6,8,10},
        area style,
    ]
    \addplot+[ybar interval] plot coordinates { (2,4) (5,9) (8,5) (11,4) (14,1) (17,0) };
\end{axis}
\end{tikzpicture}
\end{center}
There are $23$ values in the ordered list, so the median is the $12^{th}$ value, which is a 7, while the mean is $\frac{1}{23}(2 + 3 + 3 + \dots + 14) = 7.57$. Notice that the mean is a little bit larger than the median. When the distribution of a variable has a longer tail on one side, the mean is pulled towards the tail more than the median.
\end{example}

The most extreme form of this effect occurs when a data value which is much larger than or much smaller than all the other values is present. Such values could occur by mistake, as typos, or could occur because of the nature of the population that the data was sampled from.

\begin{keypoint}
Adding an extra value (or two, or three) to a dataset which is very different from the rest, that is, adding an \newterm{outlier}\index{Outlier}, will affect the mean more than the median. It's common to summarize this by saying the median is more \emph{stable} than the mean.
\end{keypoint}

For this reason, I encourage students to pay more attention to the median grades in their classes than the mean grades. If a few students miss a test and receive a zero, this can have a noticeable effect on the mean, and give students a false impression of where they stand relative to their classmates. Students are also often concerned about whether their grade ranks them among the top half of the class, which is a question the median answers, not the mean.

In general, if the data being studied contains outliers, the median is often preferred over the mean. This is why it's more common to measure median income than mean income. The distribution of income has such a long tail that citing the mean income can be deceptive when studying populations with high income inequality.

\begin{example}
Mr.\,Brown's grade three class has fifteen students, and the mean height in his class is 117\,cm. Mrs.\,Green's grade three class has twelve students, and the mean height in her class is 122\,cm. What is the mean height in the combined group of twenty-seven students?

We know that $\frac{S_B}{15} = 117$ and $\frac{S_G}{12} = 122$, where $S_B$ and $S_G$ are the sums of the heights of all students in Mr.\,Brown's and Mrs.\,Green's classes, respectively. Therefore, $S_B = 15 \cdot 117$ and $S_G = 12\cdot 122$. We can then calculate the mean of the entire group as
$$\frac{S_B + S_G}{27} = \frac{15\cdot 117 + 12 \cdot 122}{27} = 119.22.$$
\end{example}

The example above, and the one below, are typical of more involved problems about means, which often require you to pass from a mean to the corresponding sum. You should get used to doing this, as the same idea will reappear later in the course when we study the central limit theorem.

\begin{example}
Mr.\,Brown's grade three class has fifteen students, and the mean height in his class is 117\,cm. If a new student who is 114\,cm tall joins the class, how will the mean change?

As above, we know that $\frac{S_B}{15} = 117$, so $S_B = 15 \cdot 117$. We can then calculate the mean of the new group of sixteen students as below, and conclude the mean height will drop by 0.19\,cm.
$$\frac{S_B + 114}{16} = 116.81$$
\end{example}

\begin{remark}
It will be extremely important to distinguish the mean of a population from the mean of a random sample taken from a population. For this reason, the greek letter $\mu$ is used for the mean of a population (a parameter), while $\littlexbar$ is used for the mean of a sample (a statistic). It is standard practice to reserve greek letters for parameters to keep the parameter / statistic distinction clear.
\end{remark}

\begin{example}
A random number generator produces the values $1$, $2$, $3$, $4$, $5$, and $6$ with equal probability. Three random numbers are produced, and the results are $2$, $2$, and $5$.

The mean value generated by the random number generator is $\mu = \frac{1}{6}(1+2+3+4+5+6) = 3.5$, while the mean in the sample of three values is $\littlexbar = \frac{1}{3}(2+2+5) = 3$.
\end{example}
%Examples

%Exercise for the reader: show these minimize different expectations <- Later!

\section{Measures of Dispersion}\label{MeasuresOfDispersionSec}

\begin{center}
\begin{minipage}{0.47\textwidth}
\centering
\begin{tikzpicture}[scale=0.65]
    \begin{axis}[
        width=11cm,
        height=7cm,
        xmin=0, xmax=10,
        ymin=0, ymax=8,
        xtick = {0,1,2,3,4,5,6,7,8,9,10},
        ytick = {2,4,6,8,10},
        area style,
    ]
    \addplot+[ybar interval] plot coordinates { (0,0) (1,0) (2,1) (3,2) (4,5) (5,7) (6,4) (7,1) (8,0) (9,0) (10,0)};
\end{axis}
\end{tikzpicture}
\end{minipage}
\begin{minipage}{0.47\textwidth}
\centering
\begin{tikzpicture}[scale=0.65]
    \begin{axis}[
        width=11cm,
        height=7cm,
        xmin=0, xmax=10,
        ymin=0, ymax=8,
        xtick = {0,1,2,3,4,5,6,7,8,9,10},
        ytick = {2,4,6,8,10},
        area style,
    ]
    \addplot+[ybar interval] plot coordinates { (0,1) (1,2) (2,2) (3,3) (4,4) (5,4) (6,4) (7,3) (8,2) (9,1) (10,0)};
\end{axis}
\end{tikzpicture}
\end{minipage}
\end{center}

The two distributions pictured in the histograms above have a similar mean, somewhere between 5 and 6 in both cases, but differ noticeably in how the data is distributed around the mean. The variable whose distribution is pictured on the right is more likely to take values farther from it's mean. We say it has higher \newterm{dispersion} than the variable on the left. In this section we'll introduce a few ways to quantify this notion.

We could measure dispersion by simply taking the difference between the smallest and largest data values. This is known as the \newterm{range}\index{Range} of the variable. However, the range is extremely sensitive to outliers, and ignores almost all of the data (all but the largest and smallest values). We can do better.

\begin{definition}
Let $Q_1$ denote the median of all values smaller than the median, and let $Q_3$ denote the median of all values larger than the median. These are known, respectively, as the first and third quartile\index{Quartiles}. The \newterm{interquartile range}\index{Interquartile Range} is given by $IQR = Q_3 - Q_1$.
\end{definition}

The idea here is to divide the sorted data into blocks: the smallest $25\%$, the middle $50\%$, and the largest $25\%$. The interquartile range is the width of the middle block. This is a quick way to quantify dispersion, which as with median, is not strongly influenced by the presence of outliers.

\begin{example}
Calculate median and IQR of the variable that takes the values below. 
$$1\ \ 1\ \ 2\ \ 4\ \ 7$$
The values are already in sorted order, so we can see the median is $2$, and the first and third quartiles are $Q_1 = \frac{1+1}{2} = 1$, and $Q_3 = \frac{4+7}{2} = 5.5$. Then $IQR = 5.5 - 1 = 4.5$.
\end{example}

Although the interquartile range is useful for summarizing data, by far the most important measure of dispersion in statistical theory and practice is the variance, together with its square root, the standard deviation.

\begin{definition}\label{VarianceDef}
The average squared distance from the mean is called the \newterm{variance}\index{Variance! of a Population}, and denoted $\sigma^2$. If $x_1, x_2,\,\dots\,, x_n$ are the values of a statistical variable that occur across the $n$ individuals in the dataset, and $\mu$ is the mean of these values, then the variance is given by $\sigma^2 = \textstyle\frac{1}{n}\sum_{i=1}^{n}(x_i - \mu)^2$. The square root of the variance, $\sigma$, is known as the \newterm{standard deviation}.\index{Standard Deviation! of a Population}
\end{definition}

\begin{example}
Find the variance and standard deviation of the variable that takes the values below. 
$$1\ \ 1\ \ 2\ \ 4\ \ 7$$
First we compute $\mu = \frac{1}{5}(1+1+2+4+7) = 3$. Then we can calculate the variance as
$$\sigma^2 = \textstyle\frac{1}{5}[(1-3)^2+(1-3)^2+(2-3)^2+(4-3)^2+(7-3)^2] = \frac{1}{5}[4+4+1+1+16] = \frac{26}{5},$$
and the standard deviation is $\sigma = \sqrt{\frac{26}{5}}=2.28$.
\end{example}

\begin{warning}
The standard deviation is not the average distance from the mean. We know that $\sqrt{a^2 + b^2} \neq a + b$, and similarly, the average of a sequence of terms is not generally the same as the square root of the average of their squares.

The actual average distance from the mean is known as the mean absolute deviation\index{Mean Absolute Deviation}, which we will not study in this course. It's a fine measure of dispersion, but does not play as central a role in statistics as the variance and standard deviation do.
\end{warning}

%Note: Percentiles in R.V. section along with CDFs

%Examples

%Def: Variance, Sample Variance, 

\subsection*{Bessel's Correction}

%Typically, when we measure some property of a sample, the point is to use that information as an estimate of the corresponding property of the population. To get an idea of what the mean age is in a certain geographic area, we can take a random sample, calculate the mean age in the sample, and use that statistic as an estimate.

%This process seems completely natural, however, there's a lot of subtlety hiding here. If we take the variance in a random sample as an estimate of the variance in the population the sample was drawn from, it turns out that, on average, we will be underestimating.

Suppose we want to determine the age of the oldest person in a population, and we do so by taking a random sample, and using the age of the oldest person in the sample as our estimate. It should be clear that, typically, the result will be an underestimate. Why? This process will certainly never produce an overestimate, and unless the random sample happens to contain the oldest person in the population, or someone else of the same age, the estimate will be too low. Thus, if we average over all possible samples, the value of our estimate will be lower than the actual age of the oldest person in the population.

This problem doesn't occur when using the sample mean to estimate the population mean. It turns out that the average value of the sample mean, across all possible samples, is always precisely equal to the population mean (we say the sample mean is an \newterm{unbiased estimator}\index{Unbiased Estimator} of the population mean, while the sample maximum is a \newterm{biased estimator}\index{Biased Estimator} of the population maximum). These notions are developed rigorously in the Probability \& Random Variables option course, 201-PRV.

Unfortunately, if we calculate the variance in our sample using the formula introduced earlier, this results in a biased estimator of the population variance, which undershoots on average. The good news is that we can fix the problem easily, and obtain an unbiased estimator for the population variance, by making a small change in the way we calculate the variance of a sample, called Bessel's correction, after German astronomer and mathematician Friedrich Bessel.

\begin{definition}\index{Sample Standard Deviation}\index{Standard Deviation! of a Sample}\index{Variance! of a Sample}\index{Bessel's Correction}\label{SampleStdDev}The (Bessel corrected) \newterm{sample variance} is $s^2 = \frac{1}{n-1}\sum_{i} (x_i - \overline{x})^2$ and the (Bessel corrected) \newterm{sample standard deviation}, $s$, is its square root.
\end{definition}

There are two important changes to note. First, the mean is denoted $\littlexbar$, since we're working with sample data, and second, instead of averaging as usual, by dividing by the number of terms in the sum, we decrease the divisor by one. This is Bessel's correction.

\begin{example} 
Find the mean and sample variance of the sample of five values $2$, $0$, $7$, $3$, $3$.
$$\littlexbar = \textstyle\frac{1}{n}\sum_{i} x_i = \frac{1}{5}(2+0+7+3+3) = 3$$
$$s^2 = \textstyle\frac{1}{n-1}\sum_{i} (x_i - \overline{x})^2 = \frac{1}{4}[(-1)^2+(-3)^2+(4)^2+(0)^2+(0)^2] = \frac{13}{2}$$
\end{example}

In practice, the Bessel corrected variance and standard deviation are used almost universally. Try calculating the standard deviation of a small list of values using spreadsheet software on a computer. You'll find two commands to calculate the standard deviation, one where Bessel's correction is applied (STDEV.S in Excel or Google Sheets) and one where it isn't (STDEV.P in Excel or Google Sheets). 

\begin{remark}
Sample variance and sample standard deviation always refer to the Bessel corrected values. When referring to the formula in Definition \ref{VarianceDef}, we'll omit the prefix sample, or sometimes explicitly use the terms population variance or population standard deviation to be as clear as possible.
\end{remark}

\begin{example}
Consider a standard fair six-sided die. Find the mean and variance for the outcome of a single die roll. Find the mean and variance for a sample of four rolls whose results are $2$, $3$, $6$, $1$.

Treating the list of values $1$, $2$, $3$, $4$, $5$, $6$ as a complete population, we have
$$\mu = \textstyle\frac{1}{6}(1+2+3+4+5+6) = 3.5$$
$$\sigma^2 = \textstyle\frac{1}{6}[(-2.5)^2+(-1.5)^2+(-0.5)^2+(0.5)^2+(1.5)^2+(2.5)^2] = \frac{35}{12}.$$

From the sample of values $2$, $3$, $6$, $1$, we have
$$\littlexbar = \textstyle\frac{1}{4}(2+3+6+1) = 3$$
$$s^2 = \textstyle\frac{1}{3}[(-1)^2+(0)^2+(3)^2+(2)^2] = \frac{14}{3}.$$
\end{example}





% Modified 31 Oct 2005:  Conditioning fallacy alluded to.
% This chapter has been modified on 6-4-05.
% There are two \choice
\pagestyle{headings}
\chapter{Sets \& Counting} \label{chp 2}

\section{Sets and Set Notation}

Sets formalize the concept of a collection, and the language of set theory is ubiquitous in many branches of mathematics, so being familiar with the language and notation of set theory will be useful not only in this course, but also in mathematics more generally.

\begin{definition}
A \newterm{set}\index{Set} is an unordered collection of objects, called its \newterm{elements}\index{Element}, which could be anything. Typically we denote sets with capital letters, and elements of sets with lower case letters when we want to make a clear distinction.
\end{definition}

\begin{example}
Let $A = \{\heartsuit, p, 2, 3, \diamondsuit\}$. This set has five elements. Note that if $B = \{3,2,p,\heartsuit,\diamondsuit\}$, then $A = B$. Sets are equal when they have the same elements.
\end{example}

\begin{keypoint}
Sets have no notion of ordering or repetition. The sets $S =\{a,b,c\}$ and $T=\{c,b,b,a\}$ are equal. You'll never (from now on) see an element repeated in a set for this reason. \end{keypoint}

It's helpful to conceive of a set as a way of dividing the entire universe into two groups, the things that are in the set, and the things that are not. When asked whether any particular thing is in the set, there is always a definitive answer of yes or no. Two sets are equal when the answers to that question always agree.

\subsection*{Elements and Subsets}

If $a$ is an element of the set $A$, we write $a \in A$, and if not, we write $a \not\in A$. If $A = \{\heartsuit, p, 2, 3, \diamondsuit\}$ then $\heartsuit \in A$ and $2 \in A$, but $1 \not\in A$. The set $S$ satisfying $p \in S$ and $q \in S$, but $x \not\in S$ for all other $x$ can be written as $S = \{p,q\}$.

\begin{example}
Let $S = \{2,3,5,7\}$ and $T = \{5,10,15\}$. Then $15 \not\in S$, but $15 \in T$.
\end{example}

If $A$ and $B$ are sets, and every element of $A$ is also an element of $B$, we say that $A$ is a \newterm{subset}\index{Subset} of $B$, and write $A \subseteq B$. Notice that every set is a subset of itself.

\begin{example}
Let $X = \{2,3,5,7,9\}$, $Y = \{2,5,6\}$, and $Z = \{2,7,9\}$. Then $Z \subseteq X$. However, $Y \not\subseteq X$ since $6$ is an element of $Y$, but it's not an element of $X$.
\end{example}

\begin{example}
Let $P = \{a,b,c,d\}$. Then $c \in P$ and $\{c\} \subseteq P$. Note that $\{c\} \not \in P$ (there are four elements of $P$ and $\{c\}$ is not one of them) and $c \not\subseteq P$ ($c$ is not a set, so cannot be a subset of anything).
\end{example}

One set that comes up often enough that it's deserving of its own name is the set with no elements, known as the \newterm{empty set}\index{Empty Set}, and denoted $\emptyset = \{\}$. When asked whether anything is in $\emptyset$, the answer is always no. Note that $\emptyset \subseteq A$ for any set $A$, since if not, there would have to be something in $\emptyset$ which is not an element of $A$, but $\emptyset$ has no elements, so the statement $\emptyset \subseteq A$ is vacuously true.

In some contexts, we also define a \newterm{universal set}\index{Universal Set} $U$ as the set of all objects in the universe we are considering. For example, if we're in a setting where the only objects being considered are real numbers, we can let $U = \mathbb{R}$.

\begin{example}
Consider a single roll of a six-sided die. Let $U = \{1,2,3,4,5,6\}$. If we define $L$ as the set of outcomes which are less than five, then $L = \{1,2,3,4\}$, and if we define $E$ as the set of outcomes which are even, then $E = \{2,4,6\}$.
\end{example}

One useful way to describe a set is with \newterm{set comprehension}\index{Set Comprehension}. This notation builds a new set out of a known one by selecting only those elements that satisfy a given condition. The condition can be specified in algebraic notation, in written language, or a mix of the two. The point is that the condition is clear and unambiguous to the reader.

\begin{example}
Let $S = \{1,2,3,4,5\}$, and define $T = \{x \in S \, | \, x > 3\}$. Then $T = \{4,5\}$.
\end{example}

\begin{example}
The set $\{\frac{p}{q} \in \mathbb{Q} \, | \, q=2 \text{ and $p$ is positive}\}$ is the set of positive half-integers. These are the positive numbers which are either a whole number, or halfway between two whole numbers.
\end{example}

\section{Operations on Sets}

Given two sets $A, B \subseteq U$, we'll define four fundamental operations, the \newterm{union}\index{Union} $A \cup B$, the \newterm{intersection}\index{Intersection} $A \cap B$, the \newterm{complement}\index{Complement} $A^c$, and the \newterm{difference}\index{Difference} $A \setminus B$.

\begin{center}
\begin{minipage}{1.6in}
\begin{center}
\begin{tikzpicture}[scale=0.8]
\draw (-1.5,-1.5) rectangle (3.0,1.5) node[below left]{$U$};
\draw (1.6,-1.75) node[below left]{$A \cup B$};
\begin{scope}                       % start of clip scope
\clip (0,0) circle (1cm);
\fill[gray] (0,0) circle (1cm);
\end{scope}                         % end of clip scope
\begin{scope}                       % start of clip scope
\clip (1.5,0) circle (1cm);
\fill[gray] (1.5,0) circle (1cm);
\end{scope}                         % end of clip scope
\draw (0,0) circle (1cm) node[] {$A$};
\draw (1.5,0) circle (1cm) node[] {$B$};
\end{tikzpicture}
\end{center}
\end{minipage}
\qquad\begin{minipage}{1.6in}
\begin{center}
\begin{tikzpicture}[scale=0.8]
\draw (-1.5,-1.5) rectangle (3.0,1.5) node[below left]{$U$};
\draw (1.6,-1.75) node[below left]{$A \cap B$};
\begin{scope}                       % start of clip scope
\clip (0,0) circle (1cm);
\fill[white] (0,0) circle (1cm);
\end{scope}                         % end of clip scope
\begin{scope}                       % start of clip scope
\clip (1.5,0) circle (1cm);
\fill[white] (1.5,0) circle (1cm);
\end{scope}                         % end of clip scope
\begin{scope}                       % start of clip scope
\clip (0,0) circle (1cm);
\fill[gray] (1.5,0) circle (1cm);
\end{scope}                         % end of clip scope
\draw (0,0) circle (1cm) node[] {$A$};
\draw (1.5,0) circle (1cm) node[] {$B$};
\end{tikzpicture}
\end{center}
\end{minipage}
\end{center}

\begin{center}
\begin{minipage}{1.6in}
\begin{center}
\begin{tikzpicture}[scale=0.8]
\begin{scope}                       % start of clip scope
\fill[gray] (-1.5,-1.5) rectangle (3.0,1.5);
\draw (1.3,-1.75) node[below left]{$A^c$};
\fill[white] (0,0) circle (1cm);
\end{scope}                         % end of clip scope
\draw (0,0) circle (1cm) node[] {$A$};
\draw (1.5,0) circle (1cm) node[] {$B$};
\draw (-1.5,-1.5) rectangle (3.0,1.5) node[below left]{$U$};
\end{tikzpicture}
\end{center}
\end{minipage}\qquad\begin{minipage}{1.6in}
\begin{center}
\begin{tikzpicture}[scale=0.8]
\begin{scope}                       % start of clip scope
\fill[gray] (0,0) circle (1cm);
\draw (1.3,-1.75) node[below left]{$A \setminus B$};
\fill[white] (1.5,0) circle (1cm);
\end{scope}                         % end of clip scope
\draw (0,0) circle (1cm) node[] {$A$};
\draw (1.5,0) circle (1cm) node[] {$B$};
\draw (-1.5,-1.5) rectangle (3.0,1.5) node[below left]{$U$};
\end{tikzpicture}
\end{center}
\end{minipage}
\end{center}

One can interpret $\cup$ as `or', $\cap$ as `and', and $^{c}$ as `not', but remember that $\cup$ refers to the \emx{inclusive or}, as in `to participate you must be older than 16 or have permission of a parent' (\emx{possibly both!}), not the \emx{exclusive or}, as in `the meal comes with a choice of soup or salad' (\emx{not both!}).

\begin{warning}
In colloquial language, the word `or' is ambiguous. The set $A \cup B$ is the set of all elements that are in $A$ or in $B$ or in both.
\end{warning}

The set difference $A \setminus B$ is what remains when all elements of $B$ are removed from the set $A$. Note that we can express this as $A \setminus B = A \cap B^c$ (the set of elements which are in $A$ \emx{and} \emx{not} in $B$), so the difference operation does not add any more expressive power to the language, since it can be defined in terms of the others. However, it's often a convenient shorthand.

\begin{example}
Let $U = \{p,q,r,s,t,u,v\}$, and take $A = \{p,q,r,s\}$ and $B = \{p,r,t,v\}$. Then we can compute $A \cap B = \{p,r\}$, $A^c = \{t,u,v\}$, and $(A \cup B)^c = (\{p,q,r,s,t,v\})^c = \{u\}$.
\end{example}

\begin{example}
Let $U = \{\heartsuit,\diamondsuit,\clubsuit,\spadesuit\}$, and take $S = \{\heartsuit,\diamondsuit,\clubsuit\}$ and $T = \{\heartsuit\}$. Then we can compute $S \setminus T = \{\diamondsuit,\clubsuit\}$ and $T \setminus S = \emptyset$.
\end{example}

\subsection*{The Algebra of Sets}

The set operations described above satisfy certain algebraic laws. For the most part, these laws should seem intuitively clear once you've had enough experience with the set operations, and for this reason, you should not attempt to memorize them. The exceptions, however, are the \newterm{distributive laws}\index{Distributive Laws}\ and \newterm{DeMorgan's laws}\index{DeMorgan's Laws}, which take some getting used to, and are not (at least to me and most others) immediate from the meaning of the operations.

\renewcommand{\arraystretch}{1.5}
\begin{center}
\begin{tabular}{@{} >{\bfseries}l >{\arraybackslash}m{6.5cm} @{}}

\hline

Identity & $A \cup \emptyset = A$, \quad $A \cap U = A$ \\

Domination & $A \cup U = U$, \quad $A \cap \emptyset = \emptyset$ \\

Idempotent & $A \cup A = A$, \quad $A \cap A = A$ \\

Complement & $A \cup A^c = U$, \quad $A \cap A^c = \emptyset$, \quad $(A^c)^c = A$ \\

Commutative & $A \cup B = B \cup A$, \quad $A \cap B = B \cap A$ \\

Absorption & $A \cup (A \cap B) = A$, \quad $A \cap (A \cup B) = A$ \\

Associative & $(A \cup B) \cup C = A \cup (B \cup C)$ \\ 
			& $(A \cap B) \cap C = A \cap (B \cap C)$ \\

Distributive & $A \cup (B \cap C) = (A \cup B) \cap (A \cup C)$ \\
             & $A \cap (B \cup C) = (A \cap B) \cup (A \cap C)$ \\

De Morgan's & $(A \cup B)^c = A^c \cap B^c$, \quad $(A \cap B)^c = A^c \cup B^c$ \\

\hline

\end{tabular} 
\end{center}

\begin{example}
Use laws of set algebra to simplify the expression $(A \cup (B \cap A^c)) \setminus B$.
$$\begin{aligned}
(A \cup (B \cap A^c)) \setminus B &= (A \cup (B \cap A^c)) \cap B^c \\
&= ((A \cup B) \cap (A \cup A^c)) \cap B^c \\
&= ((A \cup B) \cap U) \cap B^c \\
&= (A \cup B) \cap B^c \\
&= (A \cap B^c) \cup (B \cap B^c) \\
&= (A \cap B^c) \cup \emptyset \\
&= A \cap B^c \\
\end{aligned}$$
\end{example}

\begin{remark}
The example above might leave you with some questions. Why were the laws used in that particular order? What does it even mean for one set expression to be `simpler' than another?

Questions like these are addressed in a branch of mathematics known as Boolean algebra, which we're just scratching the surface of. Take the example as an illustration of how the laws above can be used in a productive way, and an invitation to think about some of the basic questions in Boolean algebra, the answers to which are beyond the scope of this course.
\end{remark}

\section{Inclusion-Exclusion Principle}\label{InclusionExclusionSec}

The number of elements in a set $A$ is known as its \newterm{cardinality}\index{Cardinality}, denoted $N(A)$ or $|A|$. Note that sets can have infinitely many elements, and we will write $N(A) = \infty$ when this occurs.

\begin{example}
Let $A = \{\heartsuit, \spadesuit, \diamondsuit\}$, and $B = \{\frac{1}{2}, \frac{1}{4},\frac{1}{8},\frac{1}{16}, \dots\}$. Then $N(A) = 3$ and $N(B) = \infty$.
\end{example}

There is only one set whose cardinality is zero, the empty set $\emptyset$. There are countless sets whose cardinality is infinite. The set $B$ in the example above, the natural numbers $\mathbb{N} = \{0,1,2,3,4,\dots\}$, the integers $\mathbb{Z} = \{\dots, -2, -1, 0 ,1 ,2, \dots\}$, and the real numbers $\mathbb{R}$ are examples you've encountered before.

It's important, and may come as a surprise, to know that some of these infinite sets have more elements than others. The sets $\mathbb{N}$, $\mathbb{Z}$, and $\mathbb{E} = \{2,4,6, \dots\}$ (the positive even numbers) all have the same number of elements, but the set $\mathbb{R}$ has more. The classic introduction to how this works is known as \newterm{Hilbert's Hotel}.

If a set has a larger cardinality than $\mathbb{N}$, we say it's \newterm{uncountable}. If it has a finite cardinality, or the same cardinality as $\mathbb{N}$, we say it's \newterm{countable}.

In the case where the cardinalities $N(A)$ and $N(B)$ are both finite, there is an important relationship between $N(A)$, $N(B)$, $N(A \cup B)$, and $N(A \cap B)$ that's very useful when calculating probabilities, which we'll get to in the next chapter.

\begin{theorem}
(Inclusion-Exclusion Principle) $N(A \cup B) = N(A) + N(B) - N(A \cap B)$.
\end{theorem}
\begin{proof} Consider the sum $N(A) + N(B)$. Every element in exactly one of the two sets $A$ and $B$ is counted once in this sum, but an element which is in both sets (that is, an element of $A \cap B$) is counted twice.

\begin{center}
\begin{tikzpicture}[scale=0.8]
\draw (-1.5,-1.5) rectangle (3.0,1.5) node[below left]{$U$};
\begin{scope}                       % start of clip scope
\clip (0,0) circle (1cm);
\fill[pattern=north west lines, pattern color=gray] (0,0) circle (1cm);
\end{scope}                         % end of clip scope
\begin{scope}                       % start of clip scope
\clip (1.5,0) circle (1cm);
\fill[pattern=north east lines, pattern color=gray] (1.5,0) circle (1cm);
\end{scope}                         % end of clip scope
\draw (0,0) circle (1cm) node[] {$A$};
\draw (1.5,0) circle (1cm) node[] {$B$};
\end{tikzpicture}
\end{center}

In $N(A \cup B)$, every element of $A \cup B$ (including those in $A \cap B$) is counted only once. To count the elements in $A \cap B$ twice, as above, we can sum $N(A \cup B) + N(A \cap B)$. Thus, $N(A \cup B) + N(A \cap B) = N(A) + N(B)$, and rearranging gives $N(A \cup B) = N(A) + N(B) - N(A \cap B)$.

\end{proof}

\begin{example}
How many cards in a standard deck (52 cards, 13 of each suit) are hearts or face cards?

If we let $U$ be the set of all cards in the deck, $H$ the set of all hearts, and $F$ the set of all face cards (the J, Q, and K of each suit), then $N(H \cap F) = 3$ (the J, Q, and K of hearts), and hence
$$N(H \cup F) = N(H) + N(F) - N(H \cap F) = 13 + 12 - 3 = 22.$$
\end{example}

The inclusion-exclusion principle can be extended to unions of three sets or more. In the case of three, if we write $N(A \cup B \cup C) = N(A) + N(B) + N(C)$ then we have double counted the outcomes in the lined areas in the Venn diagram below, and triple counted the outcomes in $A \cap B \cap C$ (the shaded central area).

\begin{center}
\begin{tikzpicture}[scale=0.7]
\begin{scope}                       % start of clip scope
\clip (-1,0.67) circle (1.5cm);
\fill[pattern=north west lines, pattern color=black] (1,0.67) circle (1.5cm);
\end{scope}                         % end of clip scope
\begin{scope}                       % start of clip scope
\clip (0,-1) circle (1.5cm);
\fill[pattern=north west lines, pattern color=black] (1,0.67) circle (1.5cm);
\end{scope}                         % end of clip scope
\begin{scope}                       % start of clip scope
\clip (0,-1) circle (1.5cm);
\fill[pattern=north west lines, pattern color=black] (-1,0.67) circle (1.5cm);
\end{scope}                         % end of clip scope
\begin{scope}                       % start of clip scope
\clip (-1,0.67) circle (1.5cm);
\clip (1,0.67) circle (1.5cm);
\fill[gray] (0,-1) circle (1.5cm);
\end{scope}                         % end of clip scope
\draw (-1,0.67) circle (1.5cm) node[above left] {$A$};
\draw (1,0.67) circle (1.5cm) node[above right] {$B$};
\draw (0,-1) circle (1.5cm) node[below] {$C$};
\draw (-3.0,-3.0) rectangle (3.0,3.0) node[below left]{$U$};
\end{tikzpicture}
\end{center}

To correct for double counting the lined areas, we can subtract the cardinalities of $A \cap B$, $B \cap C$, and $C \cap A$, but each of these includes $A \cap B \cap C$. Elements of this triple intersection are counted three times and subtracted out three times. Thus, we need to add them back in to count them once. All together we obtain the formula below.
$$\begin{aligned}
N(A \cup B \cup C) &= N(A) + N(B) + N(C)\\ &\  - N(A \cap B) - N(B \cap C) - N(C \cap A) \\ & \ + N(A \cap B \cap C)
\end{aligned}$$

\begin{example}
At a daycare, 24 children will eat Apples, 28 children will eat Bananas, and 30 children will eat Clementines. If 18 will eat Apples \& Bananas, 22 will eat Bananas \& Clementines, 19 will eat Apples \& Clementines, and 14 will eat all three fruits, how many children will eat at least one of the three?
$$\begin{aligned}
N(A \cup B \cup C) &= N(A) + N(B) + N(C)\\ &\  - N(A \cap B) - N(B \cap C) - N(C \cap A) \\ & \ + N(A \cap B \cap C) \\
&= 24 + 28 + 30 - 18 - 22 - 19 + 14 \\
&= 37
\end{aligned}$$
\end{example}

Notice that when we apply the inclusion-exclusion principle, we can solve for any of the terms. If we were given the cardinality of the union of all three sets, we could instead solve for the cardinality of the intersection of all three sets.

\begin{example}
At a daycare, 24 children will eat Apples, 28 children will eat Bananas, and 30 children will eat Clementines. If 18 will eat Apples \& Bananas, 22 will eat Bananas \& Clementines, 19 will eat Apples \& Clementines, and 37 will eat at least one of the three fruits, how many children will eat all three?
$$\begin{aligned}
N(A \cup B \cup C) &= N(A) + N(B) + N(C)\\ &\  - N(A \cap B) - N(B \cap C) - N(C \cap A) \\ & \ + N(A \cap B \cap C) \\
37 &= 24 + 28 + 30 - 18 - 22 - 19 + N(A \cap B \cap C) \\
14 &= N(A \cap B \cap C)
\end{aligned}$$
\end{example}

These kind problems can also be done by filling in the regions of the Venn diagram carefully with the correct counts, while being very careful to never double count. In many cases, this approach is faster and more natural then writing out the algebraic form of the law.

\begin{example}
At a daycare, 24 children will eat Apples, 28 children will eat Bananas, and 30 children will eat Clementines. If 18 will eat Apples \& Bananas, 22 will eat Bananas \& Clementines, 19 will eat Apples \& Clementines, and 14 will eat all three fruits, how many children will eat only apples, and neither of the other two fruits?

We'll start with the count in the centre, 14, and move from the inside out. 

\begin{center}
\begin{tikzpicture}[scale=0.7]
\draw (-2.75,1.5) node {$A$};
\draw (2.75,1.5) node {$B$};
\draw (1.75,-2) node {$C$};
\draw (-1,0.67) circle (1.5cm) node[above left] {$?$};
\draw (1,0.67) circle (1.5cm) node[above right] {$?$};
\draw (0,-1) circle (1.5cm) node[below] {$?$};
\draw (0,0.1) node {$14$};
\draw (0,1) node {$?$};
\draw (-0.75,-0.3) node {$?$};
\draw (0.75,-0.3) node {$?$};
\end{tikzpicture}
\end{center}

Since $N(A \cap B) = 18$, and the 14 children in the centre will all eat Apples and Bananas, the region directly above the centre has 4 children in it, that is, there are 4 children who will eat Apples and Bananas but not Clementines ($A \cap B \cap C^c$). Similarly, since $N(A \cap C) = 19$ and $N(A \cap B \cap C) = 14$, there are 5 children who will eat Apples and Clementines but not Bananas ($A \cap B^c \cap C$).

\begin{center}
\begin{tikzpicture}[scale=0.7]
\draw (-2.75,1.5) node {$A$};
\draw (2.75,1.5) node {$B$};
\draw (1.75,-2) node {$C$};
\draw (-1,0.67) circle (1.5cm) node[above left] {$?$};
\draw (1,0.67) circle (1.5cm) node[above right] {$?$};
\draw (0,-1) circle (1.5cm) node[below] {$?$};
\draw (0,0.1) node {$14$};
\draw (0,1) node {$4$};
\draw (-0.75,-0.3) node {$5$};
\draw (0.75,-0.3) node {$?$};
\end{tikzpicture}
\end{center}

Finally, $N(A) = 24$, and three of the four regions that make up $A$ have been filled out, and total $14+4+5 = 23$ children. Thus, only a single child eats Apples and none of the other fruits ($A \cap B^c \cap C^c$). 

\begin{center}
\begin{tikzpicture}[scale=0.7]
\draw (-2.75,1.5) node {$A$};
\draw (2.75,1.5) node {$B$};
\draw (1.75,-2) node {$C$};
\draw (-1,0.67) circle (1.5cm) node[above left] {$1$};
\draw (1,0.67) circle (1.5cm) node[above right] {$?$};
\draw (0,-1) circle (1.5cm) node[below] {$?$};
\draw (0,0.1) node {$14$};
\draw (0,1) node {$4$};
\draw (-0.75,-0.3) node {$5$};
\draw (0.75,-0.3) node {$?$};
\end{tikzpicture}
\end{center}

\end{example}

\begin{keypoint}
Always fill in the Venn diagram from the inside out. Even if the quantity you're solving for is in the center, call it $x$ and start filling in from the center, moving out.
\end{keypoint}

\begin{example}
In a group of 45 people, 31 of them went swimming, 26 of them went canoeing, and 7 of them did neither activity. How many did both activities?

\begin{center}
\begin{tikzpicture}[scale=0.7]
\draw (-1.6,1.25) node {$S$};
\draw (3.5,-1.25) node {$C$};
\draw (-2,-2) rectangle (4.0,2);
\draw (0,0) circle (1.5cm);
\draw (2,0) circle (1.5cm);
\draw (1,0) node[] {$x$};
\draw (3.5,1.5) node[] {$7$};
\draw (-0.5,0) node[] {$31-x$};
\draw (2.5,0) node[] {$26-x$};
\end{tikzpicture}
\end{center}

Let $x$ be the number of people that did both activities. If we label the Venn diagram as above and sum the values in all four regions, then $(31-x) + x + (26-x) + 7 = 45$. Solving yields $x = 19$.

Alternatively, we could use the algebraic form of the inclusion-exclusion principle, after noting that since 7 people did neither activity, $N((S \cup C)^c) = 7$, and hence $N(S \cup C) = 45 - 7 = 38$. Now applying the inclusion-exclusion principle,
$$\begin{aligned}
N(S \cup C) &= N(S) + N(C) - N (S \cap C) \\
38 &= 31 + 26 - N (S \cap C) \\
N (S \cap C) &= 31 + 26 - 38 \\
N (S \cap C) &= 19
\end{aligned}$$

\end{example}

\section{Counting Principles \& Factorials}

Suppose you order fries and root beer at a fast food restaurant. There are two options for the size of the fries and three options for the size of the root beer.

Since there are two options for the size of the fries, and three options for the size of the root beer, you have a total of six possible choices when you order both. These are enumerated below, the first letter in each ordered pair denoting the size of the fries (small or large), and the second letter the size of the root beer (small, medium, or large).
$$(S,s)\ \ (S,m)\ \ (S,l)\ \ (L,s)\ \ (L,m)\ \ (L,l)$$
These are exactly the elements of the cartesian product $F \times R$, where $F = \{S,L\}$ and $R = \{s,m,l\}$. 

If it turns out that, unfortunately, there's only enough change in your pocket to pay for a single item (which could be any size), then you only have five choices. Either you only order fries, and choose an element from $F$, or only order root beer, and choose an element from $R$. This is equivalent to choosing an element of $F \cup R = \{S,L,s,m,l\}$.

To summarize, the number of ways to choose an element from $F$ \emx{and} an element from $R$ is $|F| \cdot |R|$, while the number of ways to choose an element from $F$ \emx{or} an element from $R$ is $|F|+|R|$. This idea generalizes considerably, and forms the foundational principle that will allow us to solve many problems.

\begin{proposition}(Fundamental Counting Principle) \label{FundamentalCountingPrinciple}\index{Fundamental Counting Principle} In a sequence of $k$ choices, if the first choice can be made in $n_1$ ways, the second choice can be made in $n_2$ ways, and so on, the number of ways of making all choices is the product $n_1 n_2 \,\cdots\, n_k$, while the number of ways of selecting one of the choices to make and making only that choice is the sum $n_1 + n_2 + \,\cdots\, + n_k$.
\end{proposition}

\begin{example}
How many ways are there to select three students to be class representatives in three different classes, one with twenty students and two with thirty?

There are three choices to be made, choosing a representative for each of the three classes, and since we must make all three choices, the number of ways this can be done is $20 \cdot 30 \cdot 30 = 18\,000$.
\end{example}

\begin{example}
How many four symbol sequences can be made with the symbols in $S = \{\heartsuit, \diamondsuit, \clubsuit, \spadesuit\}$?

We can imagine filling in four blank spaces with symbols, and write in each blank space the number of choices we have for the symbol that gets placed there. If we allow the same letter to be used many times, then the number of sequences is $\underline{4} \cdot \underline{4} \cdot \underline{4} \cdot \underline{4} = 256$.

On the other hand, if each space must be filled with a distinct symbol, then initially there are four choices for the first symbol, but after making that choice, there are three remaining for the second symbol, and so on. In total then, the number of sequences with distinct symbols is $\underline{4} \cdot \underline{3} \cdot \underline{2} \cdot \underline{1} = 24$. 
\end{example}

This second result is the number of ways of arranging four different symbols in a line, and counting arrangements is something we'll do frequently, so it's convenient to have a shorthand for this kind of descending product of consecutive integers.

\begin{definition}\index{Factorial}
Given $n \in \mathbb{N}$, the \newterm{factorial} of $n$, denoted $n!$, is the number of ways of arranging $n$ distinct objects in a line, and is defined by
$$n! = n \cdot (n-1) \cdot (n-2) \cdot \, \cdots \, \cdot 2 \cdot 1$$
\end{definition}

\begin{note} Though the usual interpretation of the factorial in terms of counting arrangements no longer applies, we define $0!=1$ for convenience. This makes many formulas involving factorials (like the Taylor series formula) much easier to write.
\end{note}

Factorials have some very nice algebraic properties that make them easy to work with, in particular, ratios of factorials can be simplified easily.
\eqnsgap{\frac{8!}{6!} = \frac{8 \cdot 7 \cdot 6 \cdot 5 \cdot 4 \cdot 3 \cdot 2 \cdot 1}{6 \cdot 5 \cdot 4 \cdot 3 \cdot 2 \cdot 1}= \frac{8 \cdot 7 \cdot \cancel{6} \cdot \cancel{5} \cdot \cancel{4} \cdot \cancel{3} \cdot \cancel{2} \cdot \cancel{1}}{\cancel{6} \cdot \cancel{5} \cdot \cancel{4} \cdot \cancel{3} \cdot \cancel{2} \cdot \cancel{1}} = 56}
\par

%Factiorial algebra examples

\section{Permutations \& Combinations}

\subsection*{Permutations}\label{PermutationsSec}

\begin{example}
How many ways are there to choose three students from a class of twenty and arrange them in a line?

We can use the same approach as in the last example. We're filling in three blanks, we have twenty options to choose from, and repeats are not permitted. Thus, there are $\underline{20} \cdot \underline{19} \cdot \underline{18} = 6\,840$ ways to do it. 

\noindent The notation $^{20}P_3$ is used to refer to the number of ways of selecting $3$ objects from a set of $20$ distinct options and ordering them. Thus, we have $^{20}P_3 = 6\,840$.
\end{example}

\begin{definition}\label{permutationdefinition}\index{Permutations}
If an ordered list of $r$ distinct objects is selected from a set of $n$ objects, the result is called an $r$-\newterm{permutation}. The number of $r$-permutations of $n$ objects, denoted by $^{n}P_r$, is given by
$$^nP_r = n \cdot (n-1) \cdot \, \cdots \, \cdot (n-r+1) = \frac{n!}{(n-r)!}$$
\end{definition}

\begin{remark}
We can write the formula for permutations conveniently as above, but it's usually more helpful in problems to think about the fundamental counting principle directly, and to interpret the formula for permutations as a truncated factorial (a factorial with the end cut off).
\end{remark}

\begin{example}
How many ways are there to create a sequence of five letters with no repetitions using any of the 26 lower case letters of the alphabet?

Taking $5$ letters from a set of $26$ and arranging them, the number of options is
$$^{26}P_5 = \frac{26!}{(26-5)!} = \frac{26!}{21!} = 26 \cdot 25 \cdot 24 \cdot 23 \cdot 22 = 7\,893\,600$$
\end{example}

%More examples

\subsection*{Permutations with Identical Objects}

What if some objects are indistinguishable? If this is the case, there are fewer possibilities. For example, how many arrangements of the letters in the string $AAAB$ are there? 

Notice that we just have to decide which of the four possible positions we'll place the $B$ into. Once that's done, then $A$'s go everywhere else, so there are only four distinct arrangements.

Now suppose that we number the three $A$s. How many rearrangements of the string $A_1A_2A_3B$ are there? Since the four letters are now distinguishable, there are $4! = 24$. Let's write these out, grouping together in columns all strings that will be identical after the labels on the $A$'s are removed.
\begin{center}
\begin{tabular}{cccc}
$A_1A_2A_3B$ & $A_1A_2BA_3$ & $A_1BA_2A_3$ & $BA_1A_2A_3$ \\
$A_1A_3A_2B$ & $A_1A_3BA_2$ & $A_1BA_3A_2$ & $BA_1A_3A_2$ \\
$A_2A_1A_3B$ & $A_2A_1BA_3$ & $A_2BA_1A_3$ & $BA_2A_1A_3$ \\
$A_2A_3A_1B$ & $A_2A_3BA_1$ & $A_2BA_3A_1$ & $BA_2A_3A_1$ \\
$A_3A_1A_2B$ & $A_3A_1BA_2$ & $A_3BA_1A_2$ & $BA_3A_1A_2$ \\
$A_3A_2A_1B$ & $A_3A_2B_1A$ & $A_3BA_2A_1$ & $BA_3A_2A_1$ \\
\end{tabular}
\end{center}
Notice that rearranging the labels will result in a string which is identical once the labels are removed. This means if we list all possible arrangements of the labeled strings, each possible unlabeled string will appear $3!$ times, since there are $3!$ ways of arranging the labels. Alternatively, think about how there are 3! ways to \emx{apply the labels} to the three $A$'s, and each of the 24 labeled strings can be obtained by first selecting an unlabeled string (i.e. a column) then applying the labels.

The upshot is that \emx{we can count the number of unlabeled strings by dividing the number of labeled strings by the number of ways of rearranging the labels}, in this case, $4! / 3! = 4$. The same principle applies if there are many different kinds of indistinguishable objects, and the general result can be summarized in the formula below.

\begin{proposition}\index{Mississippi Formula} (Mississippi Formula) The number of ways of rearranging $n$ objects, with indistinguishable groups of sizes $k_1, k_2, \dots\, ,k_m$, is given by
$$\frac{n!}{k_{1}! \cdot k_{2}! \cdot \, \cdots \, \cdot k_{m}!}$$
\end{proposition}

\begin{example}
How many arrangements of the letters in MISSISSIPPI are there?
\par
\noindent There are eleven letters in total, with identical groups of size 4 (the I's), 4 (the S's), and 2 (the P's). Therefore, the number of arrangements is
\eqnsbiggap{\frac{11!}{4! \cdot 4! \cdot 2!} = \frac{11 \cdot 10 \cdot 9 \cdot {8} \cdot 7 \cdot {6} \cdot 5 \cdot {4} \cdot {3} \cdot {2} \cdot 1}{{4} \cdot {3} \cdot {2} \cdot 1 \cdot {4} \cdot {3} \cdot {2} \cdot 1 \cdot {2} \cdot 1} = 11 \cdot 10 \cdot 9 \cdot 7 \cdot 5 = 34\,650.}
\end{example}

\begin{example}
An urn contains four green and five red marbles, which are removed one at a time. How many different sequences of red and green marbles can result from this process?
\par
\noindent In this case, we are rearranging nine objects, four $G$'s and five $R$'s. Each arrangement, read left to right, corresponds to a sequence of marbles that could appear. There are identical groups of size 4 (the green marbles), and 5 (the red marbles). Therefore, the number of arrangements is
\eqnsbiggap{\frac{9!}{4! \cdot 5!} = \frac{9 \cdot {8} \cdot 7 \cdot {6} \cdot 5 \cdot {4} \cdot {3} \cdot {2} \cdot 1}{{4} \cdot {3} \cdot {2} \cdot 1 \cdot {5} \cdot {4} \cdot {3} \cdot {2} \cdot 1} = 9 \cdot 2 \cdot 7 = 126.}
\end{example}

\subsection*{Combinations}

Suppose now we're interested in the number of ways that a subset of $r$ objects can be selected from a larger set of $n$ distinct objects. We're only concerned with which objects are selected not with their order, so if the same objects appear, but in a different order, we consider this to be the same outcome.

Consider how many ways there are to selecta group three children from a family with five. To do this, we can imagine lining up the children any fixed order, let's say by age. We can now specify which children are chosen using a string such as $YYNYN$, which would indicate we chose the first two children in the line as well as the fourth. In fact, every distinct five letter string made up of $Y$s and $N$s which has exactly three $Y$s will represent a different choice of three children, and every choice of three children determines such a string.

Now we can use the Mississippi formula to count the number of different choices by counting strings. Each string is a rearrangement of five objects with identical groups of sizes three and two, hence there are $\frac{5!}{3!2!} = 5 \cdot 2 = 10$ possibilities.

The general principle is that once the elements of a set have been lined up in some arbitrary order, each subset is given by an \emph{ordered} list of $Y$'s and $N$'s, where $Y$ denotes membership in the subset, and $N$ denotes non-membership. We can count the number of these ordered lists of $Y$'s and $N$'s using the Mississippi formula.

\begin{definition}\index{Combinations}\label{combinationdefinition}
If a subset of $r$ objects is selected from a set of $n$ objects, the result is called an $r$-combination. The number of $r$-\newterm{combinations}, denoted $^nC_r$, or more commonly $\binom{n}{r}$, is given by
$$^nC_r = \binom{n}{r} = \frac{n!}{r!(n-r)!}$$
\end{definition}

\begin{warning}
Note carefully the distinction between $^nP_r$ and $^nC_r$. The former counts the number of ways of selecting $r$ objects \emx{and arranging them}, while the latter is the number of selections possible when \emx{any rearrangement is regarded as the same outcome}.
\end{warning}

\begin{example}
In Lotto 6/49, if you purchase a ticket you choose 6 distinct numbers between 1 and 49. The ticket wins if its numbers match 6 distinct numbers between 1 and 49 which are randomly drawn, regardless of the order they are drawn in. How many different tickets would you need to buy to guarantee a win?

Since the order of the numbers on the ticket is irrelevant, the number of different tickets available is $\binom{49}{6} = \frac{49!}{6!43!} = 13\,983\,816$, so this is the minimum number you would need to purchase to be guaranteed one will win.
\end{example}

\begin{example}
How many ways are there to choose a president, a treasurer, and an organizing committee of three from a group of 25 people? Assume no person can hold more than one position.

There are 25 choices for president, then 24 for treasurer, then $\binom{23}{3}$ for the organizing committee. Note that when choosing the organizing committee, the order in which members are chosen is irrelevant. The committee consisting of Alice, Bob, and Chelsea and the committee consisting of Chelsea, Alice, and Bob are the same. We must make all three choices, so by the fundamental counting principle the number of possibilities is
$$25 \cdot 24 \cdot \binom{23}{3} = 1\,062\,600.$$

Notice that we could just as well have proceeded in the opposite order, that is, first chosen the organizing committee, then the treasurer, then the president, and still obtained
$$\binom{25}{3} \cdot 22 \cdot 21 = 1\,062\,600.$$
\end{example}

\begin{example}How many ways are there to divide a group of eight people into a team of three and a team of five?

To create a team of three and a team of five, we choose three individuals for the team of three, then the five remaining make up the team of five.
$$\binom{8}{3}\binom{5}{5} = 56$$
\end{example}

\begin{example}
How many ways are there to divide a group of eight people into two teams of four?

To create two teams of four, we choose four for the first team, and those that remain make up another team of four. We have to be very careful though. In the example above the two teams were distinguishable (one has more people than the other), but here they are not. 

If we select A, B, C, and D for the first team and E, F, G, and H for the second, this results in the same outcome as selecting E, F, G, and H for the first team and A, B, C, and D for the second. Because the teams are indistinguishable, every outcome has been double counted! Thus, we need to divide out by the number of rearrangements of the teams.
$$\frac{\binom{8}{4}\binom{4}{4}}{2!} = 35$$

Notice that if the two teams were distinguishable, for example, if there were a red team and blue team with four people on each, there would be $\binom{8}{4}\binom{4}{4} = 70$ ways.

\begin{remark}
As the last example shows, it's very easy to end up inadvertently double counting, even for those with lots of experience with these kinds of problems. Solving such a problem amounts to coming up with a sequence of choices which can be made to produce each possible outcome. You need to convince yourself that your sequence of choices can not only produce every possible outcome, but produce it in a unique way (making any of the choices differently results in a different outcome).
\end{remark}
\end{example}




% Modified 31 Oct 2005:  Conditioning fallacy alluded to.
% This chapter has been modified on 6-4-05.
% There are two \choice
\pagestyle{headings}
\chapter{Probability} \label{chp 3}

\section{Sample Spaces \& Events}

Suppose the outcome of an experiment is uncertain, but we can represent the collection of all possible outcomes by writing it as a set.

\begin{definition} A set which represents the collection of all possible outcomes of an experiment is known as a \newterm{sample space}\index{Sample Space}, typically denoted $\Omega$.
\end{definition}

Specifying a concrete representation of the set of all outcomes of an experiment will often clarify the context of a problem and eliminate many potential sources of confusion.

\begin{example}Consider a single roll of a single six-sided die. In this experiment, there are six possible outcomes, corresponding to the six distinct faces of the die. Thus, we can take $\Omega = \{1,2,3,4,5,6\}$.
\end{example}

\begin{example}\label{BreakingStrength}Consider measuring the breaking strength (in Newtons) of a plank of wood. If we have no information about the size or shape of the plank, or about the precision with which the breaking strength will be measured, we should allow any positive real value. Thus, $\Omega = \{x \in \mathbb{R} \, | \, x \geq 0\}$.\end{example}

\begin{example}Suppose that a coin is flipped, and then a marble is drawn from an urn which contains black, white, and grey marbles. In this scenario, we could take $\Omega = \{HB,HW,HG,TB,TW,TG\}$, using $HB$ to represent flipping a head and drawing a black marble, $HW$ to represent flipping a head and drawing a white marble, and so on. \end{example}

\begin{example}\label{FlipUntilTail}Suppose you play a game where a coin is flipped repeatedly until the first tail appears. We could take $\Omega = \{T, HT, HHT, HHHT, HHHHT, \, \dots\, \}$, the set of all finite sequences of zero or more $H$'s followed by a single $T$.  \end{example}

As a sample space is a set, it can be countable or uncountable. We'll see later on in the course that the tools of calculus will come to our aid when dealing with probability problems that involve uncountable sample spaces.

The choice of sample space depends on the experiment and on what question the experimenters are trying to answer. If we roll a pair of dice, we might use the sample space $\Omega = \{2,3,4,\,\dots\,,11,12\}$ to count the sum of the two resulting values, or $\Omega = \{(1,1),(1,2),(2,1),(2,2),\,\dots\,,(5,6),(6,5),(6,6)\}$ to distinguish the dice and represent the value on each.

One big advantage of this larger sample space is that each of the outcomes is \emx{equally likely}. Imagine one die is red and the other is blue (the colours of the dice certainly won't influence the probabilities in this context). If neither die has any bias, then when the pair is rolled, each die is equally likely to land with any of its faces up. The red die showing 2 and blue showing 5 is as likely as the red die showing 6 and blue showing 6, and so on. In other words, each ordered pair of outcomes $(m,n)$ for $1 \leq m \leq 6$ and $1 \leq n \leq 6$ is as likely as any other.

If we write the number showing on the red die across the top row, the number showing on the blue die down the left column, and the sum of the two results in a table, we obtain the following.
{\small
\begin{center}\label{DiceRollTable}
\begin{tabular}{c|cccccc}
$_{B} \setminus ^R$ & 1 & 2 & 3 & 4 & 5 & 6 \\
\hline
1 & 2 & 3 & 4 & 5 & 6 & 7 \\
2 & 3 & 4 & 5 & 6 & 7 & 8 \\
3 & 4 & 5 & 6 & 7 & 8 & 9 \\
4 & 5 & 6 & 7 & 8 & 9 & $\!$10 \\
5 & 6 & 7 & 8 & 9 & $\!$10 & $\!$11 \\
6 & 7 & 8 & 9 & $\!$10 & $\!$11 & $\!$12 \\
\end{tabular}
\end{center}
}
Note that outcomes in the sample space $\Omega = \{2,3,4,\,\dots\,,11,12\}$ are not equally likely. There are six ordered pairs $(1,6)$, $(2,5)$, $(3,4)$, $(4,3)$, $(5,2)$, and $(6,1)$ which yield a sum of seven, while only the single pair $(1,1)$ gives a sum of two. This means rolling a sum of seven is six times more likely than rolling a sum of two.

\subsection*{Events in a Sample Space}

\begin{definition}
\newterm{Events}\index{Event} are subsets of a sample space in which we might be interested. We can describe an event using words or with more formal set notation.
\end{definition}

\begin{example}
Suppose we roll a single die. Let $\Omega = \{1,2,3,4,5,6\}$, let $E$ be the event `an even number is rolled' and let $G$ be the event `a number greater than four is rolled'. Then we can write $E = \{2,4,6\}$ and $G = \{5,6\}$.
\end{example}

\begin{example}\label{UnitSquare}
Consider a point selected at random inside a square with corners at $(0,0)$, $(1,0)$, $(0,1)$, and $(1,1)$. Let $\Omega = \{(x,y) \, | \, 0 \leq x \leq 1$ and $0 \leq y \leq 1\}$, and let $T$ be the event `the point is in the top half of the square'. Then using set notation we would write $T = \{(x,y) \, | \, 0 \leq x \leq 1$ and $0.5 \leq y \leq 1\}$.
\end{example}

Since events $A, B \subseteq \Omega$ are sets, we can form the events $A \cup B$ (at least one of $A$ and $B$ occur), $A \cap B$ (both $A$ and $B$ occur), and $A^{c}$ ($A$ does not occur). Te set operations allow us to build up complex events out of simpler pieces, or vice-versa, break a complex event into simpler pieces that are easier to deal with.

%Simple example using the operations with interpretation.

Let's return for a moment to rolling a pair of dice. If one of the dice shows a five, is it possible the sum is seven? Yes, since the outcomes $(5,2)$ and $(2,5)$ are elements of both events $F =$ `one of the dice shows five' and $S =$ `the sum is seven', that is, $F \cap S = \{(5,2), (2,5)\}$.

If one of the dice shows a five, is it possible the sum is three? No. If one die shows a five, the lowest possible sum is six. There are no elements common to the events $F =$ `one of the dice shows five' and $T =$ `the sum is three', that is, $F \cap S = \emptyset$.

\begin{definition}\label{mutuallyexclusive}\index{Mutually Exclusive Events}\index{Events!Mutually Exclusive}
Two events $A$ and $B$ in a sample space $\Omega$ are \newterm{mutually exclusive} if $A \cap B = \emptyset$, that is, if there are no outcomes in $\Omega$ where both events occur.
\end{definition}

\begin{example}
Consider an experiment where a coin is flipped, and then a die is rolled, and let $\Omega = \{H1,H2,H3,H4,H5,H6,T1,T2,T3,T4,T5,T6\}$. The events $H =$ `a head is flipped' and $E =$ `an even number is rolled' are not mutually exclusive, since $H \cap E = \{H2, H4, H6\}$.
\end{example}

\begin{example}
Suppose you're waiting for a bus, and you measure the number of seconds it takes for the bus to arrive. Let $E =$ `the bus arrives in under 3 minutes' and $F =$ `the bus takes more than 5 minutes to arrive'. Then $E$ and $F$ are mutually exclusive. Formally, let $\Omega = \{ x \in \mathbb{R} \, | \, x \geq 0\}$. In interval notation, we can write $E = [0,180)$ and $F = (300,\infty)$, then clearly $E \cap F = \emptyset$.
\end{example}

\section{Probability Models \& Probability Laws}

Consider a single roll of a fair six-sided die, and let $\Omega = \{1,2,3,4,5,6\}$. Since the die is fair, we should assign each of the six outcomes an equal probability of one-sixth.

If we want to find the probability any event $E \subseteq \Omega$ occurs, we can simply sum up the probabilities of each of the outcomes. If $E = \{2,4,6\}$ (an even number is rolled), then we write $P(E) = \frac{3}{6}$. To assign a probability to any event in this sample space, we can simply count up the number of outcomes in the event, and divide that number by six to obtain its probability.

\begin{definition}
A sample space $\Omega$, together with a way of assigning probabilities to all events $E \subset \Omega$, is known as a \newterm{probability model}\index{Probability Model}.
\end{definition}

\begin{example}
Describe a probability model on the sample space $\Omega = \{1,2,3,4,5,6\}$ for a six-sided die where each outcome is equally likely except for six, which is twice as likely as all the other outcomes.

Let $P(\{1\}) = q$. Then $P(\{6\}) = 2q$, and since it's certain that one of the six outcomes will occur,
$$\begin{aligned}
P(\{1,2,3,4,5,6\}) &= 1 \\
P(\{1\})+P(\{2\})+ \dots + P(\{5\})+P(\{6\}) &= 1 \\
q + q + \dots + q + 2q &= 1 \\
7q &= 1 \\
q &= \textstyle\frac{1}{7} \\
\end{aligned}$$

Therefore, the desired probability model assigns the outcomes one through five the probability $\frac{1}{7}$, and the outcome six the probability $\frac{2}{7}$.
\end{example}

\begin{keypoint}
For a \emx{countable} sample space, a probability model is an assignment of probabilities to the elements of $\Omega$. Once this is done, the probability of any event can be computed by summing the probabilities of the outcomes it contains. For an uncountable sample space, the situation is more subtle, as you'll see in the following example.
\end{keypoint}

\begin{example}\label{RandomPointInSquare}
Let $\Omega = \{(x,y) \,|\, 0 \leq x \leq 1 \text{ and } 0 \leq y \leq 1\}$, a unit square in $\mathbb{R}^2$. How can we describe a probability model in which a point in this unit square is selected at random?

Consider the smaller square $E = \{(x,y) \,|\, 0 \leq x \leq \frac{1}{2} \text{ and } 0 \leq y \leq \frac{1}{2}\}$ shaded below. Since this smaller square occupies one-quarter of the area, we should certainly have $P(E) = \frac{1}{4}$. 

\begin{center}
\begin{minipage}{1.6in}
\begin{center}
\begin{tikzpicture}[scale=0.7]
\draw (-1.5,-1.5) rectangle (1.5,1.5);
\draw (-1.5,-1.5) rectangle (0,0);
\fill [pattern=north east lines, pattern color=black] (-1.5,-1.5) rectangle (0,0);
\end{tikzpicture}
\end{center}
\end{minipage}
\end{center}
\vspace{0pt}

This probability shouldn't change if we move the smaller square around inside the larger one, or even if we cut it into pieces and move them around, as long as we don't add or remove any area. This leads us a to a simple realization: in this model, the probability of any event is its area. Note that the area of $\Omega$ is one since it's a unit square. If not, we would take $P(E)$ to be the proportion of the area of $\Omega$ that $E$ occupies.

So the probability model has a very simple description. Each event $E \subseteq \Omega$ is assigned a probability equal to its area.
\end{example}

\begin{remark}
There are some very interesting subtleties of probability models on uncountable sample spaces that the previous example gives us an opportunity to discuss.

Firstly, each element of $\Omega$ is a point, with no area, and hence an event such as $E=\{(\frac{1}{3},\frac{2}{3})\}$ is assigned the probability zero. Every event which contains only a single point has probability zero, but one of them will occur. The implication is that events with probability zero can sometimes occur. This is indeed the way things work in the world of uncountable sample spaces.

%Even events which contain infinitely many elements of $\Omega$, like $D = \{(x,x) \, | \, 0 \leq x \leq 1\}$ (a diagonal line through the square), can have no area and hence probability zero.

Secondly, it's not clear whether we've actually described the probability model completely. Consider an event like $Q = $ `the point selected has rational coordinates'. This is an event which could occur, and a well-defined subset of $\Omega$, but what is the area of $Q$? This is a deep question which ultimately leads to a branch of mathematics called measure theory, discussed in more advanced courses in mathematics.
\end{remark}

\subsection*{Kolmogorov's Axioms}

A probability model assigns probabilities to events, but not every kind of assignment is permitted. If would be absurd, for example, to create a probability model on the sample space $\Omega = \{H, T\}$ where $P(\{H\}) = 0.6$ and $P(\{T\}) = 0.9$. 

\begin{definition}
A probability model on $\Omega$ is an assignment of probabilities to events that satisfies
\begin{itemize}
\item Any event $E \subseteq \Omega$ has $0 \leq P(E) \leq 1$.
\item $P(\Omega) = 1$.
\item If $E_1, E_2, \,\dots$ are mutually exclusive events, then $P(E_1 \cup E_2 \cup \dots) = P(E_1) + P(E_2) + \cdots$
\end{itemize}
\end{definition}

These three requirements are known as the Kolmogorov axioms for probability, after Russian mathematician Andrey Kolmogorov. Note that the sequence of events in the third axiom may be finite, or countably infinite. For this reason, it's known as the countable additivity axiom. 

With two mutually exclusive events $A$ and $B$, the third axiom states $P(A \cup B) = P(A) + P(B)$, that is, we can compute the probability of the union by summing the two probabilities. In general, whenever any finite or countably infinite number of events in $\Omega$ do not overlap (in the sense of a Venn diagram), the countable additivity axioms says the probability of their union is the sum of their probabilities.

The axioms are important because every part of the theory of probability presented in this chapter can be derived from them, so they are the pillars on which the theory of probability is built. Anyone who doubts the correctness of any part of the theory can always follow a chain of reasoning back to these three statements. Consider, for example, the proposition and justification below.

%\begin{remark}
%The second axiom actually follows from the first and third. It's a good exercise to prove this for yourself (\emph{Hint}: Read the next example, and note that $E$ and $E^c$ are always mutually exclusive events with $E \cup E^c = \Omega$).
%\end{remark}

%\begin{example}
%Use the Kolmogorov axioms to show that $P(\emptyset) = 0$ in any probability model. 

%Note that since $\Omega \cap \emptyset = \emptyset$, the events $\Omega$ and $\emptyset$ are mutually exclusive. Thus, by countable additivity,
%$$P(\Omega) = P(\Omega \cup \emptyset) = P(\Omega) + P(\emptyset),$$
%but since $P(\Omega) = 1$, the equation above says $1 = 1 + P(\emptyset)$, hence $P(\emptyset) = 0$.
%\end{example}

\begin{proposition}
$P(E^c) = 1 - P(E)$ for any event $E$.
\end{proposition}
\begin{proof}
The events $E$ and $E^c$ are mutually exclusive, and $E \cup E^c = \Omega$. Thus, by countable additivity,
$$P(\Omega) = P(E \cup E^c) = P(E) + P(E^c),$$
and since $P(\Omega) = 1$ by axiom two, we have $1 = P(E) + P(E^c)$, hence $1 - P(E) = P(E^c)$.
\end{proof}

It should be intuitively clear that if the probability an event occurs is $\frac{1}{3}$, the probability it doesn't occur is $\frac{2}{3}$, but with now we've established this is not just intuition, it's a necessary consequence of the three Kolmogorov axioms. We could call such formally justified statements \emph{laws of probability}.

\begin{example}
What is the probability that a single card drawn from a standard shuffled deck is a king or not a face card?

Let $K =$ `the card drawn is a king' and $F =$ `the card drawn is a face card', and note the events $K$ and $F^c$ are mutually exclusive (a card cannot be both a king and not a face card). Using the countable additivity axiom and the law for complements we just demonstrated above,
$$P(K \cup F^c) = P(K)+P(F^c) = P(K) + 1 - P(F) = \textstyle\frac{4}{52} + 1 - \frac{12}{52} = \frac{44}{52}.$$
\end{example}

\subsection*{Inclusion-Exclusion}

What if we want to know $P(A \cup B)$ for events $A$ and $B$ which are not mutually exclusive? Fortunately, the inclusion-exclusion principle in Section \ref{InclusionExclusionSec} is also a valid law of probability. For any two events $A$ and $B$ in any probability model,
$$P(A \cup B) = P(A) + P(B) - P(A \cap B).$$

\begin{example}
What is the probability that a single card drawn from a standard shuffled deck is red or not a face card?

Let $R =$ `the card drawn is red' and $F =$ `the card drawn is a face card'. The events $R$ and $F^c$ are not mutually exclusive (the seven of hearts, for example, is red and not a face card). In fact, there are twenty cards that are red and not face cards, ace through ten of hearts and ace through ten of diamonds. Thus,
$$P(R \cup F^c) = P(R)+P(F^c)-P(R \cap F^c) = \textstyle\frac{26}{52} + 1 - \frac{12}{52} - \frac{20}{52} = \frac{46}{52}.$$

Alternatively, notice that $(R \cup F^c)^c = R^c \cap F$ by DeMorgan's law. Now $R^c \cap F =$ `the card drawn is a black face card', and there are six of these, so we have
$$P(R \cup F^c) = 1 - P((R \cup F^c)^c) = 1 - P(R^c \cap F) = 1 - \textstyle\frac{6}{52} = \frac{46}{52}.$$
\end{example}

\begin{example}
What is the probability that a single card drawn from a standard shuffled deck is either a diamond, a queen, or has an even number on it?

Applying the three-event version of the inclusion-exclusion principle to the events $D =$ `the card is a diamond', $Q =$ `the card is a queen', and $E =$ `the card has an even number on it',
$$\begin{aligned}P(D \cup Q \cup E) &= P(D)+ P(Q) + P(E) - P(D \cap Q) - P(Q\cap E) - P(E \cap D) + P(D \cap Q \cap E) \\
&= \textstyle\frac{13}{52} + \frac{4}{52} + \frac{20}{52} - \frac{1}{52} - \frac{0}{52} - \frac{5}{52} + \frac{0}{52} = \frac{31}{52}.\\
\end{aligned}$$
\end{example}

\section{Conditional Probability}

Suppose a single die is rolled. The result is unknown, but we're told it was an even number. What is the probability the result was six? 

Let $\Omega = \{1,2,3,4,5,6\}$, $S =$ `the outcome was a six', and $E =$ `the outcome was an even number'. Initially, each outcome in $\Omega$ is equally likely, and $P(S) = \frac{1}{6}$. However, given the knowledge that an even number was rolled, the sample space has been reduced from all the outcomes in $\Omega$ to only those in $E$. Now that $E$ has occurred, each of the three even outcomes has probability $\frac{1}{3}$, while each of the odd outcomes has probability zero. 

We write $P(S \given E) = \frac{1}{3}$ to indicate that the probability of $S$ under the assumption that $E$ has occurred is $\frac{1}{3}$, and we read $P(S \given E)$ as the \newterm{conditional probability}\index{Conditional Probability} of $S$ given $E$.


\begin{example} In a certain school, 35\% of students receive an A grade in reading, 22\% of students receive an A grade in mathematics, and 16\% of students receive an A grade in both. Of those students who received an A grade in reading, what proportion also received an A grade in mathematics?

Let the number of students at the school be denoted by $n$. Then the number that received an A in reading is $0.35n$, the number that received an A in mathematics is $0.22n$, and the number that received an A in both is $0.16n$. If we are considering only students who received an A in reading, we are dealing with a pool of $0.35n$ students. The only students in this pool who received an A in math must have received an A in both subjects, so there are total of $0.16n$ such students.

Thus, the proportion of A students in reading who also received an A in math is $\frac{0.16n}{0.35n} = \frac{0.16}{0.35} \simeq 0.46 = 46\%$. Note that if we select a student at random and define the events $R =$ `the student received an A in reading' and $M =$ `the student received an A in math', then we can rewrite this calculation as
$$P(M \given R) = \frac{P(M \cap R)}{P(R)} = \frac{0.16}{0.35} \simeq 0.46.$$
\end{example}

We'll use this idea to formally define conditional probability, but it's often more productive to view conditional probability as a reduction of the sample space. In the example above, if you gathered all students who received an $A$ in reading in a room together and asked those who received an $A$ in math to raise their hands, $46\%$ of them would raise their hands (if they're honest).

\begin{definition}\label{conditionalprob}
Given two events $E$ and $F$ in some sample space $\Omega$, the conditional probability of $E$ given $F$\index{Conditional Probability} is denoted $P(E \given F)$, and defined by
$$P(E \given F) = \frac{P(E \cap F)}{P(F)}.$$
\end{definition}
Note that division by zero could be problematic here, so we'll only define the conditional probability of $E$ given $F$ when $P(F) \neq 0$.

\subsection*{The Intersection Law}\index{Intersection Law}

This definition can be used either to compute $P(E \given F)$ when $P(E \cap F)$ and $P(F)$ are known, or to compute $P(E \cap F)$ when $P(E \given F)$ and $P(F)$ are known. In fact, we'll often use it in this second way, so it's worth renaming some variables and moving terms around to make it easier to use in that direction.
$$\begin{aligned}P(B \given A) &= \frac{P(B \cap A)}{P(A)} \\
P(B \given A) P(A) &= P(B \cap A) \\
P(B \given A) P(A) &= P(A \cap B) \\
P(A \cap B) &= P(A) P(B \given A)
\end{aligned}$$

Informally, if $A$ and $B$ both occur, then $A$ must occur and $B$ must occur in this new setting where the event $A$ has already happened.
\begin{example}
Suppose we draw two cards from a shuffled deck without replacement. What is the probability that both are hearts? 

Let $H_1 =$ `the first card drawn is a heart' and $H_2 =$ `the second card drawn is a heart'. Then both cards are hearts when $H_1 \cap H_2$ occurs.
$$P(H_1 \cap H_2) = P(H_1)P(H_2 \given H_1) = \frac{13}{52}\cdot\frac{12}{51} = \frac{156}{2652} \simeq 0.059$$
The key point is that if $H_1$ has occurred, a heart has been removed from the deck. There are then 51 cards remaining, 12 of which are hearts, so $P(H_2 \given H_1) = \frac{12}{51}$.
\end{example}

\begin{example}
Consider a point selected at random in a unit square, as in Example \ref{RandomPointInSquare}. As before, we take $\Omega = \{(x,y) \, | \, 0 \leq x \leq 1 \text{ and } 0 \leq y \leq 1\}$. If the point has $x < y$, what is the probability it lies in the top third of the square?

We're looking for $P(T \given G)$ where $T = \{(x,y) \, | \, 0 \leq x \leq 1 \text{ and } \frac{2}{3} \leq y \leq 1\}$ and $G = \{(x,y) \, | \, 0 \leq x < y \leq 1\}$. To compute this probability, we can draw a picture.

\begin{center}
\begin{minipage}{1.6in}
\begin{center}
\begin{tikzpicture}[scale=0.7]
\node (v1) at (-1.5,-1.5) {};
\node (v2) at (1.5,1.5) {};
\node (v3) at (-1.5,1.5) {};
\node (t) at (0,-0.5) {$G$};
\node (g) at (2,1) {$T$};
\draw (v1.center)--(v2.center)--(v3.center);
\fill [pattern=north west lines, pattern color=black] (v1.center)--(v2.center)--(v3.center);
\draw (-1.5,-1.5) rectangle (1.5,1.5);
\draw (-1.5,0.5) rectangle (1.5,1.5);
\fill [pattern=north east lines, pattern color=black] (-1.5,0.5) rectangle (1.5,1.5);
\end{tikzpicture}
\end{center}
\end{minipage}
\end{center}
\vspace{0pt}

Recall from Example \ref{RandomPointInSquare} that in this probability model, the probability of any event is the area it occupies. The small triangle in the top-right (the event $T \setminus G$) is half of a $\frac{1}{3}$ by $\frac{1}{3}$ square, hence it has area $\frac{1}{18}$, so
$$P(T \cap G) = \textstyle\frac{1}{3} - \frac{1}{18} = \frac{5}{18} \text{\ \  and \ } P(G) = \textstyle\frac{1}{2}.$$
Therefore, $P(T \given G) = \displaystyle\frac{P(T \cap G)}{P(G)} = \frac{\frac{5}{18}}{\frac{1}{2}} = \textstyle\frac{5}{9}$.
\end{example}

Note that the intersection law generalizes to sequences of events. Formally,
$$P(A_1 \cap A_2 \cap \dots \cap A_n) = P(A_1) P(A_2 \given A_1) P(A_3 \given A_1 \cap A_2)\, \cdots \, P(A_n \given A_1 \cap A_2 \cap \, \dots \, \cap A_{n-1}) $$
where the $i^{th}$ factor on the right is the conditional probability of $A_i$ under the assumption that all prior events in the sequence have occurred.

\begin{example}
What is the probability that when we draw four cards from a deck without replacement, all four cards are different suits?

Let $D_i =$ `the $i^{th}$ card is a different suit from all those cards the came before', then $P(D_1) = 1$, since no cards were drawn before the first, and we calculate
$$\begin{aligned}P(D_1 \cap D_2 \cap D_3 \cap D_4) &= P(D_1)P(D_2 \given D_1)P(D_3 \given D_1 \cap D_2)P(D_4 \given D_1 \cap D_2 \cap D_3) \\
&= 1 \cdot \frac{39}{51} \cdot \frac{26}{50} \cdot \frac{13}{49} \approx 0.1055.\end{aligned}$$
\end{example}

\begin{example}
Suppose that seven lightbulbs are in a box, but only two work. Bob will randomly select lightbulbs one at a time and test them. What is the probability he'll find both working lightbulbs after exactly three tests?

Let $W_i =$ `the $i^{th}$ lightbulb tested works'. Then we need $W_3$ to occur, and exactly one of $W_1$ and $W_2$, since if $W_1$ and $W_2$ both occur then Bob will be finished after only two tests.
$$\begin{aligned}&P((W_1 \cap {W_2}^c \cap W_3) \cup ({W_1}^c \cap W_2 \cap W_3)) \\
&= P(W_1 \cap {W_2}^c \cap W_3) + P({W_1}^c \cap W_2 \cap W_3) \\
&= P(W_1)P({W_2}^c \given W_1)P(W_3 \given W_1 \cap {W_2}^c) + P({W_1}^c)P(W_2 \given {W_1}^c)P(W_3 \given {W_1}^c \cap W_2) \\
&= \frac{2}{7}\cdot\frac{5}{6}\cdot\frac{1}{5} + \frac{5}{7}\cdot\frac{2}{6}\cdot\frac{1}{5} = \frac{10}{210} + \frac{10}{210} \simeq 0.095\end{aligned}$$

Note that $W_1 \cap {W_2}^c \cap W_3$ and ${W_1}^c \cap W_2 \cap W_3$ are mutually exclusive, so we can pass from the first to the second line without using the inclusion-exclusion principle.
\end{example}

%\subsection*{Conditional Probability is Well-Behaved}

%Interpreting conditional probability as a reduction of the sample space from $\Omega$ to some $B \subseteq \Omega$ with $P(B) > 0$, it's not hard to convince yourself that the basic probability laws presented in the last section hold for events conditioned on the occurrence of $B$, since the laws of probability hold in any sample space.

%\begin{example}
%Suppose that in a certain area, 36\% of all cats are black, and 76\% of all black cats are mean. What is the probability that a randomly selected cat is black and not mean?
%$$P(B \cap M^c) = P(B)P(M^c \given B) = P(B)(1-P(M \given B)) = 0.36 \cdot 0.24 = 0.0864$$
%\end{example}

%\begin{note}
%If you read up on conditional probability, you might the notation $P(A \given B,C)$. This means we are assuming \emx{all} events on the right side of the vertical line occur, so $P(A \given B,C) = P(A \given B \cap C)$.
%\end{note}

\section{The Law of Total Probability}

If two cards are drawn from a shuffled deck, what is the probability the second card drawn is a heart? It may be tempting to reply that the question is not well-posed, and in order to answer, we would require more information, namely, whether the first card drawn was a heart or not.

However, the question above is indeed well-posed. If we repeat the experiment a very large number of times, then in some proportion of trials, the second card will be heart. In order to compute this value analytically, we'll need the result below.

\begin{theorem}\label{lawtotalprob}\index{Law of Total Probability}
(\newterm{Law of Total Probability}) If $\{A_i\}_{i=1}^{n}$ is any mutually exclusive sequence of events with $\bigcup_{i=1}^{n}A_i = \Omega$, then for any event $B$,
$$P(B) = P(B \given A_1)P(A_1) + P(B \given A_2)P(A_2)+ \, \cdots \, + P(B \given A_n)P(A_n).$$
Note that we require $P(A_i) > 0$ for each $A_i$, so that the conditional probabilities on the right are defined.
\end{theorem}

When a sequence $\{A_i\}_{i=1}^{n}$ is mutually exclusive and $\bigcup_{i=1}^{n}A_i = \Omega$, we say that the sequence \emx{partitions the sample space}. This means that every possible outcome is in exactly one of the $A_i$. The law of total probability says that for any partition of $\Omega$, we can compute the probability of an event by taking a weighted average of conditional probabilities across the partition, where the weights are the probabilities that the corresponding piece of the partition occurs.

\begin{center}
\begin{tikzpicture}[scale=0.7]
\draw (-1.5,-1.5) rectangle (3.0,1.5);
\node (O) at (3.5,1.25) {$\Omega$};
\node (v1) at (-1.5,-1.5) {};
\node (v2) at (1.5,0.5) {};
\node (v3) at (-1.5,1.5) {};
\node (v4) at (3.0,-0.8) {};
\node (a1) at (-0.25,0.25) {$A_1$};
\node (a2) at (2.2,0.8) {$A_2$};
\node (a3) at (1.25,-0.8) {$A_3$};
\draw (v1.center)--(v2.center)--(v3.center);
\draw (v4.center)--(v2.center);
\end{tikzpicture}
\end{center}

We can obtain a special case of the law of total probability which occurs frequently in problems by considering the sequence $A$, $A^c$, for some event $A$. Note that regardless of what event $A$ happens to be, this sequence always partitions $\Omega$.

\begin{corollary} If $A$ is an event with $0 < P(A) < 1$, then for any event $B$, 
$$P(B) = P(B \given A)P(A) + P(B \given A^c)P(A^c).$$
\end{corollary}

\begin{example}
If two cards are drawn from a shuffled deck, what is the probability the second card drawn is a heart?

Let $H_1 =$ `the first card is a heart' and $H_2 =$ `the second card is a heart'. We'll apply the corollary above to compute the probability of $H_2$.
$$\begin{aligned}P(H_2) &= P(H_2 \given H_1)P(H_1) + P(H_2 \given {H_1}^c)P({H_1}^c) \\ &= \frac{12}{51} \cdot \frac{1}{4} + \frac{13}{51} \cdot \frac{3}{4} = \frac{51}{204} = \frac{1}{4}\end{aligned}$$

This result makes intuitive sense. Because we're computing $P(H_2)$ in the absence of any information about the first card, we may as well have put that first card on the bottom of the deck after it's drawn without ever looking at it, so we're still drawing a card from a complete shuffled deck.
\end{example}

\begin{proof} (of Theorem \ref{lawtotalprob}) If the sequence $\{A_i\}_{i=1}^n$ partitions $\Omega$, and $B \subseteq \Omega$,
$$\begin{aligned}\Omega &= A _1 \cup A_2 \cup \, \dots \, \cup A_n \\ B \cap \Omega &= B \cap (A_1 \cup A_2 \cup \, \dots \, \cup A_n) \\ B &= (B \cap A_1) \cup (B \cap A_2) \cup \, \dots \, \cup (B \cap A_n).\end{aligned}$$

Now the sequence of events $B \cap A_1$, $B \cap A_2$, $\, \dots \,$, $B \cap A_n$ is mutually exclusive because $\{A_i\}_{i=1}^{n}$ was mutually exclusive, and this new sequence is obtained by reducing the number of outcomes in each event. The events had empty intersections initially, so removing outcomes will keep the intersections empty.
$$\begin{aligned}P(B) &= P((B \cap A_1) \cup (B \cap A_2) \cup \, \dots \, \cup (B \cap A_n)) \\ &= P(B \cap A_1) + P(B \cap A_2) + \, \cdots \, + P(B \cap A_n) \\ &= P(A_1 \cap B) + P(A_2 \cap B) + \, \cdots \, + P(A_n \cap B) \\ &= P(A_1)P(B \given A_1) + P(A_2) P(B \given A_2) + \, \cdots \, + P(A_n) P(B \given A_n)\\ &= P(B \given A_1)P(A_1) + P(B \given A_2)P(A_2)  + \, \cdots \, + P(B \given A_n)P(A_n)\end{aligned}$$
\end{proof}

\subsection*{Tree Diagrams}\index{Tree Diagram}

Tree diagrams are a nice graphical tool for organizing calculations, and have the law of total probability built in to their structure. Each column represents a step in an experiment. If we draw three balls from an urn, for example, each column will represent a draw. If we roll a die and then flip a coin, the first column will represent the roll and the second will represent the flip.

\begin{example}
What is the probability that, when three cards are drawn from a shuffled deck without replacement, exactly one heart appears?

\noindent In the tree diagram below, the three columns represent three draws from a deck of cards. When a draw results in a heart, we denote that with an $H$, and when a draw does not result in a heart, we denote that with an $N$.
\begin{center}
\begin{tikzpicture}[grow=right]
\node[bag] {.}
    child {
        node[bag] {N}        
            child {
                node[bag]{N}
                    child {
                     node[bag] {$H \qquad \frac{1482}{10200}$}
                     edge from parent
                     node[above]  {$\frac{13}{50}$}
                    }
                edge from parent
                node[below]  {$\frac{38}{51}$}
            }
            child {
                node[bag]{H}
                    child {
                     node[bag] {$N \qquad \frac{1482}{10200}$}
                     edge from parent
                     node[above]  {$\frac{38}{50}$}
                 }
                edge from parent
                node[above]  {$\frac{13}{51}$}
            }
            edge from parent 
            node[below]  {$\frac{3}{4}$}
    }
    child {
        node[bag] {H}        
        child {
            node[bag] {N}
                child {
                     node[bag] {$N \qquad \frac{1482}{10200}$}
                     edge from parent
                     node[above]  {$\frac{38}{50}$}
                 }
                edge from parent
                node[above]  {$\frac{39}{51}$}
         }
        edge from parent         
            node[above]  {$\frac{1}{4}$}
    };
\end{tikzpicture}
\end{center}

Each edge is labelled with a conditional probability. For example, the top-center edge which runs from $H$ to $N$ is labelled with $P(N_2 \given H_1)$, the probability a non-heart is drawn on the second draw, after drawing a heart on the first. Note that we only depict the branches where the desired event (`exactly one heart appears') occurs. To find the probability of this event, we simply multiply along each branch, and add the results. If $E =$ `exactly one heart is drawn', then
$$\begin{aligned}P(E) = \frac{1482}{10200} + \frac{1482}{10200} + \frac{1482}{10200} = \frac{4446}{10200} \simeq 43.6\%.\end{aligned}$$
\end{example}

This works because the branches represent mutually exclusive events (at each fork, a given trial of the experiment can proceed along only one path), so we can add the probabilities of the branches without having to appeal to the inclusion-exclusion principle. Furthermore, the intersection law states that we can compute the probability of all events along a branch occurring by multiplying their conditional probabilities. Note that if we applied the law of total probability in the above example, partitioning the sample space with $\{H_1 \cap H_2, H_1 \cap N_2, N_1 \cap H_2, N_1 \cap N_2\}$, we would obtain
$$\begin{aligned}P(E) =& \ P(E \given H_1 \cap H_2)P(H_1 \cap H_2) + P(E \given H_1 \cap N_2)P(H_1 \cap N_2) \\ &+ P(E \given N_1 \cap H_2)P(N_1 \cap H_2) + P(E \given N_1 \cap N_2)P(N_1 \cap N_2) \\
=& \ 0 + P(E \given H_1 \cap N_2)P(N_2 \given H_1)P(H_1) + P(E \given N_1 \cap H_2)P(H_2 \given N_1)P(N_1) \\ &+ P(E \given N_1 \cap N_2)P(N_2 \given N_1)P(N_1) \\
=& \ P(N_3 \given H_1 \cap N_2)P(N_2 \given H_1)P(H_1) + P(N_3 \given N_1 \cap H_2)P(H_2 \given N_1)P(N_1) \\ &+ P(H_3 \given N_1 \cap N_2)P(N_2 \given N_1)P(N_1)\end{aligned}$$
and this is exactly how we calculated the result above. Each path from left to right corresponds to a term in the sum above. The event $E$ never occurs if two hearts are drawn, so $P(E \given H_1 \cap H_2) = 0$ and that term (which corresponds to a branch not depicted in the diagram) vanishes. Although working through the notation is good practice, I expect you'll agree that in cases like this, the tree diagram is more intuitive and simplifies the calculation.

If at any point in the tree diagram, the event we're interested in is sure to occur, there's no need to continue along that branch any further.

\begin{example}
Suppose that an urn contains three red and five green balls. If three balls are drawn from the urn without replacement, what is the probability that at least one red ball will be drawn?
\tikzstyle{level 1}=[level distance=2.5cm, sibling distance=1.5cm]
\begin{center}
\begin{tikzpicture}[grow=right]
\node[bag] {.}
    child {
        node[bag] {G}        
            child {
                node[bag]{G}
                    child {
                     node[bag]{$R \qquad \frac{60}{336}$}
                     edge from parent
                     node[below]  {$\frac{3}{6}$}
                    }
                edge from parent
                node[below]  {$\frac{4}{7}$}
            }
            child {
                node[bag]{$R \qquad \frac{15}{56}$}
                edge from parent
                node[above]  {$\frac{3}{7}$}
            }
            edge from parent 
            node[below]  {$\frac{5}{8}$}
    }
    child {
        node[bag]{$R \qquad \frac{3}{8}$}        
        edge from parent         
            node[above]  {$\frac{3}{8}$}
    };
\end{tikzpicture}
\tikzstyle{level 1}=[level distance=2.5cm, sibling distance=2.25cm]
\end{center}

\noindent As soon as a red ball is drawn, there's no need to continue the branch. Regardless of what happens from that point on, a red ball will have been drawn. If A = `at least one red ball is drawn', then 
$$\begin{aligned}P(A) = \frac{3}{8} + \frac{15}{56} + \frac{60}{336} = \frac{276}{336} \simeq 82.1\%.\end{aligned}$$
\end{example}

\begin{example}
Suppose we draw cards with replacement from a shuffled deck until the first ace appears. What is the probability that all cards drawn are black?

\tikzstyle{level 1}=[level distance=2.5cm, sibling distance=1.5cm]
\begin{center}
\begin{tikzpicture}[grow=right]
\node[bag] {.}
    child {
        node[bag] {BNA}        
            child {
                node[bag]{BNA}
                    child {
                     node[bag]{$\dots$}
                     edge from parent
                     node[below]  {$\frac{24}{52}$}
                    }
										child {
                     node[bag]{$BA$}
                     edge from parent
                     node[above]  {$\frac{2}{52}$}
                    }
                edge from parent
                node[below]  {$\frac{24}{52}$}
            }
            child {
                node[bag]{$BA$}
                edge from parent
                node[above]  {$\frac{2}{52}$}
            }
            edge from parent 
            node[below]  {$\frac{24}{52}$}
    }
    child {
        node[bag]{$BA$}        
        edge from parent         
            node[above]  {$\frac{2}{52}$}
    };
\end{tikzpicture}
\tikzstyle{level 1}=[level distance=2.5cm, sibling distance=2.25cm]
\end{center}

Here $BA$ denotes `black ace' and $BNA$ denotes `black non-ace'. There's no limit to the number of draws that could occur before the first black ace appears, so the tree diagram is infinite. However, we can still sum the products along each branch. Let $OB = $ `only black cards are drawn', then
$$\begin{aligned}P(OB) &= \frac{2}{52} + \frac{24}{52}\cdot\frac{2}{52} + \frac{24}{52}\cdot\frac{24}{52}\cdot\frac{2}{52} +  \frac{24}{52}\cdot\frac{24}{52}\cdot\frac{24}{52}\cdot\frac{2}{52} + \ \dots \\
&= \frac{2}{52}\left(1 + \frac{24}{52} + \left(\frac{24}{52}\right)^2 + \left(\frac{24}{52}\right)^3 + \ \dots\right) \\
&= \frac{2}{52}\left(\sum_{i=0}^{\infty} \left(\frac{24}{52}\right)^n\right) \\
&= \frac{2}{52} \cdot \frac{1}{1-(\frac{24}{52})} = \frac{1}{14} \simeq 7.1\%\end{aligned}$$
\end{example}

Note that the sum of the infinite series above was evaluated with the geometric series sum formula, $a+ar+ar^2+ \dots = \frac{a}{1-r}$ when $-1 < r < 1$.

\section{Independence}\label{IndependentEventsSec}

Suppose we flip a coin and then roll a die. Let $H =$ `a head is flipped', $E =$ `an even number is rolled' and $L =$ `a number larger than three was rolled'. Note that
$$P(E) = \frac{1}{2} \text{ \ and \ } P(E \given H) = \frac{1}{2}$$
since knowing that a head was flipped doesn't change the probability an even number is rolled. On the other hand, we can calculate 
$$P(E) = \frac{1}{2} \text{ \ and \ } P(E \given L) = \frac{2}{3}.$$
In this case, knowledge that the number rolled was a four, five, or six increases the probability that the result of the roll was even. The events $E$ and $H$ are said to be \emx{independent}, while the events $E$ and $L$ are \emx{not independent}.

\begin{definition}\label{independentevents}
Events $A$ and $B$ are called \newterm{independent}\index{Events!Independent}\index{Independence!of a pair of Events} if $P(B) = P(B \given A)$.
\end{definition}

\begin{example}Suppose we flip a coin two times. Note that $\Omega = \{HH, HT, TH, TT\}$ is a sample space with equally likely outcomes for the experiment.

If we consider the events $H_1 =$ `the first flip is a head' and $H_2 =$ `the second flip is a head', then $P(H_2) = \frac{1}{2}$ and $P(H_2 \given H_1) = \frac{1}{2}$, so these events are independent.

If we instead let $H_1 =$ `the first flip is a head' and $T_1 =$ `the first flip is a tail', then $P(T_1) = \frac{1}{2}$ and $P(T_1 \given H_1) = 0$, so these two events are not independent.
\end{example}

\begin{example}
Consider two cards drawn from a shuffled deck. Let $S_1 =$ `a spade is drawn on the first draw' and $S_2 =$ `a spade is drawn on the second draw'. Then $S_1$ and $S_2$ are independent provided the draws are done with replacement, but not independent if the draws are done without replacement. In this second case, $P(S_2 \given S_1) = \frac{12}{51}$, but using the law of total probability, $P(S_2) = \frac{1}{4}$.
\end{example}

Using the definition of the conditional probability $P(A \given B)$, we can state what it means for two events to be independent in a different way. 
$$P(A) = P(A \given B) \biimp P(A) = \frac{P(A \cap B)}{P(B)} \biimp P(A)P(B) = P(A \cap B)$$
Thus, when $A$ and $B$ are independent events, $P(A \cap B) = P(A)P(B)$. This relation is often used as the definition of independence. As we just saw, it's equivalent to the definition we gave earlier but also applies when $P(A) = 0$ or $P(B) = 0$ (in these cases $P(A\given B)$ or $P(B \given A)$ would be undefined).

\begin{keypoint}
Independence is a symmetric relation, which you can interpret intuitively as `the probability of one event occurring is not influenced by the occurrence of the other' without worrying about which event comes first. The alternate form of the definition $P(A \cap B)=P(A)P(B)$ confirms this, as swapping the roles of $A$ and $B$ yields the same equation.
\end{keypoint}

\begin{theorem}\label{compindependent}
If $A$ and $B$ are independent, then $A$ and $B^c$ are independent.
\end{theorem}
\begin{proof}
Note that $A = (A \cap B) \cup (A \cap B^c)$ and hence $P(A) = P(A \cap B) + P(A \cap B^c)$ since the two events on the right are mutually exclusive. Isolating $P(A \cap B^c)$,
$$\begin{aligned}P(A \cap B^c) &= P(A) - P(A \cap B) \\
&= P(A) - P(A)P(B) \\
&=P(A)(1-P(B)) \\
&= P(A)P(B^c).\end{aligned}$$
\end{proof}

\begin{corollary}
If $A$ and $B$ are independent, then $A^c$ and $B^c$ are independent.
\end{corollary}
\begin{proof}
If $A$ and $B$ are independent, then $A$ and $B^c$ are independent by Theorem \ref{compindependent}, but independence is symmetric, so applying Theorem \ref{compindependent} again, this time to $B^c$ and $A$, we conclude $B^c$ and $A^c$ are independent.
\end{proof}

\begin{remark}
The upshot of the previous theorem and its corollary is that the independence relation is unaffected by complements. This can be useful, since calculating $P(A^c \given B)$, or $P(A \given B^c)$, or $P(A^c \given B^c)$ may be easier than calculating $P(A \given B)$.
\end{remark}

Given a sequence of events $\{A_i\}_{i=1}^{n}$, we say this sequence is independent if the occurrence of any collection of these events does not influence the chance that any other event in the sequence occurs. Formally, it's easiest to state this by extending the relation $P(A \cap B) = P(A)P(B)$ as below.

\begin{definition}\index{Independence!of a sequence of Events}
The sequence of events $\{A_i\}_{i=1}^{n}$ is independent if for any subsequence $A_{i_1}, A_{i_2}, \, \dots \, , A_{i_k}$ of events taken from $\{A_i\}_{i=1}^{n}$,
$$\begin{aligned}P(A_{i_1} \cap A_{i_2} \cap \, \dots \, \cap A_{i_k}) = P(A_{i_1})P(A_{i_2}) \, \cdots \, P(A_{i_k})\end{aligned}$$
\end{definition}

\begin{example}
Suppose we toss a fair coin twice. Let $H_1 =$ `The first toss is heads', $H_2 =$ `The second toss is heads', and $E =$ `Exactly one toss results in heads'. As we already observed, $H_1$ and $H_2$ are independent. Moreover, $P(E) = \frac{1}{2}$ since exactly one head is flipped in two of the four equally likely outcomes, and $P(E \given H_1) = \frac{1}{2}$ since assuming $H_1$ occurs, $E$ occurs when the second flip is a tail. Thus, $H_1$ and $E$ are independent, and essentially the same argument shows $H_2$ and $E$ are independent.

Since each pair of events in the sequence $\{H_1, H_2, E\}$ is independent, we say this sequence is pairwise independent\index{Pairwise Independence}. However, note that if we assume both $H_1$ and $H_2$ occur, the probability $E$ occurs becomes zero, that is, $P(H_1 \cap H_2 \cap E) = 0$, but we can calculate $P(H_1)P(H_2)P(E) = \frac{1}{2}\cdot\frac{1}{2}\cdot\frac{1}{2} = \frac{1}{8}$.
\end{example}

The example above illustrates that even if every pair of events in a sequence is independent, there could still be some event in the sequence whose probability is influenced by the joint occurrence of some larger collection of other events. In other words, for sequences of events, pairwise independence does not imply independence.

\begin{example}
Suppose that a 64 bit sequence (a sequence of 64 zeros and ones) is sent over an unreliable channel. The probability that each bit gets flipped is 0.005 independent of what happens to the other bits. What is the probability that the entire sequence is received without errors?

Let $F_i = $ `the $i^{th}$ bit gets flipped', then $P(F_i) = 0.005$, and the probability that the $i^{th}$ bit is received correctly is $P({F_i}^c) = 1-0.005 = 0.995$. If $E =$ `the entire string is received without any flipped bits' then
$$\begin{aligned}P(E) &= P({F_1}^c \cap {F_2}^c \cap \, \dots \, \cap {F_{64}}^c) \\
&= P({F_1}^c)P({F_2}^c) \, \cdots \, P({F_{64}}^c) \\
&= 0.995 \cdot 0.995 \cdot \, \cdots \, \cdot 0.995 \\
&\simeq 0.72556 \simeq 72.6\%\end{aligned}$$
\end{example}






% Modified 31 Oct 2005:  Conditioning fallacy alluded to.
% This chapter has been modified on 6-4-05.
% There are two \choice
\pagestyle{headings}
\chapter{Random Variables} \label{chp 4}

\section{Discrete Random Variables}\label{RVIntroSec}

\begin{definition}\index{Random Variable}\label{RandomVariableDef} A \newterm{random variable} is a function from a sample space $\Omega$ to $\mathbb{R}$. If the range of this function is countable, the random variable is called discrete. In this section, we'll deal only with discrete random variables.
\end{definition}

Since a sample space is used to represent the possible outcomes of an experiment, a random variable associates a real number with each possible outcome. In other words, a random variable is a variable whose value is determined by the outcome of an experiment involving some element of chance.

Random variables are typically denoted using capital letters at the end of the alphabet, such as $X$, $Y$, or $Z$, and the probability the random variable $X$ takes the real number value $x$ is denoted $P(X = x)$. 
%Notice that $x$ is the usual kind of real number variable you're used to dealing with in algebra and calculus, while $X$ is a random variable with an associated sample space $\Omega$ and a potentially very limited range of possible values.

\begin{example} Let $X$ denote the outcome of single roll of a fair die. Then $X$ is a random variable defined on the sample space $\Omega = \{1,2,3,4,5,6\}$.

Not surprisingly, $P(X = 1) = \frac{1}{6}$, $P(X = 2) = \frac{1}{6}$, and so on. In general, $P(X = x) = \frac{1}{6}$ for all $x \in \{1,2,3,4,5,6\}$ since the die is fair. We could also ask for $P(X > 4)$, and again you should not be surprised that $P(X > 4) = P(X=5) + P(X=6) = \frac{2}{6}$.
\end{example}

\begin{remark} It may take some time to get used to this new notation. If $X$ is a random variable, an equation like $X = 2$ or an inequality like $X > 2$ is an event, that may or may not occur depending on the value $X$ takes, so the expression $P(X = 2)$ denotes the probability of an event. \end{remark}

\begin{example} Let $Y$ denote the number of heads in three flips of a fair coin. Then $Y$ is a random variable defined on the sample space $\Omega = \{HHH, HHT, HTH, HTT, THH, THT, TTH, TTT\}$.

The range of $Y$ is $\{0,1,2,3\}$. We can calculate the probability $Y$ takes any of these values with the techniques of the previous chapter, and summarize the results in. a table.
\begin{center}
\begin{minipage}{0.4\textwidth}
$$\begin{aligned}P(Y = 0) &= P(\{TTT\}) = \textstyle\frac{1}{8} \\
P(Y = 1) &= P(\{HTT, THT, TTH\}) = \textstyle\frac{3}{8} \\
P(Y = 2) &= P(\{HHT, HTH, THH\}) = \textstyle\frac{3}{8} \\
P(Y = 3) &= P(\{HHH\}) = \textstyle\frac{1}{8}\end{aligned}$$
\end{minipage}\begin{minipage}{0.4\textwidth}
\renewcommand*{\arraystretch}{1.35}
\begin{center}
\begin{tabular}{c|c}
$y$ & $P(Y = y)$ \\
\hline
$0$ & $\frac{1}{8}$ \\
$1$ & $\frac{3}{8}$ \\
$2$ & $\frac{3}{8}$ \\
$3$ & $\frac{1}{8}$ \\
\end{tabular}
\end{center}
\renewcommand*{\arraystretch}{1}
\end{minipage}
\end{center}
\end{example}
\begin{keypoint}
When we actually perform the experiment and flip the coin three times, the random variable will take a real number value, known as a \newterm{realization}\index{Realization}. In the table above, the leftmost column is labelled with every possible realization of $Y$. Upper case letters will denote random variables, and the corresponding lower case letters will be used for realizations.
\end{keypoint}

%\begin{example} Let $X$ denote the number of hearts in three draws, with replacement, from a shuffled deck. What is $P(X > 1)$?

%We need the probability more than one heart appears in three draws, performed with replacement. Using the law of total probability / a tree diagram, we can calculate
%\eqnsgap{P(X > 1) = \frac{1}{64}+ \frac{3}{64}+ \frac{3}{64}+ \frac{3}{64} = \frac{10}{64}\approx 0.156.}
%\end{example}

%\begin{remark} Note that random variables must take real number values. If we let $X$ denote the number of heads in a single flip of a coin, then $X$ is a random variable, but if we let $X$ denote the outcome (Heads or Tails) of a single flip of a coin, then $X$ is not a random variable.
%\end{remark}

%\index{Distribution}\index{Random Variable! distribution of}The association between events that are defined in terms of a random variable $X$ and their probabilities is known as the probability distribution of $X$, which is often shortened to simply `the distribution of $X$'. There are many ways to present the distribution of a discrete random variable. One way is to simply list the possible values of $X$ and their probabilities in a table.


\subsection*{Probability Mass Functions}

%The table in the example above describes the \newterm{distribution}\index{Distribution} of the random variable $Y$, since it associates every possible value of $Y$ with a probability.

Given a random variable $X$, the function $f_X: \mathbb{R} \to \mathbb{R}$ that assigns to each possible realization of $X$ the probability it occurs (the function which is described by the table in our example above) is called the probability mass function of $X$. If $x$ is any value outside the range of $X$, then we omit this value from the table.

\begin{definition}\index{Probability Mass Function}
The \newterm{probability mass function} (abbreviated pmf) $f_X$ for the random variable $X$ is defined by $f_X(x) = P(X = x)$. We often represent this function with a table, and visualize it with a figure known as a lollipop plot.
\end{definition}

\begin{example}\label{diepairex}
Let $X$ denote the sum of the two numbers showing when a pair of fair six-sided dice is rolled. Graph the probability mass function $f_X$ with a lollipop plot.

Each die is equally likely to show any value in the set $A = \{1,2,3,4,5,6\}$, so for a pair of dice, all outcomes in $\Omega = A \times A$ occur with probability $\frac{1}{36}$. 
{\small
\begin{center}
\begin{tabular}{c|cccccc}
$_{D_1} \setminus ^{D_2}$ & 1 & 2 & 3 & 4 & 5 & 6 \\
\hline
1 & 2 & 3 & 4 & 5 & 6 & 7 \\
2 & 3 & 4 & 5 & 6 & 7 & 8 \\
3 & 4 & 5 & 6 & 7 & 8 & 9 \\
4 & 5 & 6 & 7 & 8 & 9 & $\!$10 \\
5 & 6 & 7 & 8 & 9 & $\!$10 & $\!$11 \\
6 & 7 & 8 & 9 & $\!$10 & $\!$11 & $\!$12 \\
\end{tabular}
\end{center}
}
Thus, $P(Y = 2) = \frac{1}{36}$, $P(Y = 3) = \frac{2}{36}$, $P(Y = 4) = \frac{3}{36}$, and so on. Drawing one lollipop at each realization of $X$ whose height is its probability, we obtain the lollipop plot shown below.
\begin{center}
\begin{tikzpicture}[scale = 0.6]
\begin{axis} [ymin = 0, ymax = 0.2,
	tick label style={font=\scriptsize, scale = 1/0.6},
    xtick={2,3,...,12},
    ytick = {0.1,0.2},
    width=12cm, height=8cm
    ]
\addplot+[ycomb] plot coordinates {(2,1/36) (3,2/36) (4,3/36) (5,4/36) (6,5/36) (7,6/36) (8,5/36) (9,4/36)(10,3/36) (11,2/36) (12,1/36)}; 
\end{axis} 
\end{tikzpicture}
\end{center}
\end{example} 

\begin{proposition}\label{pmfproperties} Suppose that $X$ is a discrete random variable with range $A = \{x_1, x_2, x_3, \, \dots\}$. Then the probability mass function $f_X$ satisfies the following properties.
\begin{enumerate}
\item $f_X(x) \geq 0$ for all $x \in \mathbb{R}$.
\item $f_X(x) = 0$ for all $x \not\in A$.
\item $\sum_{x \in A} f_X(x) = 1$.
\end{enumerate}
\end{proposition}

\begin{remark}These properties completely characterize the class of probability mass functions for discrete random variables. In other words, if $f$ is a function with these three properties, then it is the pmf of a discrete random variable.\end{remark}

\begin{example}
Find the value of $k$ so that the function $h$ below is a pmf.
\renewcommand*{\arraystretch}{1.35}
\eqns{h(x) = \left\{
\begin{array}{cl}
      \left(\frac{1}{3k}\right)^x & \text{ if \ } x = 1, 2, 3, \dots \\
      0 & \text{otherwise} \\ \end{array} 
\right.}
\renewcommand*{\arraystretch}{1}

As long as $k$ is positive, $h$ will satisfy properties $(i)$ and $(ii)$ above. The last property allows us to determine $k$ uniquely.
$$\sum_{x \in A} h(x) =  \sum_{x = 1}^{\infty} \left(\frac{1}{3k}\right)^x = \frac{1}{3k} +  \left(\frac{1}{3k}\right)^2 +  \left(\frac{1}{3k}\right)^3 + \cdots = \frac{\frac{1}{3k}}{1-\frac{1}{3k}} = \frac{1}{3k - 1}$$

The geometric series sum formulas was used to evaluate the infinite sum. Setting the last expression equal to 1 gives $3k - 1 = 1$, and hence $k = \frac{2}{3}$. Using this value of $k$ gives the function $h$ below, which we can graph using a lollipop plot. Only the first few values can be shown, since the graph continues off to the right indefinitely.
\renewcommand*{\arraystretch}{1.35}
$$\begin{aligned}h(x) = \left\{
\begin{array}{cl}
      \left(\frac{1}{2}\right)^x & \text{ if \ } x = 1, 2, 3, \dots \\
      0 & \text{otherwise} \\ \end{array} 
\right.\end{aligned}$$

\renewcommand*{\arraystretch}{1}
\begin{center}
\begin{tikzpicture}[scale = 0.6]
\begin{axis}[ymin = 0, ymax = 0.3,
	tick label style={font=\scriptsize, scale = 1/0.6},
    xtick={2,3,...,8},
    ytick = {0.1,0.2},
    width=12cm, height=8cm]
\addplot+[ycomb] plot coordinates {(1,1/2) (2,1/4) (3,1/8) (4,1/16) (5,1/32) (6,1/64) (7,1/128) (8,1/256)}; 
\end{axis} 
\end{tikzpicture}
\end{center}
\end{example}

%\subsection*{Functions of a Discrete Random Variable}\label{FunctionsOfAnRVSec}
%\par
%If $X$ is a discrete random variable, and $g: \mathbb{R} \to \mathbb{R}$ is any function, then $Y = g(X)$ is also a discrete random variable. Its value is completely determined by the value of $X$, and given the pmf of $X$, we can determine the pmf of $Y$ by simply running all values in the range of $X$ through the function $g$. 

%\begin{example}\label{FunctionRVEx}
%Let $X$ be the random variable whose pmf is graphed below, and let $Y = 1 + X^2$ (so the function $g$ in this example is $g(x) =1+x^2$). Find the pmf of $Y$.
%\begin{center}
%\begin{tikzpicture}[scale = 0.6]
%\begin{axis}[ymin = 0, ymax = 0.4,
%xmin = -2, xmax = 4, tick label style={font=\scriptsize, scale = 1/0.6}, xtick={-2,-1,0,...,4}, ytick = {0.1,0.2,0.3,0.4}, width=12cm, height=8cm]
%\addplot+[ycomb] plot coordinates {(-1,2/10) (0,1/10) (1,2/10) (2,2/10) (3,3/10)}; 
%\end{axis} 
%\end{tikzpicture}
%\end{center}

%We calculate the realization of $Y$ that corresponds to each realization of $X$ by running them through the function $g(x) = 1+x^2$.
%\renewcommand*{\arraystretch}{1.35}
%\begin{center}
%\begin{tabular}{cc|c}
%$x$ & $y$ & $f_X(x)$ \\
%\hline
%$-1$ & $2$ & $0.2$ \\
%$0$ & $1$ & $0.1$ \\
%$1$ & $2$ & $0.2$ \\
%$2$ & $5$ & $0.2$ \\
%$3$ & $10$ & $0.3$
%\end{tabular}
%\end{center}
%\renewcommand*{\arraystretch}{1}
%Now we have a list of every possible realization of $Y$ and the probability of each. Note that there are two realizations of $X$ which result in $Y = 2$, so $f_{Y}(2) = f_X(-1) + f_X(1)$.
%\renewcommand*{\arraystretch}{1.35}
%\begin{center}
%\begin{minipage}{0.4\textwidth}
%\centering
%\begin{tabular}{c|c}
%$y$ & $f_Y(y)$ \\
%\hline
%$1$ & $0.1$ \\
%$2$ & $0.4$ \\
%$5$ & $0.2$ \\
%$10$ & $0.3$
%\end{tabular}
%\end{minipage}\begin{minipage}{0.6\textwidth}
%\centering
%\begin{tikzpicture}[scale = 0.6]
%\begin{axis}[ymin = 0, ymax = 0.5,
%xmin = 0, xmax = 11, tick label style={font=\scriptsize, scale = 1/0.6},xtick={0,1,...,11},ytick = {0.1,0.2,0.3,0.4,0.5},width=12cm, height=8cm]
%\addplot+[ycomb] plot coordinates {(1,1/10) (2,4/10) (5,2/10) (10,3/10)}; 
%\end{axis}
%\end{tikzpicture}
%\end{minipage}
%\end{center}
%\renewcommand*{\arraystretch}{1}
%\end{example}

\subsection*{Cumulative Distribution Functions}\label{DiscreteExpectationSec}

\begin{definition}\index{Cumulative Distribution Function}\index{Distribution Function}
The \newterm{cumulative distribution function} (abbreviated cdf) $F_X$ for the random variable $X$ is defined by $F_X(x) = P(X \leq x)$.
\end{definition}

Notice that we use capital letters for cdfs and lower case letters for pmfs. This is conventional and we'll always do it (for a good reason, which we'll see later). It's also worth mentioning that the word `cumulative' is frequently dropped, and the cdf is simply called the distribution function.

\begin{example} Suppose we roll a pair of dice, and let $X$ be the minimum of the two values that appear on the dice. Find the pmf $f_X$ and the cdf $F_X$.

The random variable $X$ takes integer values between 1 and 6 inclusive, and referring back to the table in Example \ref{diepairex}, we can find the probability of each realization. We have $P(X = 1) = \frac{11}{36}$ (outcomes in the first row or column), $P(X = 2) = \frac{9}{36}$ (outcomes in the second row to the right of the 3, or in the second column below the 3), $P(X = 3) = \frac{7}{36}$ (outcomes in the third row to the right of the 5, or in the third column below the 5), and so on.

The graph of the pmf $f_X$ is shown on the left, and the cdf $F_X$ is shown on the right.
\begin{center}
\begin{tikzpicture}[scale = 0.6]
\begin{axis} [ymin = 0.01, ymax = 0.35, xmin=0, xmax = 7,
	tick label style={font=\scriptsize, scale = 1/0.6},
    xtick={0,1,...,7},
    ytick = {0.1,0.2,0.3},
    width=10cm, height=6cm]
\addplot+[ycomb] plot coordinates {(1,11/36) (2,9/36) (3,7/36) (4,5/36) (5,3/36) (6,1/36)}; 
\end{axis} 
\end{tikzpicture}\ \ \ \ \ \  \begin{tikzpicture}[scale = 0.6]
\begin{axis}[
	tick label style={font=\scriptsize, scale = 1/0.6},
    xtick={0,1,...,7},
    ytick = {0.5,1},
    width=10cm, height=6cm,
    ymin=0,ymax=1,
    xmin=0, xmax=7,
    clip=false,
    jump mark left,
    every axis plot/.style={very thick},
    cdf,
    table/create on use/cumulative distribution/.style={
        create col/expr={\pgfmathaccuma + \thisrow{f(x)}}   
    }
]
\addplot [cdf init, blue] table [y=cumulative distribution]{
x f(x)
0 0
1 11/36
2 9/36
3 7/36
4 5/36
5 3/36
6 1/36
7 0
};
\end{axis}
\end{tikzpicture}
\end{center}

Notice that $F_X(2.5) = P(X \leq 2.5) = P(X = 1) + P(X = 2) = \frac{11}{36} + \frac{9}{36} \approx 0.56$. The cumulative distribution function adds up the probabilities that $X$ takes any value smaller than or equal to its input. 
\end{example}

Consider two real numbers $a$ and $b$ with $a \leq b$. Given any random variable $X$,  $F_X(a) \leq F_X(b)$, because if $X$ is smaller than $a$ then it must also be smaller than $b$, in other words, the event $X \leq a$ is a subset of the event $X \leq b$, and hence it cannot be more likely. This means that a cdf is always non-decreasing. This and other key properties that all cdfs share are listed in the proposition below.

\begin{proposition}\label{cdfproperties}
For any random variable $X$, the distribution function $F_X$ satisfies the following properties.
\vspace{-0.5em}
\begin{enumerate}
\item $F_X(x) \geq 0$ for all $x \in \mathbb{R}$.
\item $F_X$ is non-decreasing.
\item $\displaystyle\lim_{x \to -\infty} F_X(x) = 0$ and $\displaystyle\lim_{x \to \infty} F_X(x) = 1$.
\item $F_X(x)$ is right-continuous, meaning $\displaystyle\lim_{x \to a^{+}} F_X(x) = F_X(a)$.
\end{enumerate}
\end{proposition}

\begin{remark}These properties completely characterize the class of distribution functions, that is, any function satisfying all four properties is the cdf of some random variable.
\end{remark}

When dealing with cdfs, $F_X(a+)$ is used as an abbreviation of $\lim_{x \to a^{+}}F_X(x)$, and similarly, $F_X(a-)$ is used to abbreviate $\lim_{x \to a^{-}}F_X(x)$. Using this notation we can express the probabilities of many events succinctly in terms of the cdf.

\begin{center}
\begin{tabular}{c|c}
Event & Probability \\
\hline
$X \leq a$ & $F_X(a)$ \\
$X < a$ & $F_X(a-)$ \\
$X = a$ & $F_X(a) - F_X(a-)$ \\
$X \geq a$ & $1 - F_X(a-)$ \\
$X > a$ & $1 - F_X(a)$ \\
$a < X < b$ & $F_X(b-) - F_X(a)$ \\
$a \leq X \leq b$  & $F_X(b) - F_X(a-)$ \\
\end{tabular}
\end{center}

Of particular interest is that fact that $P(X=a) = F_X(a) - F_X(a-)$. This is a formal statement of the observation that the possible values of a discrete random variable $X$ are the values where $F_X$ has a discontinuity, and the distance the graph jumps at $a$ is $P(X = a)$.

\begin{example}
Consider the function whose graph is given below. Check that this function is a cdf for some random variable. If we call the random variable with this cdf $Z$, find $P(1 < Z < 2)$, $P(Z = 4)$, and $P(Z > 1)$.

\begin{center}
\begin{tikzpicture}[scale = 0.6]
\begin{axis}[
	tick label style={font=\scriptsize, scale = 1/0.6},
    xtick={-2,-1,...,6},
    ytick = {1/6,2/6,3/6,4/6,5/6,1},
    yticklabels = {$\frac{1}{6}$,$\frac{2}{6}$,$\frac{3}{6}$,$\frac{4}{6}$,$\frac{5}{6}$,$1$},
    width=12cm, height=8cm,
    ymin=0,ymax=1,
    xmin=0, xmax=7,
    clip=false,
    jump mark left,
    ymin=0,ymax=1,
    xmin=-3, xmax=6,
    every axis plot/.style={very thick},
    cdf,
    table/create on use/cumulative distribution/.style={
        create col/expr={\pgfmathaccuma + \thisrow{f(x)}}   
    }
]
\addplot [cdf init,blue] table [y=cumulative distribution]{
x f(x)
-3 0
-2 1/6
0 1/6
2 1/3
4 1/6
5 1/6
6 0
};
\end{axis}
\end{tikzpicture}
\end{center}

It's easy to see that this function is non-negative, non-decreasing, and tends to $0$ as x $\to -\infty$ and 1 as $x \to \infty$ (we're assuming the graph continues off to the left and right along the horizontal lines shown). Furthermore, it's right-continuous since at each jump the value of the function is equal to the limit from the right.
$$\begin{aligned}P(1 < Z < 2) &= F_Z(2-) - F_Z(1) & \qquad P(Z = 4) &= F_Z(4) - F_Z(4-) \\
&=\frac{2}{6} - \frac{2}{6} = 0 &  &=\frac{5}{6} - \frac{4}{6} = \frac{1}{6}\end{aligned}$$
\vspace{1.5em}
$$\begin{aligned}P(Z > -1) &= 1 - F_Z(-1) \\ & = 1 - \frac{1}{6} = \frac{5}{6}\end{aligned}$$
\end{example}

\subsection*{Independent Random Variables}

We've seen in Section \ref{IndependentEventsSec} that events $A$ and $B$ in $\Omega$ are independent if $P(A \given B) = P(A)$, or equivalently, if $P(A \cap B) = P(A)P(B)$. This latter form of the definition extends to discrete random variables in a natural way.

\begin{definition}\label{IndependenceOfRVs}\index{Independence!of Discrete Random Variables}
Two random variables $X$ and $Y$ defined on the same sample space $\Omega$ are \newterm{independent} if $P(X = x \,\cap\, Y = y) = P(X = x)P(Y = y)$ for all realizations $x$ and $y$.\end{definition}

\begin{remark}
The condition $P(X = x \,\cap\, Y = y) = P(X = x)P(Y = y)$ for all realizations $x$ and $y$ implies that $P(X \in A \,\cap\, Y \in B) = P(X \in A)P(Y \in B)$ for any events $X \in A$ and $Y \in B$. The definition is stated in terms of realizations since it's easier to check that way.
\end{remark}

\begin{example}Suppose a fair die is rolled twice. Let $X$ be the result of the first roll and $Y$ be the result of the second. Then $X$ and $Y$ are independent random variables, since for any $x$ and $y$ in $\{1,2,3,4,5,6\}$, we have $P(X = x \,\cap\, Y = y) = \frac{1}{36}$, while $P(X = x) = \frac{1}{6}$ and $P(Y=y) = \frac{1}{6}$.
\end{example}

\begin{example}Let X be a number chosen at random from the set $\{1,2,3,4,5,6\}$, and let $Y$ be the remainder when $X$ is divided by 3. Then $X$ and $Y$ are not independent random variables, since, for instance, $P(X = 3 \,\cap\, Y = 1) = 0$, because if $X = 3$ then we must have $Y = 0$, but we can calculate $P(X = 3)P(Y = 1) = \frac{1}{6}\cdot\frac{2}{6} \neq 0$.
\end{example}

As it did in Section \ref{IndependentEventsSec}, the notion of independence extends to sequences. If the distribution of each random variable in the sequence is unaffected by knowing the realizations any number of other random variables in the sequence take, then the sequence is independent.

\begin{example}
Four fair six-sided dice are rolled. What is the probability at least one shows a six?

Let $X_i$ be the result that shows on the $i^{th}$ die. Then we have
$$\begin{aligned}
P(X_1 = 6 \cup \dots \cup X_4 = 6) &= P((X_1 \neq 6 \cap \dots \cap X_4 \neq 6)^c) \\
&= 1 - P(X_1 \neq 6 \cap \dots \cap X_4 \neq 6) \\
&= 1 - P(X_1 \neq 6) \cdot \dots \cdot P(X_4 \neq 6) \\
&= 1 - \textstyle\frac{5}{6} \cdot \dots \cdot \frac{5}{6} \\
&= 1 - (\textstyle\frac{5}{6})^4.
\end{aligned}$$
\end{example}

Independence of the random variables $X_1$, $X_2$, $X_3$, and $X_4$ is used to pass from the intersection on the second line to the product on the third. Note that this example was a pure probability problem, but now that we have access to the language of random variables, we can refer to the outcome on each die in a simple and precise way, without having to define the relevant events, which helps make the solution as clear as possible. 

\section{Expected Value}

Suppose we flip a fair coin three times and count the number of heads that appear. If we repeat this experiment many times, what do we expect the average number of heads to be?

We saw in the last section that if $X$ is the number of heads observed, then $X$ is a discrete random variable whose distribution is given in the table below.
\renewcommand*{\arraystretch}{1.35}
\begin{center}
\begin{tabular}{c|c}
$y$ & $P(Y = y)$ \\
\hline
$0$ & $\frac{1}{8}$ \\
$1$ & $\frac{3}{8}$ \\
$2$ & $\frac{3}{8}$ \\
$3$ & $\frac{1}{8}$ \\
\end{tabular}
\end{center}
\renewcommand*{\arraystretch}{1}
If we perform the experiment many times, then we should expect to observe no heads about $12.5\%$ of the time, one head $37.5\%$ of the time, and so on. If we weight each possible number of heads by the probability it occurs, we obtain
$$0 \cdot \frac{1}{8} + 1 \cdot \frac{3}{8} + 2 \cdot \frac{3}{8} + 3 \cdot \frac{1}{8} = \frac{12}{8} = 1.5.$$

Not surprisingly we should, on average, obtain 1.5 heads in every three flips of a fair coin. This notion of the average value of a random variable is formalized in the definition below.

\begin{definition}\label{expectedvaluedef}\index{Expected Value}
Given a discrete random variable $X$ with range $A = \{x_1, x_2, x_3, \dots\}$, the \newterm{expected value} of $X$, also called the \newterm{mean} of X, is denoted $E(X)$ or $\mu_X$, and is computed by multiplying each possible realization of X with its probability and summing the results.
$$E(X) = \sum_{x \in A}^{\ } x \cdot P(X = x) = \sum_{x \in A} x \cdot f_X(x)$$
\end{definition}

\begin{warning}
The sum in the definition above could be infinite, in which case it may or not converge. If the sum is not absolutely convergent, then $E(X)$ is undefined.
\end{warning}

\begin{remark}
Given a statistical variable in a dataset with $n$ individuals, if we define a random variable $X$ to be the value that variable takes on a randomly selected individual in the dataset, then $E(X)$ is indeed the mean of that variable, as defined in Section \ref{MeasuresOfCenterSec}, so the definition above agrees with, and generalizes, the definition of the mean given there.
\end{remark}

\begin{example}
Alice needs to decide whether to buy individual metro tickets at \$3.25 each, or an unlimited weekend pass at \$13.75. If she knows she'll take the metro either three, four, or five times this weekend with equal probability, what is the best decision?

Let $X$ represent the amount of money Alice will spend if she buys individual tickets. Then $X$ is equally likely to take the three values $9.75$, $13$, and $16.25$. Therefore,
$$E(X) = 9.75 \cdot \frac{1}{3} + 13 \cdot \frac{1}{3} + 16.25 \cdot \frac{1}{3} = 13.$$

On the other hand, if she purchases the unlimited pass, she'll pay $\$13.75$. Thus, if Alice is interested in saving money on average, she should opt for the individual tickets.
\end{example}

\begin{example}
Consider the random variable $Z$ whose cdf is given. Find $E(Z)$.
\renewcommand*{\arraystretch}{1.35}
$$F_Z(z) = \left\{
\begin{array}{cl}
      0 & \text{ if \ } z < -3 \\
			\frac{1}{3} & \text{ if \ } -3 \leq z < 1 \\
			\frac{1}{2} & \text{ if \ } 1 \leq z < 5 \\
			\frac{5}{6} & \text{ if \ } 5 \leq z < 6 \\
      1 & \text{ if \ } z \geq 6 \\ \end{array} 
\right.$$
\renewcommand*{\arraystretch}{1}

Since $P(Z = a) = F_Z(a) - F_Z(a-)$ we can recover the pmf $f_Z$ by examining the discontinuities of $F_Z$. From the pmf we compute $E(Z)$.
\renewcommand*{\arraystretch}{1.35}
$$f_Z(z) = \left\{
\begin{array}{cl}
			\frac{1}{3} & \text{ if \ } z = -3 \\
			\frac{1}{6} & \text{ if \ } z = 1 \\
			\frac{1}{3} & \text{ if \ } z = 5 \\
      \frac{1}{6} & \text{ if \ } z = 6 \\ \end{array} 
\right.$$
\vspace{1em}
$$E(Z) = \sum_{z \in A} z \cdot f_Z(z) = -3 \cdot \frac{1}{3} + 1 \cdot \frac{1}{6} + 5 \cdot \frac{1}{3} + 6 \cdot \frac{1}{6} = \frac{11}{6}$$
\renewcommand*{\arraystretch}{1}
\end{example}

One immediate application of expected value is to games of chance. Given such a game, if we let $X$ represent the player's profit from playing the game once, then if $E(X) > 0$ it's a \newterm{winning bet}, if $E(X) < 0$ it's a \newterm{losing bet}, and if $E(X) = 0$, it's a \newterm{fair game}. 

\begin{example}
On a roulette wheel, there are 18 black numbers, 18 red numbers, and 2 green numbers. Any wager made on black will double if successful. Find the expected value of a \$20 bet on black, assuming each of the numbers on the roulette wheel is equally likely to be selected.

Let $X$ represent the profit from a bet of $\$20$ on black. Then $X$ can take only values in $A = \{20, -20\}$ since the bet is either won or lost. There are a total of 38 possible outcomes, 18 of which result in a win, so $P(X = 20) = \frac{18}{38}$ and $P(X = -20) = \frac{20}{38}$.
$$E(X) = 20 \cdot \frac{18}{38} + (-20) \cdot \frac{20}{38} = -\frac{40}{38} = -1.05$$

Therefore, this is a losing bet. If you were to sit at a roulette table and repeatedly make \$20 bets on black for a very long time, you should expect your long-term losses to total about \$1 per game played. Conversely for the casino, they should expect to make a \$1 profit for every \$20 bet on black played.
\end{example}

\begin{example}
(St-Petersburg lottery) A coin is flipped until the first tail appears. You are initially given \$1 before the coin is flipped for the first time, and every time a head appears, the prize doubles. When the first tail appears, you receive the prize. If the sequence of flips that appears is $HHHT$, for example, you would receive \$8. What is the expected value of the prize?

If we define a random variable $X$ on the sample space $\Omega = \{T, HT, HHT, \dots\}$ whose value is the prize, then $X$ can take any value in $A = \{1,2,4,8,16, \,\dots\}$, i.e., any value of the form $2^k$ for $k \in \mathbb{N}$. Furthermore, $X = 2^k$ when the outcome is a sequence with $k$ heads followed by a tail. Thus, $P(X = 2^k) = \left(\frac{1}{2}\right)^{k+1}$.
$$\begin{aligned}E(X) &= \sum_{x \in A} x P(X = x) \\
&= \sum_{k = 0}^{\infty} 2^k P(X = 2^k) \\
&= 1 \cdot \frac{1}{2} + 2 \cdot \frac{1}{4} + 4 \cdot \frac{1}{8} + 8 \cdot \frac{1}{16}  + \cdots \\
&= \frac{1}{2} + \frac{1}{2} + \frac{1}{2} + \frac{1}{2} + \cdots \rightarrow \infty\end{aligned}$$

The expected value is undefined, since the sum diverges by growing arbitrarily large. 
\end{example}

This is a very strange result, since it implies that even if you're required to pay \$1\,000\,000 to play this game, it's still a winning bet. How can this be?

Notice that this is not a game one could actually play, as there's only a finite amount of money in the world. If this example is redone with an upper limit on the payout, the expected value will be defined. Using the current global GDP as the payout limit, the result is around \$45. It's also worth considering that someone with \$\,$2 \times 10^{100}$ has twice as much money as someone with \$\,$1 \times 10^{100}$ in a literal sense, but both could spend the rest of their lives throwing briefcases of money into a fire and neither would make a dent in their net worth. In practice, twice as much money doesn't have twice as much utility. 

Regardless, if one were actually able to play this game as stated, and play it as many times as one desires, it would be rational do so even if every play cost \$1\,000\,000, since one would \emph{eventually} win an amount so large that the sum of all prior losses would be rendered insignificant.

%Example!

\section{Variance}

The expected value gives a natural measure of center for random variables. In fact, if the lollipops in the lollipop plot of the pmf $f_X$ each represent literal masses whose weights are proportional to their heights, then the value of $E(X)$ is the point on the $x$-axis where the distribution balances, and that's certainly a reasonable definition for the center.

We can also use the expected value to quantify the amount of dispersion present in the distribution of a random variable. Consider for example the two random variables $X$ and $Y$ whose pmfs are given on the left and right respectively.
\begin{center}
\begin{tikzpicture}[scale = 0.6]
\begin{axis} [
	tick label style={font=\scriptsize, scale = 1/0.6},
    xtick={0,1,...,7,8},
    ytick = {0.1,0.2,0.3,0.4},
    xmin=0, xmax=8,ymin = 0, ymax = 0.4]
\addplot+[ycomb] plot coordinates {(1, 0.01) (2, 0.02) (3, 0.04) (4, 0.12) (5, 0.36) (6, 0.30) (7, 0.07) (8, 0.03) (9, 0.02) (10, 0.01)}; 
\end{axis}
\end{tikzpicture}\qquad\qquad\begin{tikzpicture}[scale = 0.6]
\begin{axis} [
	tick label style={font=\scriptsize, scale = 1/0.6},
    xtick={0,1,...,11},
    ytick = {0.1,0.2},
    xmin=0, xmax=11,ymin = 0, ymax = 0.2]
\addplot+[ycomb] plot coordinates {(1, 0.10) (2, 0.07) (3, 0.09) (4, 0.12) (5, 0.08) (6, 0.03) (7, 0.12) (8, 0.15) (9, 0.1) (10, 0.05)}; 
\end{axis}
\end{tikzpicture}
\end{center}

The values of $X$ tend to be clustered close to the center of the distribution, only very rarely straying under 4 or over 7. On the other hand, $Y$ takes on values far from the center of its distribution quite often. In other words, the distance between the value of $X$ and the center of its distribution is usually smaller than the distance between the value of $Y$ and its center of its distribution. 

In Section \ref{MeasuresOfDispersionSec} we defined the variance of a statistical variable in a dataset as the average squared distance from the mean, and we'll define the variance of a random variance analagously, as the expected squared distance between a random variable and its expected value.

\begin{definition}\index{Random Variable!Variance of}\index{Random Variable!Standard Deviation of}\index{Variance}\index{Standard Deviation}\label{VarianceDefRV} The \newterm{variance} of a discrete random variable $X$ is denoted $\Var(X)$ or $\sigma^{2}_{X}$, and given by 
$$\Var(X) = E[(X - \mu_X)^2] = \sum_{x \in A} (x - \mu_X)^2 f_X(x).$$
The \newterm{standard deviation} of $X$ is then $\sigma_X = \sqrt{\sigma^{2}_X} = \sqrt{\Var(X)}$.
\end{definition}

\begin{remark}
In this section, many expressions are made clearer by using the notation $\mu_X$, or simply $\mu$, in place of $E(X)$, so keep in mind that $\mu_X$ is the expected value of $X$, which is a constant value at the center of the distribution of $X$.
\end{remark}

\begin{example} Determine the expected value, variance, and standard deviation of the random variable $X$ whose pmf is given below.
\renewcommand*{\arraystretch}{1.35}
$$f_X(x) = \left\{
\begin{array}{cl}
      \frac{2}{8} & \text{ if \ } x = 1 \\
			\frac{1}{8} & \text{ if \ } x = 3 \\
			\frac{3}{8} & \text{ if \ } x = 5 \\
			\frac{2}{8} & \text{ if \ } x = 6 \\
      0 & \text{ otherwise} \\ \end{array} 
\right.$$
\renewcommand*{\arraystretch}{1}

First, $E(X) = 1 \cdot \frac{2}{8} + 3 \cdot \frac{1}{8} + 5 \cdot \frac{3}{8} + 6 \cdot \frac{2}{8} = 4$, that is, $\mu_X = 4$. Then we can calculate the variance as
$$\begin{aligned}\Var(X)&=E[(X-\mu_x)^2] \\
&=(1-4)^2 \cdot \frac{2}{8} + (3-4)^2 \cdot \frac{1}{8} + (5-4)^2 \cdot \frac{3}{8} + (6-4)^2 \cdot \frac{2}{8} \\
&=9 \cdot \frac{2}{8} + 1 \cdot \frac{1}{8} + 1 \cdot \frac{3}{8} + 4 \cdot \frac{2}{8} = \frac{30}{8}.\end{aligned}$$

Thus, the standard deviation of $X$ is $\sigma_X = \sqrt{\frac{30}{8}} \approx 1.936$.
\end{example}

\begin{example} In Big Win Bob's game, each player pays $\$1$ to place a bet on a number between 1 and 100, a winning number is selected at random, and the prize for a correct bet is $\$75$. In Conservative Charles' game, each player pays $\$1$ to bet on a number between 1 and 4, a winning number is selected at random, and the prize for a correct bet is $\$3$. Which game would you rather play?

Let $X$ represent a player's net gain in a single play of Bob's game, and let $Y$ represent a player's net gain in a single play of Charles' game.
\eqns{E(X) &= 74 \cdot \frac{1}{100} + (-1) \cdot \frac{99}{100} = -0.25 \\
E(Y) &= 2 \cdot \frac{1}{4} + (-1) \cdot \frac{3}{4} = -0.25}

Thus, it does not matter which game you play in the long run. Over a very large number of plays the games are equivalent, either way the player loses 25\textcent\ per game on average. However, the variances of $X$ and $Y$ are not equal.
\eqns{\Var(X) &= (74 - (-0.25))^2 \cdot \frac{1}{100} + (-1-(-0.25))^2 \cdot \frac{99}{100} \approx 55.69 \\
\Var(Y) &= (2 - (-0.25))^2 \cdot \frac{1}{4} + (-1-(-0.25))^2 \cdot \frac{3}{4} \approx 1.68}

This large difference in the variances indicates Big Win Bob's game is much riskier. In Bob's game players win rarely but win big, while in Charles' game players win small amounts quite often.
\end{example}

Calculating the variance by hand can be tedious, and in practice such calculations are usually done with a computer. It's important to do a few examples by hand though, to understand the subtleties and develop some intuition. In the example above, note that the variance of Big Win Bob's game is dominated by the contribution of the term corresponding to the rare outcome where the player gains \$74, despite the tiny probability, because the difference between the rare \$74 gain and the expected gain is squared.

\begin{example}\label{stdevoneroll}
If $X$ represents the outcome of a single roll of a fair six-sided die, find $\sigma_X$.

First, we compute the mean $E(X) = 1 \cdot \frac{1}{6} + 2 \cdot \frac{1}{6} + 3 \cdot \frac{1}{6} + 4 \cdot \frac{1}{6} + 5 \cdot \frac{1}{6} + 6 \cdot \frac{1}{6} = \frac{7}{2}$. Then we calculate the variance as
$$\begin{aligned}
Var(X) &= \sum_{x \in A} \left(x - \frac{7}{2}\right)^2 \cdot \frac{1}{6} \text{\ , where } A = \{1,2,3,4,5,6\} \\
&= \left(-\frac{5}{2}\right)^2 \cdot \frac{1}{6} + \left(-\frac{3}{2}\right)^2 \cdot \frac{1}{6} + \left(- \frac{1}{2}\right)^2 \cdot \frac{1}{6} + \left(\frac{1}{2}\right)^2 \cdot \frac{1}{6} + \left(\frac{3}{2}\right)^2 \cdot \frac{1}{6} + \left(\frac{5}{2}\right)^2 \cdot \frac{1}{6} \\
&= \frac{25}{24} + \frac{9}{24} + \frac{1}{24} + \frac{1}{24} + \frac{9}{24} + \frac{25}{24} = \frac{35}{12} \\
\end{aligned}$$

The standard deviation is then $\sigma_X = \sqrt{\frac{35}{12}}\approx 1.707$.
\end{example}

\begin{proposition}If $X$ is a discrete random variable with $\Var(X) = 0$, then there is some realization $x$ of $X$ such that $P(X = x) = 1$. That is, $X$ takes a single real value with probability one.
\end{proposition}

\begin{proof}
Suppose there were two different realizations $x_1$ and $x_2$ which both have positive probabilities of occurring. Then at least one of $(x_1 - \mu_X)^2$ and $(x_2 - \mu_X)^2$ is nonzero (since $\mu_X$ cannot be equal to both $x_1$ and $x_2$), but this means at least one of the terms in the sum that defines $E[(X-\mu_X)^2]$ is positive. There are no negative terms in this sum since both squares and probabilities are always non-negative, hence $\Var(X) > 0$.
\end{proof}

\section{The Uniform Distribution}

\begin{definition}\index{Uniform Distribution}\index{Distribution!Uniform}\index{Uniform Distribution} If $X$ is a discrete random variable whose range is an arithmetic progression, and $X$ takes each value with equal probability, we say that $X$ has a \newterm{uniform distribution} on $A$. \end{definition}

\begin{example} Suppose that $X$ represents the result of a single fair die roll, then $X$ is uniformly distributed on $A = \{1,2,3,4,5,6\}$, since $P(X = x) = \frac{1}{6}$ for all $x \in A$.
\end{example}

The notation $X \sim \Uniform(n)$ indicates $X$ is a discrete random variable which is uniformly distributed on $A = \{1,2,3, \, .. \,, n\}$. We could write $X \sim \Uniform(6)$ to describe the distribution of the random variable in the example of a die roll above. 

If $X$ is uniformly distributed on the set $A$, then its pmf has the form
$$f_X(x) = \left\{
\renewcommand*{\arraystretch}{1.35}
\begin{array}{cl}
      \frac{1}{n} & \text{ if \ } x \in A \\
      0 & \text{ otherwise} \\ \end{array},
      \renewcommand*{\arraystretch}{1}
\right.$$
which yields a lollipop plot with a finite number lollipops of the equal height, spaced evenly along the horizontal axis. For example, the lollipop plot below shows the pmf of a random variable uniformly distributed on $A = \{-1,0,1,2\}$.

\begin{center}
\begin{tikzpicture}[scale = 0.6]
\begin{axis} [
	tick label style={font=\scriptsize, scale = 1/0.6},
    xtick={-2,-1,...,3},
    ytick = {0.1,0.2,0.3},
    xmin=-2, xmax=3,ymin = 0, ymax = 0.3]
\addplot+[ycomb] plot coordinates {(-1, 0.25) (0, 0.25) (1, 0.25) (2, 0.25)}; 
\end{axis}
\end{tikzpicture}
\end{center}

The expected value of a uniformly distributed variable on the set $A$ is simply the average of the values in $A$. Formally, if $X$ is uniformly distributed on the set $A = \{a_1, a_2,\, \dots \, , a_n\}$, then
$$E(X) = a_1 \cdot \textstyle\frac{1}{n} + a_2 \cdot \frac{1}{n} + \cdots + a_n \cdot \frac{1}{n} = \frac{a_1 + a_2 + \dots + a_n}{n}.$$
In particular, if $X \sim \Uniform(n)$, then $E(X) = \frac{1+2+3+\cdots+n}{n} = \frac{\frac{n(n+1)}{2}}{n} = \frac{n+1}{2}$.

\begin{remark}
When an element is being chosen from a set, and each element in that set has the same probability of being selected, we say the choice is made \emx{uniformly at random}.
 
One could `randomly' select a number from the set $A = \{0,1,2,3\}$ by flipping a coin three times and counting the number of heads, but not all outcomes would be equally likely. Typically when an author uses the words `at random' in mathematics, what's meant is that each possible outcome is equally likely, not that the outcome is the result of some unspecified random process. To be as clear as possible, it's good practice to use the term \emx{uniformly}.
\end{remark}

\section{The Bernoulli Distribution}\label{BernoulliDist}

Suppose that we perform an experiment with only two possible outcomes, usually called success and failure, and the probability of success is known to be $p$. Such an experiment is called a Bernoulli trial, after the mathematician Jacob Bernoulli. If $X$ is a random variable that counts the number of successes in a single Bernoulli trial (either zero or one), we say $X$ has a Bernoulli distribution with parameter $p$.

\begin{definition}\index{Bernoulli Distribution}\index{Distribution!Bernoulli}The random variable $X$ has a \newterm{Bernoulli distribution} with parameter $p$, written $X \sim \Bernoulli(p)$, if
$$f_X(x) = \left\{
\renewcommand*{\arraystretch}{1.35}
\begin{array}{cl}
      p & \text{ if \ } x =1 \\
      1-p & \text{ if \ } x =0 \\ 
      0 & \text{ otherwise} \\ \end{array}.
      \renewcommand*{\arraystretch}{1}
\right.$$

In other words, $X$ takes the value $1$ with probability $p$, and $0$ with probability $1-p$.
\end{definition}

Think of a Bernoulli random variable as counting the number of heads observed in a single flip of a coin with a given bias. If $X$ represents the number of heads in a single flip of a fair coin, then $X \sim \Bernoulli(0.5)$, and if $Y$ represents the number of heads in a single flip of a biased coin which lands on heads only $30\%$ of the time, then $Y \sim \Bernoulli(0.3)$. The pmf of $Y$ is graphed below.

\begin{center}
\begin{tikzpicture}[scale = 0.6]
\begin{axis} [
	tick label style={font=\scriptsize, scale = 1/0.6},
    xtick={0,1},
    ytick = {0.3,0.6,0.9},
    xmin=-0.5, xmax=1.5, ymin = 0, ymax = 0.9]
\addplot+[ycomb] plot coordinates {(0, 0.7) (1, 0.3)}; 
\end{axis}
\end{tikzpicture}
\end{center}

\begin{proposition}\label{BernoulliExpectation}\index{Bernoulli Distribution!Expected Value of}\index{Bernoulli Distribution!Variance of} If $X \sim \Bernoulli(p)$, then $E(X) = p$ and $\Var(X) = p(1-p)$.
\end{proposition}

\begin{proof} These results both follow from direct calculations. From the probability mass function above we have $E(X) = 0 \cdot (1-p) + 1 \cdot p  = p$. We can then calculate
$$\begin{aligned}\Var(X) &= E[(X - \mu_X)^2] = E[(X - p)^2] = (0-p)^2 \cdot (1-p) + (1-p)^2 \cdot p \\
&= p^2(1-p) + (1-p)^2p = p(1-p)(p+(1-p)) = p(1-p).\end{aligned}$$
\end{proof}

Bernoulli random variables are simple but fundamental in probability and statistics. The Binomial, Geometric, and Poisson distributions, all of which come up frequently in scientific practice, can be defined in terms of Bernoulli random variables. In this course, we'll cover only the Binomial distribution, the others are studied in the Probability \& Random Variables option course, 201-PRV.

\section{The Binomial Distribution}

Suppose we perform a fixed number of Bernoulli trials (call this number $n$). Each trial has the same probability of success (call this value $p$), and the results of the $n$ consecutive trials form a sequence of independent Bernoulli random variables. If $X$ counts the total number of successes after all the trials have been performed, then we say that $X$ has a Binomial distribution.

\begin{definition}\index{Binomial Distribution}\index{Distribution!Binomial} The random variable $X$ has a \newterm{binomial distribution} with parameters $n$ and $p$, written $X \sim \Binomial(n,p)$, if
$$f_X(x) = \left\{
\renewcommand*{\arraystretch}{1.35}
\begin{array}{cl}
      \displaystyle\binom{n}{x}(p)^x(1-p)^{n-x} & \text{ if \ } x = 0, 1, 2, \dots \, , n \\
      0 & \text{ otherwise } \\ \end{array}.
      \renewcommand*{\arraystretch}{1}
\right.$$
\end{definition}

Below on the left is the pmf of a binomial distribution with $n = 5$, $p = 0.5$, and on the right, the pmf of a binomial distribution with $n = 9$, $p = 0.85$.

\begin{center}
\begin{tikzpicture}[scale = 0.6]
\begin{axis} [
	tick label style={font=\scriptsize, scale = 1/0.6},
    xtick={0,1,...,5},
    ytick = {0.1,0.2,0.3,0.4},
    xmin=-1, xmax=6, ymin = 0, ymax = 0.4]
\addplot+[ycomb] plot coordinates {(0, 0.03125) (1, 0.15625) (2, 0.3125) (3, 0.3125) (4, 0.15625) (5, 0.03125)}; 
\end{axis}
\end{tikzpicture} \qquad\qquad  \begin{tikzpicture}[scale = 0.6]
\begin{axis} [
	tick label style={font=\scriptsize, scale = 1/0.6},
    xtick={0,1,...,9},
    ytick = {0.1,0.2,0.3,0.4},
    xmin=-1, xmax=10, ymin = 0, ymax = 0.4]
\addplot+[ycomb] plot coordinates {(0, 0.00000003) (1, 0.000002) (2, 0.000045) (3, 0.00058) (4,0.005) (5, 0.0283) (6, 0.10692) (7,0.2596) (8,0.3678) (9,0.2316)}; 
\end{axis}
\end{tikzpicture}
\end{center}

To derive this probability mass function, imagine recording of the results of the Bernoulli trials as an $n$ letter long sequence of $S$'s and $F$'s, representing success and failure respectively. By the Mississippi formula from Section \ref{PermutationsSec}, the number of such sequences in which $x$ successes occur is $\frac{n!}{x!(n-x)!} = \binom{n}{x}$.

Consider any such sequence, say $SFSSSFF$ for example. Here $n=7$, and $x=4$ successes occurred. The probability this particular ordered sequence occurs is not hard to calculate since we're performing independent Bernoulli trials, each with the same probability of success. Every $S$ occurs with probability $p$, and every $F$ with probability $1-p$. There are four $S$'s and three $F$'s, and the trials are independent, so this sequence occurs with probability $p^4(1-p)^3$.

In total then, there are $\binom{n}{x}$ sequences of outcomes with $x$ successes, and each sequence occurs with probability $p^x(1-p)^{n-x}$, so $P(X = x) = \binom{n}{x}p^x(1-p)^{n-x}$.

The number of heads in $n$ flips of a coin is the example that should immediately come to mind when you think of a binomial distribution, but binomial distributions also appear in many practical contexts involving samples.

\begin{example}Suppose that in a certain country's election, $56\%$ of the voting population will vote for a particular candidate, Bob. If you select seven members of the voting population uniformly at random, and ask them who they'll vote for (assume they'll answer honestly), what is the probability your poll will incorrectly conclude that Bob will win less than half the vote?

Let $X$ represent the number of people who respond by saying that they'll vote for Bob. If we regard such a response as a success, and each interview is done without knowledge of the results of any of the others, then $X \sim \Binomial(7, 0.56)$.

The result we're seeking is the probability that there are fewer than four successes, which we calculate from the pmf as follows.
$$\begin{aligned}P(X \leq 3) &= \sum_{i=0}^{3} P(X = i) = \sum_{i=0}^{3} \binom{7}{i}(0.56)^{i}(0.44)^{7-i} \\
&\approx 0.0032 + 0.0284 + 0.1086 + 0.2304 \\
&= 0.3706 \approx 37\%\end{aligned}$$

There is a $37\%$ chance that the results of this poll will incorrectly predict that Bob will not win a majority of the vote.
\end{example}

\begin{remark}
It would be easier to use the cdf to evaluate $P(X \leq 3)$, but unfortunately, the cdf for a binomial distribution does not have a simple closed form, so we're stuck summing the values of the pmf.
\end{remark}

One of the reasons the binomial distribution appears in many contexts is that the process of evaluating a claim by doing repeated independent trials, which could be interviews, simulations, or literal laboratory experiments, is such a fundamental practice.

\begin{theorem}\index{Binomial Distribution!Expected value of}\index{Binomial Distribution!Variance of}If $X \sim \Binomial(n,p)$, then $E(X) = np$ and $\Var(X) = np(1-p)$.
\end{theorem}

These formulas follow from properties of expected value and sums of random variables which are not covered in this course, but appear in the Probability \& Random Variables option course, 201-PRV.

\begin{example} Let $X$ be the number of doubles in five rolls of a pair of dice, and let $Y$ be the number of heads in three flips of a fair coin. Which quantity is larger on average? Which quantity is more variable?

$X$ is a binomial random variable with $n = 5$ and $p = \frac{6}{36} = \frac{1}{6}$, so $E(X) = np = \frac{5}{6}$ and $\Var(X) = np(1-p) = 5(\frac{1}{6})(\frac{5}{6}) = 25/36 \approx 0.694$.

$Y$ is a binomial random variable with $n = 3$ and $p = \frac{1}{2}$, so $E(Y) = np = \frac{3}{2}$ and $\Var(X) = np(1-p) = 3(\frac{1}{2})(\frac{1}{2}) = 3/4 \approx 0.75$.

Therefore, $Y$ is larger on average. The second question is somewhat ambiguous, but if we use the variance as our measure of variability, $Y$ is a little more variable then $X$.
\end{example}

\begin{example}
Edward lives in a country with an unreliable mail system, where 7\% of parcels do not make it to their destination. He wants to send two books to his brother, each with a value of \$20. If he sends them in one parcel, the postage is \$5.20, and if he sends them separately, the postage is \$3.30 for each book. To minimize his expected losses, which method is preferable?

Let $X \sim \Bernoulli(0.07)$, then sending both books in one parcel, the \$5.20 postage charge always applies, and \$40 is lost if $X = 1$. If $L$ represents Edward's loss, then $L = 5.20 + 40X$, and therefore $E(L) = 5.20 + 40E(X) = 5.20 + 40 \cdot 0.07 = 8$.

If the two books are each sent separately, then $L = 6.60 + 20Y$, where $Y \sim \Binomial(2,0.07)$. Thus, $E(L) = 6.60 + 20E(Y) = 6.60 + 20 \cdot 2 \cdot 0.07 = 9.4$, so sending both books in one package is the option with the smaller expected loss.
\end{example}

\section{Continuous Random Variables}\label{ContinuousRVSection}

In Section \ref{RVIntroSec}, we defined a random variable as a function from $\Omega$ to $\mathbb{R}$, but have so far only considered random variables that take values in some countable subset $A \subseteq \mathbb{R}$. We saw that we could specify the distribution of such a random variable using a probability mass function, or a cumulative distribution function.

Our goal is now to understand what happens when a random variable can take any value in some interval $I \in \mathbb{R}$. Consider for example a random variable $X$ which represents a number chosen uniformly at random from the interval $[0,1]$. Unfortunately, it's not possible to express the distribution of $X$ as a table, since $[0,1]$ is not a countable set, so we can't list its values one-by-one.

Fortunately, the cumulative distribution function comes to the rescue. What is $P(X \leq \frac{2}{3})$? If we divide the interval into thirds, then $X \leq \frac{2}{3}$ when $X$ falls into one of the left two thirds, which should happen two thirds of the time if $X$ is equally likely to fall anywhere in $[0,1]$.
\begin{center}
\begin{tikzpicture}
\draw (0,0) -- (3,0);
\draw[shift={(0,0)},color=black] (0pt,3pt) -- (0pt,-3pt);
\draw[shift={(3,0)},color=black] (0pt,3pt) -- (0pt,-3pt);
\foreach \x in {1,2}
\draw[shift={(\x,0)},color=black] (0pt,0pt) -- (0pt,-3pt) node[below] 
{\small$\frac{\x}{3}$};
\draw (0,-0.2) node[below]{\small$0$};
\draw (3,-0.2) node[below]{\small$1$};
\fill [pattern=north west lines, pattern color=black] (0,0.1)--(2,0.1)--(2,-0.1)--(0,-0.1);
\end{tikzpicture}
\end{center}

Thus, $P(X \leq \frac{2}{3}) = \frac{2}{3}$. By a similar argument, it should be clear that $P(X \leq x) = x$ for any $x \in [0,1]$ and therefore, the cumulative distribution function of $X$ is the function given below.

\begin{center}
    \begin{minipage}{0.5\textwidth}
        \centering
        \renewcommand*{\arraystretch}{1.35}
$$F_X(x) = \left\{
\begin{array}{cl}
      0 & \text{ if \ } x < 0  \\
      x & \text{ if \ } 0 \leq x < 1  \\
      1 & \text{ if \ } x \geq 1 \\ \end{array}
\right.$$
\vspace{0.75em}
\renewcommand*{\arraystretch}{1}
    \end{minipage}%
    \begin{minipage}{0.5\textwidth}
        \centering
    \begin{tikzpicture}[scale = 0.6]
       \begin{axis}[
       tick label style={font=\scriptsize, scale = 1/0.6},
    xtick={0,1},
    ytick = {0.2,0.4,...,1},
    xmin=-0.5, xmax=1.5, ymin = 0, ymax = 1, width=10cm, height=6cm]
       \addplot[very thick,domain=-1:0,blue] {0};
       \addplot[very thick,domain=0:1,blue] {x};
       \addplot[very thick,domain=1:2,blue] {1};
    \end{axis}
\end{tikzpicture}
\end{minipage}
\end{center}
%unit vector ratio=1 1 1, ymin=-0.0025,ymax=1.0025,xmin=-0.5, xmax = 1.5, xtick={0,1},ytick={0,0.2,0.4,0.6,0.8,1}

Using this cdf, we can determine the probability that $X$ will be in various subsets of $[0,1]$. Let's take the interval $(\frac{2}{5},\frac{2}{3})$. Using the cdf above, we can compute

\begin{center}
    \begin{minipage}{0.5\textwidth}
        \centering
        $$\begin{aligned}P(X \in (\textstyle\frac{2}{5},\frac{2}{3})) &= P(\textstyle\frac{2}{5} < X < \frac{2}{3}) \\
&= P(X < \textstyle\frac{2}{3}) - P(X \leq \frac{2}{5}) \\
&= F_X(\textstyle\frac{2}{3} - ) - F_X(\frac{2}{5}) \\
&= \textstyle \frac{2}{3} - \frac{2}{5} = \frac{4}{15}\end{aligned}$$
\vspace{0.25em}
    \end{minipage}%
    \begin{minipage}{0.5\textwidth}
        \centering
    \begin{tikzpicture}[scale = 0.6]
       \begin{axis}[tick label style={font=\scriptsize, scale = 1/0.6},
    xtick={0,2/5,2/3,1},
    xticklabels={$0$,$\frac{2}{5}$,$\frac{2}{3}$,$1$},
    ytick = {0.2,0.4,...,1},
    xmin=-0.5, xmax=1.5, ymin = 0, ymax = 1, width=10cm, height=6cm]
       \addplot[very thick,domain=-1:0,blue] {0};
       \addplot[name path = f, very thick,domain=0:1,blue] {x};
       \addplot[very thick,domain=1:2,blue] {1};
       \draw[black] (axis cs: 2/3,0) -- (axis cs: 2/3,2/3);
       \draw[black] (axis cs: 2/5,0) -- (axis cs: 2/5,2/5);
    \end{axis}
\end{tikzpicture}
\end{minipage}
\end{center}


\begin{keypoint}
If the cdf of $X$ is continuous, then $\lim_{x \to a^{-}}F_X(x) = \lim_{x \to a^{+}}F_X(x) = F_X(a)$, and hence $F_X(a-)$, $F_X(a+)$ and $F_X(a)$ are all equal. This implies a random variable $X$ with a continuous cdf will take any fixed realization with probability zero, since
$$P(X = x) = P(X \leq x) - P(X < x) = F_X(x) - F_X(x-) = 0.$$

Furthermore, $P(X \in (a,b))$, $P(X \in [a,b))$, $P(X \in (a,b])$, and $P(X \in [a,b])$ are all equal to $F_X(b) - F_X(a)$. Including or excluding endpoints makes no difference.
\end{keypoint}

\begin{definition}
If $X$ is a random variable whose cumulative distribution function $F_X(x) = P(X \leq x)$ is continuous, then we say $X$ is a \newterm{continuous random variable}\index{Continuous Random Variable}.
\end{definition}

\begin{example}\label{ContinuousCDFExample} Let X be a random variable with cdf given below. Find $P(X \in (\frac{1}{3}, 1) \cup [\frac{3}{2}, \frac{5}{2}])$.
\vspace{-1em}
\begin{center}
    \begin{minipage}{.5\textwidth}
        \centering
        \renewcommand*{\arraystretch}{1.35}
\eqns{F_X(x) = \left\{
\begin{array}{cl}
      0 & \text{ if \ } x < 0  \\
      \frac{1}{2}x & \text{ if \ } 0 \leq x < 1  \\
      \frac{1}{2} & \text{ if \ } 1 \leq x < 2 \\ 
      x - \frac{3}{2} & \text{ if \ } 2 \leq x < \frac{5}{2} \\ 
      1 & \text{ if \ } x \geq \frac{5}{2} \\ \end{array}
\right.}
\vspace{1.25em}
\renewcommand*{\arraystretch}{1}
    \end{minipage}%
    \begin{minipage}{0.5\textwidth}
        \centering
    \begin{tikzpicture}[scale = 0.6]
       \begin{axis}[tick label style={font=\scriptsize, scale = 1/0.6},
    xtick={0,0.5,...,3},
    ytick = {0.5,1},
    xmin=-0.5, xmax=3.5, ymin = 0, ymax = 1, width=10cm, height=6cm]
       \addplot[very thick,domain=-1:0,blue] {0};
       \addplot[very thick,domain=0:1,blue] {0.5*x};
       \addplot[very thick,domain=1:2,blue] {0.5};
       \addplot[very thick,domain=2:2.5,blue] {x -1.5};
       \addplot[very thick,domain=2.5:3.5,blue] {1};
    \end{axis}
\end{tikzpicture}
\end{minipage}
\end{center}

Since the events $X \in (\frac{1}{3}, 1)$ and $X \in [\frac{3}{2}, \frac{5}{2}]$ are mutually exclusive,
$$\begin{aligned}P(X \in (\textstyle\frac{1}{3}, 1) \cup [\frac{3}{2},\frac{5}{2}]) &= P(X \in (\textstyle\frac{1}{3}, 1)) + P(X \in [\frac{3}{2},\frac{5}{2}]) \\
%&= P(X < 1) - P(X \leq \textstyle\frac{1}{3}) + P(X \leq  \frac{5}{2}) - P(X < \frac{3}{2}) \\
&= F_X(1) - F_X(\textstyle\frac{1}{3}) + F_X( \frac{5}{2}) - F_X(\frac{3}{2}) \\
&= \textstyle\frac{1}{2} - \frac{1}{2}\cdot\frac{1}{3} + 1 - \frac{1}{2} \\
&= \textstyle\frac{1}{2} - \frac{1}{6} + 1 - \frac{1}{2} = \frac{5}{6}.\end{aligned}$$
\end{example}

\subsection*{Probability Density Functions}
Consider a continuous random variable $X$, and two real numbers $r$ and $s$. We now know that $P(X = r)$ and $P(X = s)$ are both zero, but we can still ask whether $X$ is more likely to end up close to $r$ or close to $s$.

We can formalize this by asking for the probability $X$ lies in each of the tiny intervals $(s-\epsilon,\, s + \epsilon)$ and $(r-\epsilon, \,r+ \epsilon)$, for some very small positive number $\epsilon$. Take for instance the random variable $X$ in Example \ref{ContinuousCDFExample} above. Is it more likely this variable takes a value close to $\frac{1}{2}$, or a value close to $\frac{9}{4}$?

\begin{center}
\begin{tikzpicture}[scale = 0.6]
       \begin{axis}[tick label style={font=\scriptsize, scale = 1/0.6},
    xtick={1/2,9/4},
    xticklabels={\small$\frac{3}{2}$,\small$\frac{9}{4}$},
    ytick = {0.5,1},
    xmin=-0.5, xmax=3.5, ymin = 0, ymax = 1, width=12cm, height=8cm]
       \addplot[very thick,domain=-1:0,blue] {0};
       \addplot[very thick,domain=0:0.38,blue] {0.5*x};
       \addplot[very thick,domain=0.38:0.62,red] {0.5*x};
       \addplot[very thick,domain=0.62:1,blue] {0.5*x};
       \addplot[very thick,domain=1:2,blue] {0.5};
       \addplot[very thick,domain=2:2.13,blue] {x -1.5};
       \addplot[very thick,domain=2.13:2.37,red] {x -1.5};
       \addplot[very thick,domain=2.37:2.5,blue] {x -1.5};
       \addplot[very thick,domain=2.5:3.5,blue] {1};
       \node[label={},circle,fill,inner sep=1pt,red] at (axis cs:0.38,0.19) {};
       \node[label={},circle,fill,inner sep=1pt,red] at (axis cs:0.62,0.31) {};
       \node[label={},circle,fill,inner sep=1pt,red] at (axis cs:2.13,0.63) {};
       \node[label={},circle,fill,inner sep=1pt,red] at (axis cs:2.37,0.87) {};
    \end{axis}
    \draw[thick,red] (2.9,0) node {\small )};
    \draw[thick,red] (2.3,0) node {\small (};
    \draw[red,thick] (2.3,0)  -- (2.9,0);
    \draw[thick,red] (7.45,0) node {\small )};
    \draw[thick,red] (6.85,0) node {\small (};
    \draw[red,thick] (7.45,0)  -- (6.85,0);
\end{tikzpicture}
\end{center}

The probability $X$ lies in the interval $(a,b)$ is $F_X(b) - F_X(a)$, the difference in the value of the cdf at the endpoints. For the cdf shown above, the graph is twice as steep when it's near $\frac{9}{4}$ than it is when it's near $\frac{1}{2}$, and hence this difference is twice as large. Therefore, $X$ is twice as likely to assume values near $\frac{9}{4}$ than it is to assume values near $\frac{1}{2}$.

In general then, the probability $X$ lies close to a given value is proportional to the slope of the cdf at that value. But we know how to measure the slope of any function, with the derivative. Thus, given any $x \in \mathbb{R}$, if we evaluate the derivative of $F_X$ at $x$, the larger the result, the more likely it is that $X$ takes a value very close to $x$. Remember that the probability $X=x$ must be zero for all $x \in \mathbb{R}$, so when we evaluate the derivative of $F_X$ at $x$, the result is not a probability, instead it's known as a probability density, and the derivative of $F_X$ is called the probability density function of $X$. 

\begin{definition}\index{Probability Density Function}\index{Probability Density}If $X$ is a continuous random variable with cdf $F_X$, the \newterm{probability density function} of $X$ is denoted $f_X$ and is defined by $f_X(x) = \frac{d}{dx}{F_X}(x)$.
\end{definition}

The cdf $F_X$ of the continuous random variable $X$ defined in Example \ref{ContinuousCDFExample} above is shown once again on the left, and the corresponding probability density function $f_X$ is on the right. It's standard practice to draw vertical lines on the graph of the probability density function as below. These lines are technically not part of the graph (they are jump discontinuities), but make for a more appealing image.

\begin{center}
    \begin{minipage}{.5\textwidth}
        \centering
       \begin{tikzpicture}[scale = 0.6]
       \begin{axis}[tick label style={font=\scriptsize, scale = 1/0.6},
    xtick={0,0.5,...,3},
    ytick = {0.5,1},
    xmin=-0.5, xmax=3.5, ymin = 0, ymax = 1, width=10cm, height=6cm]
       \addplot[very thick,domain=-1:0,blue] {0};
       \addplot[very thick,domain=0:1,blue] {0.5*x};
       \addplot[very thick,domain=1:2,blue] {0.5};
       \addplot[very thick,domain=2:2.5,blue] {x -1.5};
       \addplot[very thick,domain=2.5:3.5,blue] {1};
    \end{axis}
\end{tikzpicture}
    \end{minipage}%
    \begin{minipage}{0.5\textwidth}
        \centering
\begin{tikzpicture}[scale = 0.6]
       \begin{axis}[tick label style={font=\scriptsize, scale = 1/0.6},
    xtick={0,0.5,...,3},
    ytick = {0.5,1},
    xmin=-0.5, xmax=3.5, ymin = 0, ymax = 1.25, width=10cm, height=6cm]
       \addplot[very thick,domain=-1:0,blue] {0};
       \addplot[very thick,domain=0:1,blue] {0.5};
       \addplot[very thick,domain=1:2,blue] {0};
       \addplot[very thick,domain=2:2.5,blue] {1};
       \addplot[very thick,domain=2.5:3.5,blue] {0};
       \draw[very thick, blue, -] (axis cs:0,0) -- (axis cs:0,0.5);
       \draw[very thick, blue, -] (axis cs:1,0) -- (axis cs:1,0.5);
       \draw[very thick, blue, -] (axis cs:2,0) -- (axis cs:2,1);
       \draw[very thick, blue, -] (axis cs:2.5,0) -- (axis cs:2.5,1);
    \end{axis}
    \end{tikzpicture}
\end{minipage}
\end{center}

We'll often use pdf to abbreviate probability density function. The rationale for the term `probability density' is the following: when considering a rod of metal or other material, we can only ask for the mass of a contiguous piece of the material, since the mass at a point is always zero. We can, however, consider the density of the material to be well defined at a point, as the rate of change in mass at that point as we move along the rod. The situation with continuous random variables is analogous: a probability density measures the rate of change of the cdf as we move along it, in other words, the rate at which probability is accumulating as we move past that point.

\begin{proposition}If $X$ is a continuous random variable, then the probability $X$ lies in the interval $(a,b)$ is the area under $f_X$ on $(a,b)$.
\end{proposition}

\begin{proof} Calculating $P(X \in (a,b))$ using the cdf $F_X$ as we have done in the examples above, and applying the fundamental theorem of calculus, we have
$$P(X \in (a,b)) = F_X(b) - F_X(a) = \int_{a}^{b} \frac{d}{dx} F_X(x) \, dx = \int_{a}^{b} f_X(x) \, dx.$$
This integral is the area under $f_X$ on $(a,b)$. Note that the result would be the same with any of the intervals $(a,b]$, $[a,b)$, or $[a,b]$. \end{proof}

\begin{example} Let X be the continuous random variable whose pdf is given below. Find $P(X > \frac{3}{2})$.

\begin{center}
    \begin{minipage}{.5\textwidth}
        \centering
        \renewcommand*{\arraystretch}{1.35}
\eqns{f_X(x) = \left\{
\begin{array}{cl}
      \frac{2}{x^2} & \text{ if \ } 1 \leq x \leq 2  \\
      0 & \text{ otherwise \ }  \\ \end{array}
\right.}
\vspace{1.25em}
\renewcommand*{\arraystretch}{1}
    \end{minipage}%
    \begin{minipage}{0.5\textwidth}
        \centering
    \begin{tikzpicture}[scale =0.6]
       \begin{axis}[tick label style={font=\scriptsize, scale = 1/0.6},
    	xtick={1,2},
    	ytick = {1,2},
    	xmin=0.75, xmax=2.25, ymin = 0, ymax = 2.5, width=10cm, height=6cm]
        \addplot[fill = blue, very thick,domain=1.5:2,blue!50, samples=100] {2/(x*x)}\closedcycle;
       \addplot[very thick,domain=1:2,blue, samples=100] {2/(x*x)};
       \draw[very thick, blue, -] (axis cs:1,2) -- (axis cs:1,0);
       \draw[very thick, blue, -] (axis cs:2,0.5) -- (axis cs:2,0);
       \addplot[domain=-3:3] {0};
    \end{axis}
\end{tikzpicture}
\end{minipage}
\end{center}

$$P(X > \textstyle\frac{3}{2}) = \displaystyle\int_{\frac{3}{2}}^{2} f_X(x) \, dx = \int_{\frac{3}{2}}^{2} \frac{2}{x^2} \, dx =  \left.-\frac{2}{x}\,\right|_{\frac{3}{2}}^{2} = -\frac{2}{2} - \left(- \frac{2}{\frac{3}{2}}\right) = \frac{1}{3}$$
\end{example}

\begin{keypoint}
The area above the interval $I$ under the probability density function $f_X$ is the probability $X$ takes a value in $I$. This is why the probability density function is so useful. It gives a representation of the distribution of a continuous random variable which is as geometrically intuitive as a histogram, and allows us to apply the tools of calculus directly to compute probabilities.
\end{keypoint}

\begin{proposition}If $X$ is any continuous random variable with cdf $F_X$, the pdf $f_X$ satisfies each of the properties below.
\vspace{-0.25em}
\begin{enumerate}
\item For all $x \in \mathbb{R}$, $f_X(x) \geq 0$.
\item $\int_{-\infty}^{\infty} f_X(x) \, dx = P(-\infty < X < \infty) = 1$.
\item $\int_{-\infty}^{t} f_X(x) \, dx = P(X \leq t) = F_X(t)$.
\end{enumerate}
\end{proposition}

Informally, every non-negative function which encloses an area of one between its graph and the horizontal axis is the pdf of some continuous random variable, the cdf of which can be recovered through integration, as in the third property above.

\begin{example}Consider the function $f$ given below. Find $c$ so that this function is the pdf of some continuous random variable.
\renewcommand*{\arraystretch}{1.35}
\eqns{f(x) = \left\{
\begin{array}{cl}
      cxe^{-x} & \text{ if \ } x > 0  \\
      0 & \text{ otherwise \ }  \\ \end{array}
\right.}
\renewcommand*{\arraystretch}{1}

Since $e^{x}>0$ always, we have $xe^{-x} > 0$ when $x > 0$, and thus $f$ is non-negative when $c > 0$. We must find a positive value for $c$ such that the area under $f$ is one, so we evaluate the area under $f$.
$$\int_{-\infty}^{\infty} f(x) \, dx = \int_{0}^{\infty} cxe^{-x} \, dx =c \int_{0}^{\infty} xe^{-x} \, dx =  c(-x-1)e^{-x} \biggr|_{0}^{\infty}$$

The last equality comes from using integration by parts. To complete the evaluation, we can use L'H\^{o}pital's rule.
$$\begin{aligned}c(-x-1)e^{-x} \biggr|_{0}^{\infty} &= \lim_{x \to \infty} \left(c(-x-1)e^{-x}\right) - c(-0-1)e^{0}\\
&= c \lim_{x \to \infty} \left( \textstyle\frac{-x-1}{e^{x}}\right) + c \\
&\stackrel{H}{=} c \lim_{x \to \infty} \left( \textstyle\frac{-1}{e^{x}}\right) + c \\
&= 0 + c = c\end{aligned}$$

The area under $f$ is simply $c$, and therefore if we take $c=1$, $f$ is a pdf. The graph of $f$ with $c = 1$ is given below.
\begin{center}
\begin{tikzpicture}[scale=0.6]
       \begin{axis}[tick label style={font=\scriptsize, scale = 1/0.6},
    	xtick={0,1,...,5},
    	ytick = {0.2,0.4},
    	xmin=0, xmax=5, ymin = 0, ymax = 0.5, width=10cm, height=6cm]
       \addplot[thick,domain=0:5,blue, samples = 200] {x*(e^(-x))};
    \end{axis}
\end{tikzpicture}
\end{center}
\end{example}

As you can see, the material that was covered in Calculus I \& II will be very relevant in the world of continuous random variables. If you're a bit rusty, don't be worried. Using the tools of calculus to work with general continuous random variables is a topic we'll not go into in this course, and instead leave for the Probability \& Random Variables option course, 201-PRV. It's also worth noting that the integrals appearing in problems involving continuous random variables are often either very simple, or not possible to evaluate using the methods of elementary calculus (we'll see an example in the next section). The example above was created specifically to remind you about integration by parts and L'H\^{o}pital's rule.

\subsection*{Expectation for Continuous Random Variables}

The expected value of a continuous random variable $X$ is defined in essentially the same way the expected value of a discrete random variable, but in place of a probability mass function we have a probability density function, and in place of a sum we have the continuous analogue, an integral.

\begin{definition}\index{Expected Value}\index{Expected Value! of a Continuous R.V.}
If $X$ is a continuous random variable, the expected value of $X$ is denoted $E(X)$ or $\mu_X$, and defined by
$$E(X) = \int_{-\infty}^{\infty}x f_X(x) \, dx.$$

Note that this improper integral may not converge, in which case the expected value is undefined.
\end{definition}

\begin{example}
Find the expected value of the random variable $X$ whose pdf is given by $f_X(x) = 2/x^2$ on $[1,2]$.
$$E(X) = \int_{-\infty}^{\infty}x f_X(x) \, dx = \int_{1}^{2}x \,\frac{2}{x^2} \, dx = 2\int_{1}^{2}\frac{1}{x}\, dx = 2 \ln(x) \bigr|_{1}^{2} = 2\ln(2)$$
\end{example}

\begin{example}\label{CauchyExpectation}
Consider the random variable $X$ whose pdf is given below. Show $E(X)$ is undefined.
\begin{center}
    \begin{minipage}{.5\textwidth}
        \centering
\eqns{f_X(x) = \frac{1}{\pi(1+x^2)}\text{\ \ on \ }(-\infty,\infty)}
\vspace{1.25em}
    \end{minipage}%
    \begin{minipage}{0.5\textwidth}
        \centering
    \begin{tikzpicture}[scale=0.6]
       \begin{axis}[tick label style={font=\scriptsize, scale = 1/0.6},
    xtick={-2,-1,...,2},
    ytick = {0.1,0.2,0.3,0.4},
    ymin=0,ymax=0.4,xmin=-2.5, xmax = 2.5, width=10cm, height=6cm]      
       \addplot[thick,domain=-3:3,blue,samples=200] {1/(3.1415*(1+x*x))};
    \end{axis}
\end{tikzpicture}
\end{minipage}
\end{center}

$$E(X) = \int_{-\infty}^{\infty} x f_X(x) \, dx = \int_{-\infty}^{\infty} x \frac{1}{\pi(1+x^2)} \, dx = \frac{1}{\pi}\int_{-\infty}^{\infty} \frac{x}{1+x^2} \, dx$$

To evaluate a `doubly improper' integral like this, we must split it into two improper integrals to see if both of those converge. In the two resulting integrals, we make the substitution $u = 1+x^2$, so $du = 2x \, dx$ and hence $dx = \frac{1}{2x}du$.
$$\begin{aligned}E(X) &= \frac{1}{\pi}\int_{-\infty}^{\infty} \frac{x}{1+x^2} \, dx \\
&= \frac{1}{\pi}\int_{-\infty}^{0} \frac{x}{1+x^2} \, dx + \frac{1}{\pi}\int_{0}^{\infty} \frac{x}{1+x^2} \, dx \\
&= \frac{1}{\pi}\int_{\infty}^{1} \frac{x}{u} \frac{1}{2x}\, du + \frac{1}{\pi}\int_{1}^{\infty} \frac{x}{u} \frac{1}{2x}\, du \\
&= \frac{1}{2\pi}\int_{\infty}^{1} \frac{1}{u} \, du + \frac{1}{2\pi}\int_{1}^{\infty} \frac{1}{u} \, du \\ 
&= \frac{1}{2\pi}\ln(u) \biggr|_{\infty}^{1} + \frac{1}{2\pi}\ln(u) \biggr|_{1}^{\infty} \\
&= \frac{1}{2\pi}\left(\ln(1) - \lim_{u \to \infty}ln(u)\right) + \frac{1}{2\pi}\left(\lim_{u \to \infty}ln(u) - \ln(1)\right)\end{aligned}$$

Neither of the two improper integrals converge since $\lim_{u \to \infty} \ln(u) = \infty$, and hence $E(X)$ is undefined. 
\end{example}

\begin{remark}
This result may seem odd, since you would intuitively place the center of this distribution at zero. It turns out that there exist distributions (this one is known as a Cauchy distribution) that have so much area in their tails that they have an undefined mean. These \newterm{fat-tailed}\index{Distribution!Fat-tailed} distributions are very interesting, as most of the usual estimation methods we'll see later on in the course don't work well, or even at all, when the variable being studied has a fat-tailed distribution.
\end{remark}

As with the expected value, the variance of a continuous random variable is defined in essentially the same way as in the discrete case, Definition \ref{VarianceDefRV}, but with the an integral in place of the sum. The variance is the average squared distance from the mean, but now we're taking the average value of a continuous function.

\begin{definition}\index{Variance! of a Continuous R.V.}
If $X$ is a continuous random variable, the variance of $X$ is denoted $\Var(X)$ or ${\sigma_X}^2$, and defined by
$$\Var(X) = \int_{-\infty}^{\infty}(x-\mu_X)^2 f_X(x) \, dx.$$
The standard deviation, $\sigma_X$, is the square root of the variance.
\end{definition}

\begin{example}\label{MeanVarianceUnitInterval}
Let $X$ be a number chosen uniformly at random from the interval $[0,1]$. Find the mean and standard deviation of $X$.

The pdf of $X$ can be obtained by differentiating the cdf we came up with in the discussion at the start of this section. Doing so gives the function below.
$$f_X(x) = \left\{
\begin{array}{cl}
      1 & \text{ if \ } 0 < x < 1  \\
      0 & \text{ otherwise \ }  \\ \end{array}
\right.$$
Now we can calculate the mean and variance of $X$ as follows.
$$E(X) = \int_{-\infty}^{\infty} x f_X(x) \, dx = \int_{0}^{1} x \cdot 1 \, dx = \textstyle\frac{1}{2}x^2 \,\bigr|_{0}^{1}=\frac{1}{2}$$
$$\Var(X) = \int_{-\infty}^{\infty} (x-\textstyle\frac{1}{2})^2 f_X(x) \, dx = \displaystyle\int_{0}^{1} (x-\textstyle\frac{1}{2})^2  \cdot 1 \, dx = \displaystyle\int_{0}^{1} x^2-x+\textstyle\frac{1}{4} \, dx = \textstyle\frac{1}{3}x^3 - \frac{1}{2}x^2 + \frac{1}{4}x \,\bigr|_{0}^{1}=\frac{1}{12}$$
Therefore, the mean is $\mu = \frac{1}{2}$ and the standard deviation is $\sigma = \frac{1}{\sqrt{12}}$.
\end{example}

\section{The Gaussian Distribution}\label{CLTSection}

\begin{definition}\index{Distribution!Normal}\index{Distribution!Gaussian}\index{Normal Distribution}\index{Gaussian Distribution}The random variable $X$ has a \newterm{Gaussian distribution} with parameters $\mu$ (the mean) and $\sigma > 0$ (the standard deviation) if its pdf is given by the function below on the interval $(-\infty,\infty)$.
\renewcommand*{\arraystretch}{1.35}
\eqns{f_X(x) = \frac{1}{\sqrt{2\pi\sigma^2}}\, e^{-\frac{(x-\mu)^2}{2\sigma^2}}}
\renewcommand*{\arraystretch}{1}

\begin{center}
    \begin{minipage}{.5\textwidth}
        \centering
  \begin{tikzpicture}[scale = 0.6]
       \begin{axis}[title = {$Z \sim \Gaussian(0,1)$}, tick label style={font=\scriptsize, scale = 1/0.6}, title style={font=\scriptsize, scale = 1/0.6},
    xtick={-2,-1,...,2},
    ytick = {0.1,0.2,0.3,0.4, 0.5},
    ymin=0,ymax=0.5,xmin=-2.5, xmax = 2.5, width=10cm, height=6cm]
    \addplot[very thick,domain=-3:3,blue, samples=100] {0.3989*e^(-x^2/2)};
    \end{axis}
    \end{tikzpicture}
    \end{minipage}%
    \begin{minipage}{0.5\textwidth}
        \centering
  \begin{tikzpicture}[scale = 0.6]
       \begin{axis}[title = {$X \sim \Gaussian(25,3)$}, tick label style={font=\scriptsize, scale = 1/0.6}, title style={font=\scriptsize, scale = 1/0.6},
    ymin=0,ymax=0.24,xmin=17.5, xmax = 32.5, xtick = {19,22,25,28,31}, ytick = {0.1,0.2,0.3,0.4,0.5}, width=10cm, height=6cm]
       \addplot[very thick,domain=15:35,blue, samples =100] {0.19298*e^(-(x-25)^2/18)};
    \end{axis}
    \end{tikzpicture}
\end{minipage}
\end{center}
\end{definition}

The Gaussian distribution gets its name from mathematician Carl Friedrich Gauss, and is also known as the Normal distribution. It will play a central role in the estimation and testing procedures we'll develop in the next chapter. 

Unfortunately, the pdf $f_X$ for a Gaussian distribution is not integrable in elementary terms. In other words, its antiderivative can't be expressed using only the functions and operations you've encountered in Calculus II, which means we can't find areas under the curve (probabilities) using those methods.

The traditional way to get around this issue is to use a large table of values of the cdf as a reference. Unless otherwise stated, the letter $Z$ will be reserved for the random variable $Z \sim \Gaussian(0,1)$ from now on, and a $Z$-table is simply a large table of values of the cdf $F_Z(z) = P(Z \leq z)$ for many different realizations $z \in \mathbb{R}$.

What if $X$ is a Gaussian random variable with a nonzero mean and a standard deviation which is different from one? As you might expect given the two graphs shown above, it turns out the distributions of any two Gaussian random variables differ only by shifting and stretching. The tools required to show this are covered in the Probability \& Random Variables option course, 201-PRV.

\begin{proposition}If $X \sim \Gaussian (\mu,\sigma)$, then $Z = \frac{X - \mu}{\sigma} \sim \Gaussian(0,1)$.
\end{proposition}

The process of translating values of any Gaussian distributed random variable into corresponding values of $Z \sim \Gaussian (0,1)$ is known as \newterm{standardization}\index{Standardization}, and $Z \sim \Gaussian (0,1)$ is known as the \newterm{standard normal distribution}\index{Standard Normal Distribution}\index{Distribution!Standard Normal}. To standardize a realization $x$ of $X \sim \Gaussian(\mu, \sigma)$ we simply compute $z  = \frac{x - \mu}{\sigma}$, which is known as the \newterm{Z-score}\index{Z-score} of $x$, and we can treat the resulting $z$ as a realization of $Z \sim \Gaussian(0, 1)$.

\begin{example}\label{FemaleHeightsGaussian} The height of a randomly selected Canadian adult female, in inches, is approximated by the random variable $X \sim \Gaussian(64,2.5)$. What proportion of Canadian adult females are taller than 5'8"? What proportion have heights between 5'3" and 5'7"?

To standardize 5'8"\,=\,68", we calculate $z = \frac{68 - 64}{2.5} = 1.6$. Then we have
\
\begin{center}
    \begin{minipage}{0.5\textwidth}
        \centering
  $$\begin{aligned}P(X > 68) &= P(Z > 1.6) \\ &= 1 - P(Z \leq 1.6) \\ &= 1 - 0.9452 \\ &= 0.0548 \approx 5.5\%.\end{aligned}$$
  \vspace{1.25em}
    \end{minipage}\begin{minipage}{0.5\textwidth}
        \centering
   \begin{tikzpicture}[scale = 0.6]
       \begin{axis}[tick label style={font=\scriptsize, scale = 1/0.6}, ymin=0,ymax=0.5,xmin=-3, xmax = 3, xtick = {-2,-1,0,1,2}, ytick = {0.1,0.2,0.3,0.4,0.5}, width = 10cm, height = 6cm]
       \addplot[fill = blue, very thick,domain=1.6:3,blue!50, samples=100] {0.3989*e^(-x^2/2)}\closedcycle;
       \addplot[very thick,domain=-3:3,blue, samples=100] {0.3989*e^(-x^2/2)};
       \addplot[domain=-3:3] {0};
    \end{axis}
    \end{tikzpicture}
\end{minipage}
\end{center}

Thus, about 5.5\% of Canadian adult females are taller than 5'8". For the second part we standardize 5'3"\,=\,63" and 5'7"\,=\,67" to obtain $z_1 = \frac{63-64}{2.5} = -0.4$ and $z_2 = \frac{67-64}{2.5} = 1.2$, then calculate

\begin{center}
    \begin{minipage}{0.5\textwidth}
        \centering
  $$\begin{aligned}\qquad P(63 < X < 67) &= P(-0.4 < Z < 1.2) \\ &= P(Z \leq 1.2) - P(Z \leq -0.4) \\ &= 0.8849 - 0.3446 \\ &= 0.5404 \approx 54\%.\end{aligned}$$
  \vspace{1.25em}
    \end{minipage}\begin{minipage}{0.5\textwidth}
        \centering
   \begin{tikzpicture}[scale = 0.6]
       \begin{axis}[tick label style={font=\scriptsize, scale = 1/0.6}, ymin=0,ymax=0.5,xmin=-3, xmax = 3, xtick = {-2,-1,0,1,2}, ytick = {0.1,0.2,0.3,0.4,0.5}, width = 10cm, height = 6cm]
       \addplot[fill = blue, very thick,domain=-0.4:1.2,blue!50, samples=100] {0.3989*e^(-x^2/2)}\closedcycle;
       \addplot[very thick,domain=-3:3,blue, samples=100] {0.3989*e^(-x^2/2)};
       \addplot[domain=-3:3] {0};
    \end{axis}
    \end{tikzpicture}
\end{minipage}
\end{center}

Therefore, around 54\% of Canadian adult females are between 5'3" and 5'7".
\end{example}

Since there is now almost universal access to computers which can calculate areas under the pdf of $Z \sim \Gaussian(0,1)$ efficiently using numerical integration, $Z$-tables are something of a mathematical artifact whose uses are limited to classrooms and standardized tests.

\subsection*{Gaussian Distribution Shortcuts}

Any Gaussian distribution has two inflection points, and these occur precisely one standard deviation from the mean. This fact is useful when drawing a sketch of a Gaussian distribution.
\begin{center}    
    \begin{tikzpicture}[scale = 0.6]
       \begin{axis}[title = {$X \sim \Gaussian(25,3)$}, tick label style={font=\scriptsize, scale = 1/0.6}, title style={font=\scriptsize, scale = 1/0.6},
    ymin=0,ymax=0.24,xmin=17.5, xmax = 32.5, xtick = {19,22,25,28,31}, ytick = {0.1,0.2,0.3,0.4,0.5}, width=10cm, height=6cm]
       \addplot[very thick,domain=15:35,blue, samples =100] {0.19298*e^(-(x-25)^2/18)};
       \addplot [only marks] table {
28 0.11704
22 0.11704
};
    \end{axis}
    \end{tikzpicture}
 \end{center}

For approximate reasoning with Gaussian distributions without the aid of a $Z$-table or a computer, there is a well-known rule of thumb which is sometimes useful.

\begin{proposition}\index{68-95-99.7 Rule}\label{GaussianRuleThumb}(68-95-99.7 Rule) For any Gaussian distribution, 68\% of the area under its pdf occurs within one standard deviation of the mean, 95\% occurs within two standard deviations, and 99.7\% occurs within three standard deviations.
\begin{center}
  \begin{tikzpicture}[scale = 0.6]
       \begin{axis}[title = {$X \sim \Gaussian(25,3)$}, tick label style={font=\scriptsize, scale = 1/0.6}, title style={font=\scriptsize, scale = 1/0.6},
    ymin=0,ymax=0.24,xmin=15, xmax = 35, xtick = {16,19,22,25,28,31,34}, ytick = {0.1,0.2,0.3,0.4,0.5}, width=12cm, height=6cm]
       \addplot[very thick,domain=15:35,blue, samples =100] {0.19298*e^(-(x-25)^2/18)};
       \draw (axis cs:22,0.17) -- node[below]{68\%} (axis cs:28,0.17);
       \draw (axis cs:19,0.12) -- node[below]{95\%} (axis cs:31,0.12);
       \draw (axis cs:16,0.07) -- node[below]{99.7\%} (axis cs:34,0.07);
       \draw (axis cs:22,0.18) -- (axis cs:22,0.16);
       \draw (axis cs:28,0.18) -- (axis cs:28,0.16);
       \draw (axis cs:19,0.13) -- (axis cs:19,0.11);
       \draw (axis cs:31,0.13) -- (axis cs:31,0.11);
       \draw (axis cs:16,0.06) -- (axis cs:16,0.08);
       \draw (axis cs:34,0.06) -- (axis cs:34,0.08);
    \end{axis}
    \end{tikzpicture}
 \end{center}
\end{proposition}

Primarily used for quick approximate calculations, the rule also illustrates how quickly a Gaussian distribution approaches the horizontal axis when it moves away from its mean. To contrast this with the strange behaviour we saw in Example \ref{CauchyExpectation} at the end of the last section, we can say that Gaussian distributions have \newterm{thin tails}\index{Distribution! Thin-tailed}.

\begin{example}If $X \sim \Gaussian(64,2.5)$ is used to model the height of a randomly selected Canadian female, provide a range of heights which will include 95\% of all Canadian adult females.

Using the 68-95-99.7 Rule, $95\%$ of Canadian adult females will have heights within two standard deviations of the mean height, 64". So we calculate $64 - 2(2.5) = 59$ and $64 + 2(2.5) =69$, then conclude that 95\% of all Canadian adult females are taller than 59"\,=\,4'11" and shorter than 69"\,=\,5'9".
\end{example}

\section{Sampling Distributions}

Recall that a statistic is a numerical property of a sample, so the value of any statistic is determined by the sample it was derived from. Therefore, a statistic is a function whose input is a sample, and whose output is a real number.

If we fix a sample size, $n$, define a sample space $\Omega$ as the set of possible samples of that size, and assign probabilities to each possible sample, a statistic is a random variable defined on $\Omega$. This is the premise of \newterm{frequentist statistics}: a parameter is a real number, which we may or may not know, but a \emph{statistic is a random variable} and like any random variable, \emph{it has a distribution}.

\begin{definition}\index{Sampling Distribution}
The \newterm{sampling distribution} of a statistic, over samples of size $n$, is the distribution of possible values of that statistic.
\end{definition}

\begin{example}\label{DistOfMeanAge} Consider a population of three individuals, Alice, Bob, and Charles. Alice is 12 years old, Bob is 16 years old, and Charles is 22 years old. Let $\overline{X}$ denote the mean age in a sample of two individuals drawn from this population (now we're treating the sample mean as a random variable, so we write it as $\overline{X}$). Find the sampling distribution of $\overline{X}$ over all samples of size two taken with replacement.

\noindent If we let $X_1$ and $X_2$ denote the ages of the first and second individuals selected in our sample, then $\overline{X} = \frac{1}{2}(X_1 + X_2)$. There are $3\cdot3 = 9$ possible samples, which we enumerate below and for each, calculate the value of $\overline{X}$.

\begin{center}
\begin{tabular}{c|c|c|c}
Sample & $X_1$ & $X_2$ & $\overline{X}$ \\
\hline
A,A & $12$ & $12$ & $12$ \\
A,B & $12$ & $16$ & $14$ \\
A,C & $12$ & $22$ & $17$ \\
B,A & $16$ & $12$ & $14$ \\
B,B & $16$ & $16$ & $16$ \\
B,C & $16$ & $22$ & $19$ \\
C,A & $22$ & $12$ & $17$ \\
C,B & $22$ & $16$ & $19$ \\
C,C & $22$ & $22$ & $22$ \\
\end{tabular}
\end{center}

\noindent If we assume every sample is equally likely, then each of the nine samples above occurs with probability $\frac{1}{9}$, and we obtain the sampling distribution below.

\vspace{-1em}
\begin{center}
    \begin{minipage}{.45\textwidth}
        \centering
      \renewcommand*{\arraystretch}{1.35}
\begin{tabular}{c|c}
$x$ & $P(\overline{X} = x)$ \\
\hline
$12$ & $\frac{1}{9}$ \\
$14$ & $\frac{2}{9}$ \\
$16$ & $\frac{1}{9}$ \\
$17$ & $\frac{2}{9}$ \\
$19$ & $\frac{2}{9}$ \\
$22$ & $\frac{1}{9}$
\end{tabular}
\renewcommand*{\arraystretch}{1}
\vspace{0.25em}
    \end{minipage}%
    \begin{minipage}{0.5\textwidth}
        \centering
\begin{tikzpicture}[scale = 0.6]
\begin{axis} [title = {Sampling Distribution of Mean Age ($n=2$)}, width = 10cm, height = 6cm, ymin = 0, ymax = 0.3, xmin = 10.5, xmax = 23.5, tick label style={font=\scriptsize, scale = 1/0.6}, title style={font=\scriptsize, scale = 1/0.6}, ytick = {0.1,0.2,0.3}, xtick = {12,14,16,18,20,22}]
\addplot+[ycomb] plot coordinates {(12,0.1111) (14,0.2222) (16,0.1111) (17,0.2222) (19,0.2222) (22,0.1111)}; 
\end{axis}
\end{tikzpicture}
\end{minipage}
\end{center}
\begin{keypoint}
To create this distribution, we assigned equal probability to every possible sample, which means we are assuming random sampling (to be completely explicit, we're assuming the sampling process results in a simple random sample taken with replacement).
\end{keypoint}
If we compute the mean of this distribution, $E(\overline{X}) = 12 \cdot \frac{1}{9} + 14 \cdot \frac{2}{9} + \dots + 22 \cdot \frac{1}{9} = \frac{50}{3}$. Note that the mean age in our population is $\mu = \frac{12+16+22}{3} = \frac{50}{3}$ also. This is not a coincidence.
\end{example}

\begin{proposition}\label{MeanOfSampleMean}
If $\overline{X}$ is the mean of a random sample taken from a population with mean $\mu$, then $E(\overline{X}) = \mu$. In other words, the average sample mean, taken across all samples, is equal to the population mean.
\end{proposition}

\begin{remark}
The notation $\mu_{\overline{X}}$ is also used for $E(\overline{X})$, so you may see the result above written as $\mu_{\overline{X}} = \mu$.
\end{remark}

This is a key result that all the inference procedures in the next Chapter depend on, and it's a very special property of the mean that is generally not true of other statistics.

Now that we know that in any scenario where we're doing random sampling, the mean of the distribution of $\overline{X}$ is $\mu$, it's natural to wonder about the variance. Is there a simple relationship between the variance of $\overline{X}$ and the variance of the population the random sample was drawn from?

\begin{example}
Consider again the scenario in Example \ref{DistOfMeanAge}. Let's compute the variance of the random variable $\overline{X}$ and the compare it to the variance of the population.
$$\Var(\overline{X}) = (12 - \textstyle\frac{50}{3})^2 \cdot \frac{1}{9} + (14 - \textstyle\frac{50}{3})^2 \cdot \frac{2}{9} + \dots + (22 - \textstyle\frac{50}{3})^2 \cdot \frac{1}{9} = \frac{76}{9}$$
$$\sigma^2 = \textstyle\frac{1}{3}\left((12-\frac{50}{3})^2 + (16-\frac{50}{3})^2 + (22-\frac{50}{3})^2 \right) = \frac{152}{9}$$
In fact, the variance of $\overline{X}$ is precisely half the population variance. This is, again, not a coincidence.
\end{example}

\begin{proposition}\label{VarianceOfSampleMean}
If $\overline{X}$ is the mean of a random sample of $n$ individuals taken from a population with variance $\sigma^2$, then $\Var(\overline{X}) = \frac{\sigma^2}{n}$. In other words, the variance of the distribution of the sample mean is $n$ times smaller than the variance of the population.
\end{proposition}

\begin{remark}
The notation $\sigma_{\overline{X}}$ is used for the standard deviation of $\overline{X}$, so taking the square root of each side of $\Var(\overline{X}) = \frac{\sigma^2}{n}$, we obtain $\sigma_{\overline{X}} = \frac{\sigma}{\sqrt{n}}$.
\end{remark}

The proofs of Propositions \ref{MeanOfSampleMean} and \ref{VarianceOfSampleMean} require results on properties of expected value and variance that are developed in the Probability \& Random Variables option course, 201-PRV.

To sum up, we've learned some important things about the \emx{distribution of means of random samples}. In any scenario where we compute the mean of a random sample, the average sample mean is precisely equal to the population mean, and the variance of the distribution of sample means is $n$ times smaller than the variance of the population. 

%This implies that if we use the mean of random sample $\overline{X}$ to guess the value of the population mean $\mu$, the target value we're trying to hit is precisely in the center of the distribution of our guess, and the larger the sample size, the smaller the dispersion in the distribution of our guess, that is, the more the distribution of our guess clusters around its center. Bigger samples usually result in better guesses, as you would expect.

%Put Grphics here


\section{The Central Limit Theorem}

Now that we know the mean and variance of the sampling distribution of $\xbar$, what can we say about the shape? Consider, for example, the continuous random variable with density function $f_X(x) = \frac{1}{2}e^{-\frac{1}{2}x}$ on $[0,\infty)$. We can approximate the sampling distribution of $\xbar$ by taking ten thousand random samples, calculating their means, and plotting the results.

If we take samples of size one from the distribution of $X$, the distribution of sample means should look just like the distribution of $X$, since the mean of a single observation is that observation itself. We're just building a histogram for the distribution of $X$ by taking many realizations.
\begin{center}
    \begin{minipage}{.5\textwidth}
        \centering
  \begin{tikzpicture}[scale = 0.8]
       \begin{axis}[title = {$X \sim Exponential(\frac{1}{2})$},tick label style={font=\scriptsize, scale = 1/0.8}, title style={font=\scriptsize, scale = 1/0.8}, ymin=0,ymax=0.6,xmin=-0.0025, xmax = 6.2, xtick = {0,1,2,3,4,5,6}, ytick = {1}]
       \addplot[very thick,domain=0:6.5,blue] {0.5*e^(-0.5*x)};
       \draw[very thick, blue, -] (axis cs:0,0) -- (axis cs:0,2);
    \end{axis}
    \end{tikzpicture}
    \end{minipage}%
    \begin{minipage}{0.5\textwidth}
        \centering
  \begin{tikzpicture}[scale = 0.8]
\begin{axis}[title = {Distribution of $\xbar$ for $n=1$}, tick label style={font=\scriptsize, scale = 1/0.8}, title style={font=\scriptsize, scale = 1/0.8},ymin=0,ymax=1200,xmin=-0.0025, xmax = 6.2, xtick = {0,1,2,3,4,5,6}, ytick = {1500}, area style]
\addplot+[ybar interval,mark=no] plot coordinates {(0,1153) (0.25,1063) (0.5,939) (0.75,882) (1,678) (1.25,655) (1.5,557) (1.75,479) (2,443) (2.25,364) (2.5,313) (2.75,298) (3,275) (3.25,208) (3.5,211) (3.75,198) (4,161) (4.25,146) (4.5,99) (4.75,99) (5,86) (5.25,73) (5.5,86) (5.75,41) (6,65) (6.25,54)};
\end{axis}
\end{tikzpicture}
\end{minipage}
\end{center}
\begin{remark}
The continuous random variable $X$ with density function $f_X(x) = \frac{1}{2}e^{-\frac{1}{2}x}$ on $[0,\infty)$ is known as an exponential random variable with parameter $\frac{1}{2}$, which we can abbreviate $X \sim Exponential(\frac{1}{2})$. Exponential distributions are studied in detail in the Probability \& Random Variables option course, 201-PRV. For our purposes, it's simply a convenient distribution to use to illustrate the main result in this section. Note that $E(X) = \int_{0}^{\infty}\frac{1}{2}xe^{-\frac{1}{2}x}\,dx = 2$.
\end{remark}

What happens when we increase the sample size? Now we'll take ten thousand random samples, each consisting of two independent observations $X_1$ and $X_2$, and compute the mean $\xbar = \frac{1}{2}(X_1 + X_2)$ of each sample. The resulting distribution of sample means is given below on the right. Note that the most likely realizations of $X$ are very close to zero, but the most likely realizations of $\xbar$ are close to one. Each of the sample values $X_1$ and $X_2$ is more likely to be very close to zero than anywhere else, but samples where both values are simultaneously very close to zero are nonetheless quite unusual.
\begin{center}
    \begin{minipage}{.5\textwidth}
        \centering
  \begin{tikzpicture}[scale = 0.8]
       \begin{axis}[title = {$X \sim Exponential(\frac{1}{2})$}, tick label style={font=\scriptsize, scale = 1/0.8}, title style={font=\scriptsize, scale = 1/0.8},ymin=0,ymax=0.6,xmin=-0.0025, xmax = 6.2, xtick = {0,1,2,3,4,5,6}, ytick = {1}]
       \addplot[very thick,domain=0:6.5,blue] {0.5*e^(-0.5*x)};
       \draw[very thick, blue, -] (axis cs:0,0) -- (axis cs:0,2);
    \end{axis}
    \end{tikzpicture}
    \end{minipage}%
    \begin{minipage}{0.5\textwidth}
        \centering
  \begin{tikzpicture}[scale = 0.8]
\begin{axis}[title = {Distribution of $\xbar$ for $n=2$}, tick label style={font=\scriptsize, scale = 1/0.8}, title style={font=\scriptsize, scale = 1/0.8},ymin=0,ymax=1200,xmin=-0.0025, xmax = 6.2, xtick = {0,1,2,3,4,5,6}, ytick = {1500}, area style]
\addplot+[ybar interval,mark=no] plot coordinates {(0,282) (0.25,620) (0.5,801) (0.75,894) (1,959) (1.25,880) (1.5,803) (1.75,706) (2,608) (2.25,537) (2.5,469) (2.75,394) (3,366) (3.25,283) (3.5,260) (3.75,231) (4,158) (4.25,142) (4.5,103) (4.75,105) (5,57) (5.25,67) (5.5,57) (5.75,37) (6,27) (6.25,36)};
\end{axis}
\end{tikzpicture}
\end{minipage}
\end{center}
If we continue increasing the sample size, taking ten thousand random samples of larger and larger sizes from the same distribution, how will the distribution of the sample mean $\xbar$ change?
\begin{center}
    \begin{minipage}{.5\textwidth}
        \centering
  \begin{tikzpicture}[scale = 0.8]
       \begin{axis}[title = {$X \sim Exponential(\frac{1}{2})$},tick label style={font=\scriptsize, scale = 1/0.8}, title style={font=\scriptsize, scale = 1/0.8}, ymin=0,ymax=0.6,xmin=-0.0025, xmax = 6.2, xtick = {0,1,2,3,4,5,6}, ytick = {1}]
       \addplot[very thick,domain=0:6.5,blue] {0.5*e^(-0.5*x)};
       \draw[very thick, blue, -] (axis cs:0,0) -- (axis cs:0,2);
    \end{axis}
    \end{tikzpicture}
    \end{minipage}%
    \begin{minipage}{0.5\textwidth}
        \centering
  \begin{tikzpicture}[scale = 0.8]
\begin{axis}[title = {Distribution of $\xbar$ for $n=5$},tick label style={font=\scriptsize, scale = 1/0.8}, title style={font=\scriptsize, scale = 1/0.8}, ymin=0,ymax=1300,xmin=-0.0025, xmax = 6.2, xtick = {0,1,2,3,4,5,6}, ytick = {1500}, area style]
\addplot+[ybar interval,mark=no] plot coordinates {(0,4) (0.25,85) (0.5,355) (0.75,668) (1,954) (1.25,1180) (1.5,1224) (1.75,1140) (2,1045) (2.25,861) (2.5,678) (2.75,495) (3,354) (3.25,318) (3.5,218) (3.75,139) (4,89) (4.25,67) (4.5,49) (4.75,32) (5,17) (5.25,11) (5.5,4) (5.75,6) (6,2) (6.25,2)};
\end{axis}
\end{tikzpicture}
\end{minipage}
\end{center}
\begin{center}
    \begin{minipage}{.5\textwidth}
        \centering
  \begin{tikzpicture}[scale = 0.8]
       \begin{axis}[title = {$X \sim Exponential(\frac{1}{2})$},tick label style={font=\scriptsize, scale = 1/0.8}, title style={font=\scriptsize, scale = 1/0.8}, ymin=0,ymax=0.6,xmin=-0.0025, xmax = 6.2, xtick = {0,1,2,3,4,5,6}, ytick = {1}]
       \addplot[very thick,domain=0:6.5,blue] {0.5*e^(-0.5*x)};
       \draw[very thick, blue, -] (axis cs:0,0) -- (axis cs:0,2);
    \end{axis}
    \end{tikzpicture}
    \end{minipage}%
    \begin{minipage}{0.5\textwidth}
        \centering
  \begin{tikzpicture}[scale = 0.8]
\begin{axis}[title = {Distribution of $\xbar$ for $n=10$},tick label style={font=\scriptsize, scale = 1/0.8}, title style={font=\scriptsize, scale = 1/0.8}, ymin=0,ymax=1750,xmin=-0.0025, xmax = 6.2, xtick = {0,1,2,3,4,5,6}, ytick = {2000}, area style]
\addplot+[ybar interval,mark=no] plot coordinates {(0,0) (0.25,1) (0.5,55) (0.75,269) (1,709) (1.25,1275) (1.5,1631) (1.75,1624) (2,1434) (2.25,1073) (2.5,782) (2.75,493) (3,288) (3.25,175) (3.5,102) (3.75,42) (4,22) (4.25,18) (4.5,5) (4.75,0) (5,2) (5.25,0) (5.5,0) (5.75,0) (6,0) (6.25,0)};
\end{axis}
\end{tikzpicture}
\end{minipage}
\end{center}
\begin{center}
    \begin{minipage}{.5\textwidth}
        \centering
  \begin{tikzpicture}[scale = 0.8]
       \begin{axis}[title = {$X \sim Exponential(\frac{1}{2})$},tick label style={font=\scriptsize, scale = 1/0.8}, title style={font=\scriptsize, scale = 1/0.8}, ymin=0,ymax=0.6,xmin=-0.0025, xmax = 6.2, xtick = {0,1,2,3,4,5,6}, ytick = {1}]
       \addplot[very thick,domain=0:6.5,blue] {0.5*e^(-0.5*x)};
       \draw[very thick, blue, -] (axis cs:0,0) -- (axis cs:0,2);
    \end{axis}
    \end{tikzpicture}
    \end{minipage}%
    \begin{minipage}{0.5\textwidth}
        \centering
  \begin{tikzpicture}[scale = 0.8]
\begin{axis}[title = {Distribution of $\xbar$ for $n=30$},tick label style={font=\scriptsize, scale = 1/0.8}, title style={font=\scriptsize, scale = 1/0.8}, ymin=0,ymax=2900,xmin=-0.0025, xmax = 6.2, xtick = {0,1,2,3,4,5,6}, ytick = {3000}, area style]
\addplot+[ybar interval,mark=no] plot coordinates {(0,0) (0.25,0) (0.5,0) (0.75,6) (1,108) (1.25,653) (1.5,1826) (1.75,2653) (2,2385) (2.25,1430) (2.5,636) (2.75,230) (3,59) (3.25,12) (3.5,2) (3.75,0) (4,0) (4.25,0) (4.5,0) (4.75,0) (5,0) (5.25,0) (5.5,0) (5.75,0) (6,0) (6.25,0)};
\end{axis}
\end{tikzpicture}
\end{minipage}
\end{center}

In the last section, we saw that if $\xbar$ is the mean of a random sample, then $E(\xbar) = \mu$ and $\Var(\xbar) = \frac{\sigma^2}{n}$. These results are visible in the graphics above. Each distribution of $\xbar$ is centered at 2, which is the mean of the exponential distribution the samples were drawn from, and the dispersion of the distribution of $\xbar$ decreases as the sample size grows (note that $\frac{\sigma^2}{n}$ is a decreasing function of $n$).

Moreover, there is a clear trend in the shapes of the sampling distributions. As the sample size increases, the distribution looks less like an exponential distribution, and more like a Gaussian distribution. This, once again, is not a coincidence.

This phenomenon is why the Gaussian distribution plays such a central role in statistics. If some quantity is obtained by averaging a random sample, and the sample size is large enough, then we know that the distribution of that quantity must be approximately Gaussian \emph{regardless of the shape of the distribution the sample values were drawn from}.

\begin{theorem} (\newterm{Central Limit Theorem})\index{Central Limit Theorem}\label{CLT} Consider a variable with mean $\mu$ and standard deviation $\sigma$ in some population, and let $\xbar$ denote the mean of a random sample of $n$ values drawn from that population. Then as the sample size $n$ increases, the sampling distribution of $\xbar$ becomes better approximated by $\Gaussian(\muxbar,\sigmaxbar)$, where $\mu_{\xbar} = \mu$ and $\sigma_{\xbar} = \frac{\sigma}{\sqrt{n}}$.
\end{theorem}

The conclusion of the theorem is written here in very informal language. To make it more precise, we would need to decide on a way to measure the difference between two distributions, so the theorem above could state that this difference vanishes as $n \to \infty$. For details, see \cite{DevoreBerk} or \cite{Ghahramani}.

We will write $Y \sim AG(\mu,\sigma)$ to indicate a random variable $Y$ has a distribution which is approximately Gaussian with mean $\mu$ and standard deviation $\sigma$. This way, the conclusion of the central limit theorem can be written simply as $\xbar \sim AG(\mu,\frac{\sigma}{\sqrt{n}})$, where $\mu$ and $\sigma$ are the mean and standard deviation of the population our sample was drawn from.

\begin{example}\label{CLTExamp}Twenty numbers in the interval $[0,1]$ are independently chosen uniformly at random. What is the probability their sum is higher than eight?

In order to use the central limit theorem, we need to formulate this question in terms of the mean of a sample. In this case, we have a sample of $n=20$ values drawn from a uniform distribution on the interval $[0,1]$. Observe that
$$X_1 + X_2 + \cdots + X_{20} > 8 \ \ \rightarrow \ \  \frac{X_1 + X_2 + \cdots + X_{20} }{20} > \frac{8}{20}.$$

Thus, to answer this question, we need to find the probability of obtaining a sample of twenty values with a mean $\xbar$ higher than $\frac{2}{5}$. We know from Example \ref{MeanVarianceUnitInterval} that the mean and standard deviation of the distribution our sample is drawn from are $\mu = \frac{1}{2}$ and $\sigma = \frac{1}{\sqrt{12}}$.

The central limit theorem states that the distribution of $\xbar$ is approximately Gaussian, with a mean of $\mu = \frac{1}{2}$ and a standard deviation of $\frac{\sigma}{\sqrt{n}} = \frac{1}{\sqrt{12}\sqrt{20}} = \frac{1}{\sqrt{240}}$, or more concisely, $\xbar \sim AG(\frac{1}{2}, \frac{1}{\sqrt{240}})$.

Now the probability of obtaining a sample with $\xbar$ higher than $\frac{2}{5}$ can be calculated (approximately) by standardizing and using a $Z$-table.
$$z = \frac{\xbar - \muxbar}{\sigmaxbar} = \frac{\frac{2}{5} - \frac{1}{2}}{\frac{1}{\sqrt{240}}} \approx -1.55$$
\begin{center}
    \begin{minipage}{.5\textwidth}
        \centering
  \eqns{P(\xbar > \textstyle\frac{2}{5}) &= P(Z > -1.55) \\ &= 1 - P(Z \leq -1.55) \\ &= 1 - 0.0603 \\ &= 0.9397 \approx 94\%}
  \vspace{1.25em}
    \end{minipage}%
    \begin{minipage}{0.5\textwidth}
        \centering
   \begin{tikzpicture}[scale = 0.6]
       \begin{axis}[width = 10cm, height = 6cm, tick label style={font=\scriptsize, scale = 1/0.6}, ymin=0,ymax=0.5,xmin=-2.9, xmax = 2.9, xtick = {-2,-1,0,1,2}, ytick = {0.1,0.2,0.3,0.4,0.5}]
       \addplot[fill = blue, very thick,domain=-1.55:3,blue!50, samples=100] {0.3989*e^(-x^2/2)}\closedcycle;
       \addplot[very thick,domain=-3:3,blue, samples=100] {0.3989*e^(-x^2/2)};
       \addplot[domain=-3:3] {0};
    \end{axis}
    \end{tikzpicture}
\end{minipage}
\end{center}

Note that the central limit theorem does not tell us how accurate the Gaussian approximation of the distribution of $\xbar$ is, just that it will be more accurate for larger sample sizes. Let's take ten thousand samples of twenty numbers in the interval $[0,1]$ and plot the resulting sampling distribution of $\xbar$, along with the approximating normal distribution given by the central limit theorem.

If the two distributions match well, then we'll know that our answer above is fairly accurate. If the sampling distribution is still far from being Gaussian, then we'll have to disregard our answer.
\begin{center}
\begin{tikzpicture}[scale = 0.8]
\begin{axis}[title = {Distribution of $\xbar$ and $\Gaussian(\frac{1}{2},\frac{1}{\sqrt{240}})$}, tick label style={font=\scriptsize, scale = 1/0.8}, title style={font=\scriptsize, scale = 1/0.8}, ymin=0,ymax=1700,xmin=0.2, xmax = 0.8, xtick = {0.2,0.3,0.4,0.5,0.6,0.7,0.8}, ytick = {3300}, area style, width = 10cm, height = 6cm]
\addplot+[ybar interval,mark=no] plot coordinates {(0,0) (0.025,0) (0.05,0) (0.075,0) (0.1,0) (0.125,0) (0.15,0) (0.175,0) (0.2,0) (0.225,0) (0.25,2) (0.275,6) (0.3,13) (0.325,61) (0.35,147) (0.375,328) (0.4,624) (0.425,942) (0.45,1257) (0.475,1520) (0.5,1499) (0.525,1355) (0.55,955) (0.575,668) (0.6,346) (0.625,167) (0.65,58) (0.675,34) (0.7,7) (0.725,0)};
\addplot[very thick,domain=0:1,red, samples=500] {250*6.185*e^((-(x - 0.5)^2)/(2*(1/240)))};
\end{axis}
\end{tikzpicture}
\end{center}
The distribution of $\xbar$ for random samples of size twenty is extremely well approximated by a Gaussian distribution, so we can be sure our answer is reasonably accurate.
\end{example}

\subsection*{Sample Size}

The central limit theorem states that the sampling distribution of $\xbar$ will eventually look Gaussian, once the sample size is large enough. The question is then how large must our sample be, so that using a Gaussian distribution for probability calculations, as in the example above, gives an accurate result?

Unfortunately, the answer depends on the details of the population distribution. Symmetric population distributions with tails that decay exponentially (thin tails) typically yield sampling distributions for $\xbar$ which are very close to Gaussian even for single digit sample sizes, while asymmetric population distributions with tails that decay polynomially (fat tails) can result in sampling distributions for $\xbar$ which don't look Gaussian until the sample size becomes extremely large.

As a rule of thumb, we'll require a sample size of thirty or more to treat the distribution of $\xbar$ as if it's Gaussian. There is nothing mathematically significant about this value, or slightly different values you might find in other texts. We're simply establishing a clear, yet arbitrary, baseline. 

\begin{remark}
If the population distribution is known to be Gaussian, then any sample size will suffice. In fact, an important property of Gaussian distributions is that the average of any number of observations from a Gaussian distribution is again Gaussian, no central limit theorem required.
\end{remark}

In practice, to convince themselves the sampling distribution of $\xbar$ is sufficiently close to a Gaussian distribution for a given sample size $n$, statisticians could either rigorously derive error bounds from some known properties of the population distribution, usually in combination with stronger variations of the central limit theorem, or use statistical software to take many samples from an educated guess at the population distribution, and approximate the sampling distribution empirically, as was done to generate the graphics at the beginning of this section.

\begin{example}
If a biased coin with a 55\% probability of heads is flipped one hundred times, what is the probability that the number of heads observed is less than the number of tails?

If we call a head a success, then we have a sample of $n=100$ values $X_1,X_2\,\dots\,,X_{100}$ from $X \sim \Bernoulli(0.55)$. If more tails than heads were observed, this means that more than half of the $X_i$ are zeros, so $\xbar = \frac{X_1 + X_2 + \cdots + X_{100}}{100} < \frac{50}{100} =0.5$.

Using the formulas for the mean and variance of a Bernoulli random variable in Proposition \ref{BernoulliExpectation}, the mean and standard deviation of the sampling distribution of $\xbar$ are given by 
$$\muxbar = \mu = p = 0.55 \ \ \text{and} \ \  \sigmaxbar = \frac{\sigma}{\sqrt{n}} = \frac{\sqrt{p(1-p)}}{\sqrt{n}} = \textstyle\frac{\sqrt{0.55\cdot 0.45}}{\sqrt{100}} \approx 0.05.$$

By the central limit theorem, $\xbar \sim AG(0.55,0.05)$, so we can calculate (approximately) the probability of obtaining a sample whose mean is smaller than $0.5$ by standardizing and using a $Z$-table.
$$z = \frac{\xbar - \muxbar}{\sigmaxbar} = \frac{0.5 - 0.55}{0.05} = \frac{-0.05}{0.05} = -1$$

\begin{center}
    \begin{minipage}{.5\textwidth}
        \centering
  \eqns{P(\xbar < 0.5) &= P(Z < -1) \\ &= 0.1587 \approx 16\%}
  \vspace{1.25em}
    \end{minipage}%
    \begin{minipage}{0.5\textwidth}
        \centering
   \begin{tikzpicture}[scale = 0.6]
       \begin{axis}[width = 10cm, height = 6cm, tick label style={font=\scriptsize, scale = 1/0.6}, ymin=0,ymax=0.5,xmin=-2.9, xmax = 2.9, xtick = {-2,-1,0,1,2}, ytick = {0.1,0.2,0.3,0.4,0.5}]
       \addplot[fill = blue, very thick,domain=-3:-1,blue!50, samples=100] {0.3989*e^(-x^2/2)}\closedcycle;
       \addplot[very thick,domain=-3:3,blue, samples=100] {0.3989*e^(-x^2/2)};
       \addplot[domain=-3:3] {0};
    \end{axis}
    \end{tikzpicture}
\end{minipage}
\end{center}
Therefore, there's about a $16\%$ chance we will observe more tails than heads. \\
\end{example}

\begin{warning}
It's very important to know the hypotheses of the central limit theorem. The procedures introduced in the next chapter are justified so long as we can treat the mean of our sample as a realization of a Gaussian random variable, which we can as long as these statements hold.
\begin{itemize}
\item The sample data we're working with was obtained via random sampling.
\item The sample data came from a distribution with well-defined mean and variance.
\item The sample size is large enough, or our sample data came from a Gaussian distribution.
\end{itemize}
\end{warning}


%Error of Average of guesses vs Average of Guess errors illustrates CLT





% Modified 31 Oct 2005:  Conditioning fallacy alluded to.
% This chapter has been modified on 6-4-05.
% There are two \choice
\pagestyle{headings}
\chapter{Confidence Intervals \& Hypothesis Tests} \label{chp 5}

%TODO: Examples and discussion of non-CLT confidence intervals.

\section{Confidence Interval for a Population Mean}\index{Confidence Interval! for a Mean}


\subsection*{Estimating Average Age}

In this chapter, we'll put the central limit theorem to work. Suppose we're interested in knowing the mean age of students at Champlain College. To estimate it, we take a random sample of 50 students and record their ages. It turns out that the mean age of students in the sample is $18.3$ years.

Since the sample mean, $\littlexbar = 18.3$, is the mean of a random sample of a sufficiently large size, the central limit theorem tells us $\xbar \sim AG(\muxbar, \sigmaxbar)$. Standardizing the sample mean we observed yields
$$z = \frac{\overline{x} - \muxbar}{\sigmaxbar} = \frac{18.3 - \muxbar}{\sigmaxbar}.$$
Now suppose we happen to know that the standard deviation for ages of students at Champlain College is $2.7$ years. Then since $\muxbar = \mu$ and $\sigmaxbar = \sigma / \sqrt{n} = 2.7 / \sqrt{50} \simeq 0.381$, we can write
$$z = \frac{\overline{x}  - \muxbar}{\sigmaxbar} = \frac{18.3 - \muxbar}{\sigmaxbar} = \frac{18.3 - \mu}{0.381}.$$
We've seen in Proposition \ref{GaussianRuleThumb} that approximately $95\%$ of all values of $Z$ fall into the interval $(-2, 2)$, and hence we should be about 95\% confident that
\begin{gather*}
-2 < \textstyle\frac{18.3 - \mu}{0.381} < 2 \\
-0.762 < 18.3 - \mu < 0.762 \\
-19.062 < - \mu < -17.538 \\
17.538 < \mu < 19.062.
\end{gather*}

This range of values is called a \index{Confidence Interval}\newterm{confidence interval} for the parameter $\mu$, which in this case represents the mean age of all students at Champlain College. We can't determine the exact value of $\mu$ without collecting information from every single student. However, from one random sample of fifty students, we can assert that $17.5 < \mu < 19.1$ with 95\% confidence.

Confidence intervals are so useful and pervasive in applications of statistics, it's worth our while to set up some notation and terminology to make constructing them as routine as possible.

\begin{definition}\index{Critical Value}\index{Confidence Level}Given a confidence level $C \in (0,1)$, the \newterm{critical value} corresponding to that confidence level is the value $z^*$ with $P(-z^* < Z < z^*) = C$.
\end{definition}
\begin{center}
\begin{tikzpicture}[scale = 0.8]
  \begin{axis}[
  	  scale only axis,
      tick label style={font=\scriptsize, scale = 1/0.8},
      width = 8cm,
      height = 4cm,
      ymin=-0.0025,
      ymax=0.5,
      xmin=-3,
      xmax=3,
      xtick = {-1.5,0,1.5}, xticklabels = {$-z^*$,$0$,$z^*$},
      ytick=\empty,
      legend pos=north east,
      domain=-3:3,
      samples=200,
      thick
    ]
    \addplot[blue,opacity=0.6] {0.3989*pow(e,-x^2/2)};
    \addplot[blue, opacity=0, fill=blue, fill opacity=0.3, domain=-1.5:1.5] 
      {0.3989*pow(e,-x^2/2)}\closedcycle;
    \node at (axis cs: 0,0.2) {\large{$C$}};
  \end{axis}
\end{tikzpicture}
\end{center}

\begin{example}
The critical value for a confidence level of $90\%$ is $z^* = 1.65$, since 
$$P(-1.65 < Z < 1.65) = P(Z < 1.65) - P(Z < -1.65) = 0.9505 - 0.495 = 0.9.$$
\begin{center}
\begin{tikzpicture}[scale = 0.8]
  \begin{axis}[
  	  scale only axis,
      tick label style={font=\scriptsize, scale = 1/0.8},
      width = 8cm,
      height = 4cm,
      ymin=-0.0025,
      ymax=0.5,
      xmin=-3,
      xmax=3,
      xtick={-3,-2,-1,0,1,2,3},
      ytick=\empty,
      legend pos=north east,
      domain=-3:3,
      samples=200,
      thick
    ]   
    \addplot[blue,opacity=0.6] {0.3989*pow(e,-x^2/2)}; 
    \addplot[blue, opacity=0, fill=blue, fill opacity=0.3, domain=-1.65:1.65] 
      {0.3989*pow(e,-x^2/2)}\closedcycle;   
    \node at (axis cs: 0,0.2) {\large{$90\%$}};
  \end{axis}
\end{tikzpicture}
\end{center}
Using the $Z$-table, we can find $z^*$ by noting that if $P(-z^* < Z < z^*) = 0.9$, then by symmetry, the area in both tails is 0.05, so $P(Z < z^*) = 0.95$. Thus, we're looking for the value of $z$ that has an area of 0.95 to its left.
\end{example}

If we repeat the calculation we made in the introduction to this section, but this time with a general critical value $z^*$ and sample mean $\xbar$, we'll arrive at a general formula for a confidence interval for the mean $\mu$ of any population from which we might draw a sample. With a probability of $C$, we have
\begin{gather*}
-z^* < Z < z^* \\
-z^* < \textstyle\frac{\overline{X}- \muxbar}{\sigmaxbar} < z^* \\
-z^* \sigmaxbar < \overline{X}- \mu < z^* \sigmaxbar \\[0.2ex]
-\overline{X}-z^* \sigmaxbar < - \mu < -\overline{X}+z^* \sigmaxbar \\[0.3ex]
\overline{X}-z^* \sigmaxbar <  \mu < \overline{X}+z^* \sigmaxbar \\[0.3ex]
\overline{X}-z^* \textstyle\frac{\sigma}{\sqrt{n}} <  \mu < \overline{X}+z^* 
\frac{\sigma}{\sqrt{n}}.
\end{gather*}

This result here only holds if $\xbar$ is Gaussian, but the central limit theorem assures us that as long as our data comes from a random sample of sufficiently large size, we are justified in treating it as such. The quantity we add to and subtract from $\xbar$ in the last line is called the margin of error.

\begin{definition}\index{Margin of Error! for a Mean} Given a random sample of $n$ values drawn from a distribution with standard deviation $\sigma$, and a confidence level $C \in (0,1)$, the \newterm{margin of error} for $\overline{X}$ at confidence level $C$ is 
$$E = z^* \frac{\sigma}{\sqrt{n}}$$
\end{definition}

This value represents the largest possible gap between a sample mean $\overline{X}$ and the population mean $\mu$ that is observed at the confidence level $C$. For example, if $E = 2$ when $C = 0.9$, this means that 90\% of random samples have means which differ from the population mean by less than $2$.

With these definitions established, we can describe the procedure for constructing a confidence interval for a population mean $\mu$ based on a random sample of $n$ values with mean $\littlexbar$ in three steps.
\begin{enumerate}[label=\textnormal{\Roman*}.]
\item Find the critical value $z^*$ for the desired confidence level $C$.
\item Compute the margin of error $E$.
\item The confidence interval is given by $\littlexbar - E < \mu < \littlexbar + E$.
\end{enumerate}

\begin{example} 
The standard deviation for heights of all men in Canada is known to be around 2.9 inches. In a random sample of 37 Canadian men, the average height was 5'9". Use this information to construct a $95\%$ confidence interval for the average height of all Canadian men.
\begin{enumerate}[label=\textnormal{\Roman*}.]
\item The critical value $z^*$ is defined by $P(-z^* < Z < z^*) = 0.95$, so the area on the left of $z^*$ must be $P(Z < z^*) = 0.975$, and consulting the Z-table yields $z^* = 1.96$.
\begin{center}
\begin{tikzpicture}[scale = 0.8]
  \begin{axis}[
  	  scale only axis,
      tick label style={font=\scriptsize, scale = 1/0.8},
      width = 8cm,
      height = 4cm,
      ymin=-0.0025,
      ymax=0.5,
      xmin=-3,
      xmax=3,
      xtick={-3,-2,-1,0,1,2,3},
      ytick=\empty,
      legend pos=north east,
      domain=-3:3,
      samples=200,
      thick
    ]
    \addplot[blue,opacity=0.6] {0.3989*pow(e,-x^2/2)};
    \addplot[blue, opacity=0, fill=blue, fill opacity=0.3, domain=-1.96:1.96] 
      {0.3989*pow(e,-x^2/2)}\closedcycle;
    \node at (axis cs: 0,0.2) {\large{$95\%$}}; 
  \end{axis}
\end{tikzpicture}
\end{center}
\item The margin of error is then given by $E = z^*  \frac{\sigma}{\sqrt{n}} = 1.96  \frac{2.9}{\sqrt{37}} \simeq 0.934$. 
\item In inches, the sample mean 5'9" is 69". Adding and subtracting the margin of error gives
$$\begin{aligned}69-0.934 < \ &\mu < 69+0.934 \\
68.066 < \ &\mu < 69.934.\end{aligned}$$
Thus, with $95\%$ confidence, the mean height of all Canadian men must be between 68.066" (about 5'8") and and 69.934" (about 5'10").
\end{enumerate}
\end{example}

Note that when we say `\,with 95\% confidence, $68.066 < \mu < 69.934$\,', this means there is a 95\% chance that we have selected a sample which will produce a confidence interval containing the true mean height of the population, $\mu$.

The mean height of all Canadian men is some fixed single value, we just don't know what it is. Thus, the probability it's inside our particular confidence interval is either $100\%$ or $0\%$ (it is or it isn't, there's no randomness involved). But if we were to take many random samples and compute a 95\% confidence interval from each, we would expect that, typically, nineteen of every twenty samples would produce intervals that contain the true value of $\mu$ (since 19/20 is 95\%). This is the correct interpretation of the confidence level.
\begin{center}
\begin{tikzpicture}[scale=0.75]
\draw[<->] (-3.5,0) -- (3.5,0);
\draw (0,0.1) -- (0,-0.1);
\node at (3.5,-0.4) {};
\node at (0,-0.4) {$\mu$};
\draw[blue] (-1,0.2) -- (2,0.2);
\draw[blue] (-0.5,0.4) -- (2.5,0.4);
\draw[blue] (-1.9,0.6) -- (1.1,0.6);
\draw[blue] (-1.2,0.8) -- (1.8,0.8);
\draw[red] (0.2,1.0) -- (3.2,1.0);
\draw[blue] (-1.6,1.2) -- (1.4,1.2);
\draw[blue] (-0.7,1.4) -- (2.3,1.4);
\draw[blue] (-2.1,1.6) -- (0.9,1.6);
\draw[blue] (-2.7,1.8) -- (0.3,1.8);
\draw[blue] (-1,2.0) -- (2,2.0);
\draw[blue] (-1.3,2.2) -- (1.7,2.2);
\draw[blue] (-2.6,2.4) -- (0.4,2.4);
\draw[blue] (-0.9,2.6) -- (2.1,2.6);
\draw[blue] (-1.7,2.8) -- (1.3,2.8);
\draw[blue] (-0.5,3.0) -- (2.5,3.0);
\draw[blue] (-2.2,3.2) -- (0.8,3.2);
\draw[blue] (-2.4,3.4) -- (0.6,3.4);
\draw[blue] (-1.2,3.6) -- (1.8,3.6);
\draw[blue] (-1.6,3.8) -- (1.4,3.8);
\draw[blue] (-0.7,4.0) -- (2.3,4.0);
\draw[blue] (-0.8,4.2) -- (2.2,4.2);
\draw[dashed] (0,0.1) -- (0,4.5);
\end{tikzpicture}
\end{center}

Notice that as the confidence level becomes larger, the corresponding critical value $z^*$ grows to enclose that larger area, which increases the margin of error, $E$. On the other hand, the value of $E$ is a decreasing function of the sample size, $n$, so a larger sample will lead to a smaller confidence interval, giving a more accurate estimate of the parameter $\mu$.
\begin{prop} To reduce the margin of error $E$ and produce tighter bounds on $\mu$, we can either decrease the confidence level $C$, or increase the sample size $n$.
\end{prop}

Our procedure for constructing a confidence interval for a population mean $\mu$ requires knowledge of the population standard deviation $\sigma$. This is to be expected, as means of samples from a population with a large standard deviation will be more variable than means of samples from a population with a low standard deviation. Unfortunately, this requirement is problematic since $\sigma$ is a parameter of the population distribution, so is almost always unknown in practice.

\subsection*{What if we don't know $\sigma$?}

Can we somehow estimate $\sigma$ and build this estimation process into the confidence interval procedure? The best estimate of the population distribution we have access to is the distribution of values within our sample, so what if we simply use the standard deviation of our sample in place of the standard deviation of the population?

If we use the sample standard deviation, $S$, to estimate the standard deviation in our population, $\sigma$, then when we standardize a sample mean $\xbar$, instead of the standardized value
$$\frac{\overline{X} - \muxbar}{\sigmaxbar} = \frac{\overline{X} - \mu}{\frac{\sigma}{\sqrt{n}}}, \ \ \text{we'll have the estimate} \ \ \frac{\overline{X} - \muxbar}{\sigmaxbar} \approx \frac{\overline{X} - \mu}{\frac{S}{\sqrt{n}}}.$$
Our construction of a confidence interval for the population mean $\mu$ was justified by the central limit theorem, which tells us the distribution of the quantity on the left approaches $Z \sim \Gaussian(0,1)$ as the sample size $n$ grows. But what about our estimate on the right? If we repeatedly draw samples and calculate this value, what kind of distribution will we obtain?

\begin{definition}\label{TDistDef}\index{Distribution! Student's T}\index{T-distribution} Draw a random sample $X_1, X_2, \, \dots \, , X_n$ from a Gaussian distribution with mean $\mu$. Let
$$T = \frac{\overline{X} - \mu}{\frac{S}{\sqrt{n}}}$$
where $S$ is the sample standard deviation, as in Definition \ref{SampleStdDev}. Then $T$ is a continuous random variable whose distribution is called \newterm{Student's $\boldsymbol{T}$-distribution} (or simply the $T$-distribution) with $n-1$ degrees of freedom.
\end{definition}
\begin{center}
\begin{tikzpicture}[scale = 1]
  \begin{axis}[
  	  scale only axis,
      tick label style={font=\scriptsize, scale = 1},
      width = 8cm,
      height = 4cm,
      ymin=-0.0025,
      ymax=0.5,
      xmin=-3,
      xmax=3,
      xtick={-3,-2,-1,0,1,2,3},
      ytick=\empty,
      legend pos=north east,
      domain=-3:3,
      samples=200,
      thick
    ]
    
    \addplot[blue,opacity=1] {0.318*pow(1+x*x,-1)};
    \addplot[blue,opacity=0.8] {0.353*pow(1+(0.5)*x*x,-1.5)};
    \addplot[blue,opacity=0.6] {0.375*pow(1+(0.25)*x*x,-(0.5*(4+1)))};
    \addplot[blue,opacity=0.4] {0.389*pow(1+(0.1)*x*x,-(0.5*(10+1)))};
    \addplot[blue,opacity=0.2] {0.397*pow(1+(0.01)*x*x,-(0.5*(100+1)))};
    \addplot[black,dashed] {0.3989*pow(e,-x^2/2)};
    
  \end{axis}
\end{tikzpicture}
\end{center}

Above in blue are plots of probability density functions for $T$-distributions with one, two, four, ten, and one hundred degrees of freedom, shaded lighter as the degree of freedom grows. The standard normal distribution is shown in dashed black.

The $T$-distribution has fatter tails and less area near the center than the standard normal distribution. This additional variability appears because the sample standard deviation $S$ differs from one sample to the next, which adds a source of variability to $T$-scores which is not present in $Z$-scores. As the sample size grows, the sample standard deviation $S$ becomes a better estimator of the population standard deviation $\sigma$, and the $T$-distribution slowly approaches the standard normal distribution. Notice that once the degree of freedom reaches one hundred, the $T$-distribution and the standard normal distribution are effectively indistinguishable.

If we can find areas under the $T$-distribution, then the same procedure we've been using to construct confidence intervals with the population standard deviation can be modified to work with the sample standard deviation.

\begin{proposition} Let $\xbar$ and $S$ be the sample mean and (Bessel-corrected) sample standard deviation in a random sample of $n$ values drawn from a Gaussian distribution with mean $\mu$. 

Take $t^*$ such that $P(-t^* < T < t^*) = C$, and let $E = t^*\frac{S}{\sqrt{n}}$. Then $P(\xbar - E < \mu < \xbar + E) = C$.
\end{proposition}
\begin{proofnobox}
By definition of $t^*$, with probability $C$ we have
\begin{gather*}
-t^* < T < t^* \\
-t^* < \textstyle\frac{\overline{X}- \mu}{\frac{S}{\sqrt{n}}} < t^* \\
-t^* \textstyle\frac{S}{\sqrt{n}} < \overline{X}- \mu < t^* \frac{S}{\sqrt{n}} \\[0.2ex]
-\overline{X}-t^* \textstyle\frac{S}{\sqrt{n}} < - \mu < -\overline{X}+t^* \frac{S}{\sqrt{n}} \\[0.3ex]
\overline{X}-t^* \textstyle\frac{S}{\sqrt{n}} <  \mu < \overline{X}+t^* \frac{S}{\sqrt{n}}.
\end{gather*}
\end{proofnobox}

\begin{example}\label{TIntExample} A sample of five values is drawn from a Gaussian distribution with unknown mean and standard deviation. These values are $43$, $27$, $30$, $22$, and $38$. Construct a 90\% confidence interval for the mean of the Gaussian distribution.

Computing the mean and (Bessel corrected) standard deviation of the sample yields $\littlexbar = 32$ and $s = 8.456$. We can now construct a confidence interval for the population mean $\mu$ just as we've done before in the last section, but using $s$ in place of $\sigma$ and the $T$-distribution in place of the standard normal distribution.
%\begin{center}
%   \begin{tikzpicture}[scale = 0.6]
%       \begin{axis}[unit vector ratio=1 6 1, ymin=0,ymax=0.5,xmin=-2.9, xmax = 2.9, xtick = {-2.13,0,2.13}, xticklabels = {$-2.13$,$0$,$2.13$}, ytick = {0.1,0.2,0.3,0.4,0.5}]
%       \addplot[fill = blue, very thick,domain=-2.13:2.13,blue!50, samples=100]  gnuplot {(gamma(0.5*(4+1))/(sqrt(4*pi)*gamma((0.5*4))))*(1+(0.25)*x*x)**(-(0.5*(4+1)))}\closedcycle;
%       \addplot[very thick,domain=-3:3,blue, samples=100]  gnuplot {(gamma(0.5*(4+1))/(sqrt(4*pi)*gamma((0.5*4))))*(1+(0.25)*x*x)**(-(0.5*(4+1)))};
%       \addplot[domain=-3:3] {0};
%       \node at (axis cs: 0,0.2) {\Large{$0.9$}};
%    \end{axis}
%    \end{tikzpicture}
%\end{center}
\begin{center}
\begin{tikzpicture}[scale = 0.8]
  \begin{axis}[
  	  scale only axis,
      tick label style={font=\scriptsize, scale = 1/0.8},
      width = 8cm,
      height = 4cm,
      ymin=-0.0025,
      ymax=0.5,
      xmin=-3,
      xmax=3,
      xtick={-3,-2,-1,0,1,2,3},
      ytick=\empty,
      legend pos=north east,
      domain=-3:3,
      samples=200,
      thick
    ]
    
    \addplot[blue,opacity=0.6] {0.375*pow(1+(0.25)*x*x,-(0.5*(4+1)))};
    
    \addplot[blue, opacity=0, fill=blue, fill opacity=0.3, domain=-2.13:2.13] 
      {0.375*pow(1+(0.25)*x*x,-(0.5*(4+1)))}\closedcycle;
      
    \node at (axis cs: 0,0.2) {\large{$90\%$}};
    
  \end{axis}
\end{tikzpicture}
\end{center}

\begin{enumerate}[label=\textnormal{\Roman*}.]
\item Using a table of values for the $T$-distribution (with 4 degrees of freedom), we find $t^* = 2.13$.
\item The margin of error is $E = t^{*} \frac{s}{\sqrt{n}} = 2.13 \cdot \frac{8.456}{\sqrt{5}} = 8.05$
\item Therefore, with $90\%$ confidence we have
$$\begin{aligned}32 - 8.05 < \ &\mu < 32 + 8.05 \\
23.95 < \ &\mu < 40.05.\end{aligned}$$
\end{enumerate}
\end{example}

\subsection*{Robustness}\index{Robustness! of $T$-intervals}
Notice that the $T$-distribution is defined as the distribution of $T$-scores of random samples drawn from a Gaussian distribution. In practice though, we would rarely know enough about the population we are studying to be able to determine if the random variable whose mean we want to estimate is Gaussian.

Fortunately, whenever the hypotheses of the central limit theorem hold, the distribution of $T$-scores of random samples will, in fact, eventually converge to the $T$-distribution (which itself converges to the standard normal distribution) as the sample size grows larger, though the justification if this fact is well beyond the scope of this course \cite{vanderVaart}. 

Even with smaller sample sizes, statisticians like to say that the confidence interval procedure above in Example \ref{TIntExample} is `robust to non-normality'. This means that although the assumption that we're sampling from a Gaussian distribution is necessary for the formal justification of the procedure, bending the rules on this assumption will typically only introduce a small source of error. However, with small samples from a population whose distribution is not well understood, one can be justifiably suspicious, especially if there's reason to believe the distribution of the population could be fat-tailed or very asymmetric.

\begin{example}
In a study of a particular species of maple in a provincial park, the height of every tree in a random sample of 38 mature trees was measured. If the sample mean and sample standard deviation for the heights of the trees were $21.3$ metres and $4.9$ metres respectively, construct a $90\%$ confidence interval for the mean height of all trees of this species in the park.

We have a sample of $38$ values with a mean of $\littlexbar = 21.3$ and standard deviation $s = 4.9$.
\begin{center}
\begin{tikzpicture}[scale = 0.8]
  \begin{axis}[
  	  scale only axis,
      tick label style={font=\scriptsize, scale = 1/0.8},
      width = 8cm,
      height = 4cm,
      ymin=-0.0025,
      ymax=0.5,
      xmin=-3,
      xmax=3,
      xtick={-3,-2,-1,0,1,2,3},
      ytick=\empty,
      legend pos=north east,
      domain=-3:3,
      samples=200,
      thick
    ]
    \addplot[blue,opacity=0.6] {0.389*pow(1+(0.1)*x*x,-(0.5*(10+1)))};
    \addplot[blue, opacity=0, fill=blue, fill opacity=0.3, domain=-1.69:1.69] 
      {0.389*pow(1+(0.1)*x*x,-(0.5*(10+1)))}\closedcycle;
    \node at (axis cs: 0,0.2) {\large{$90\%$}};
  \end{axis}
\end{tikzpicture}
\end{center}
\begin{enumerate}[label=\textnormal{\Roman*}.]
\item Using a table of values for the $T$-distribution (with 37 degrees of freedom), we find $t^* = 1.69$.
\item We can then compute the margin of error $E = t^* \frac{s}{\sqrt{n}} = 1.69 \frac{4.9}{\sqrt{38}} = 1.343$.
\item Thus, with $90\%$ confidence,
$$\begin{aligned}21.3-1.343 < \ &\mu < 21.3+1.343 \\ 
19.957 < \ &\mu < 22.643.\end{aligned}$$
\end{enumerate}
So we conclude that with $90\%$ confidence, the mean height of all mature trees of this species in the park is between $19.9$ and $22.7$ metres.
\end{example}

In contexts like this, where we can be well assured that the distribution our sample is drawn from (the distribution of heights of adult trees of a specific species) is somewhat bell-shaped, and without fat tails, the $T$-distribution confidence interval procedure should hit very close to its target confidence, that is, around 90\% of the intervals constructed at the 90\% confidence level should contain $\mu$.

\begin{keypoint}
We have two variations of the confidence interval procedure for a population mean. In the first, the critical value comes from the standard normal distribution $Z \sim \Gaussian(0,1)$, and in the second it comes from the $T$-distribution with $n-1$ degrees of freedom.

Deciding which variation to use is simple: if the population standard deviation $\sigma$ is known, use the first variation (with $Z$) and if not, use the second (with $T$). 
\end{keypoint}

\begin{remark}
In many treatments of this subject, you'll see that this decision is made on the basis of sample size.

Since the $T$-distribution converges to $Z \sim \Gaussian(0,1)$ as $n \to \infty$, the two variations of the procedure will yield the same interval for large enough samples, so there's an argument to be made that we should use the $Z$-table whenever we can for simplicity's sake. On the other hand, there's also an argument to be made that we should emphasize the distinction between parameters and statistics ($\sigma$ and $S$ in this case) whenever we can, since the easiest way for students to lose their way in statistics is to blur this distinction.
\end{remark}

\section{Confidence Interval for a Population Proportion}\index{Confidence Interval! for a Proportion}

Suppose we're interested in estimating the proportion of individuals in a population that have some property. For example, a pollster might want to estimate the proportion of all voters in a certain riding that intend to vote for a particular political party.

If we randomly select a member of the population and consider selecting someone with the desired property a success, then $X \sim \Bernoulli(p)$ counts the number of successes in a single selection (either zero or one), where the parameter $p$ is the population proportion we want to estimate. The mean of this distribution is $E(X) = p$ (see Section \ref{BernoulliDist}), so the population mean $\mu$ is equal to the population proportion $p$.

When sampling from a Bernoulli distribution, the sample mean $\xbar = \frac{1}{n}(X_1+X_2+\cdots+X_n)$ is the proportion of individuals in the sample that have the desired property ($X_i$ takes the value $1$ when the $i^{th}$ individual has the property, and $0$ if not). Thus, the sample mean $\xbar$ is equal to the sample proportion, which we'll write as $\widehat{p}$.

In summary, if we sample from $X \sim \Bernoulli(p)$, then the population mean and sample mean are equal to the population proportion and sample proportion respectively. This means we can construct a confidence interval for a proportion in exactly the same way we made a confidence interval for a mean in the last section. We're simply applying the same procedure in the special case where our sample is drawn from a Bernoulli distribution.

To produce a confidence interval, we'll need to know, or have to estimate, the standard deviation $\sigma$ of the distribution we're sampling from. For a Bernoulli distribution, this quantity depends completely on the unknown parameter $p$. To be precise, we saw that $\sigma = \sqrt{p(1-p)}$ in Section \ref{BernoulliDist}. Using the sample proportion $\widehat{p}$ in place of the population proportion $p$ gives the estimate $\sqrt{\widehat{p}(1-\widehat{p})}$ for $\sigma$, and applying this estimate in our original margin of error formula for a mean,
$$E = z^{*}\frac{\sigma}{\sqrt{n}}  \ \ \longrightarrow \ \ E = z^{*}\sqrt{\frac{\widehat{p}(1-\widehat{p})}{n}} .$$

We can now construct confidence intervals for proportions using the same three-step procedure we use for means, but with this new margin of error formula.

\begin{example}Suppose that a pollster is interested in knowing what proportion of voters in a certain riding plan to vote NDP. In a random sample of 817 voters from this riding, 227 plan to vote NDP. Construct a $92\%$ confidence interval for the proportion of all voters in this riding who plan to vote NDP.

We have a sample of $817$ values with a sample proportion of $\widehat{p} = \frac{227}{817} = 0.278$. To create a confidence interval for the proportion of NDP voters in the riding, we follow the same three-step procedure as before.
\begin{center}
\begin{tikzpicture}[scale = 0.8]
  \begin{axis}[
  	  scale only axis,
      tick label style={font=\scriptsize, scale = 1/0.8},
      width = 8cm,
      height = 4cm,
      ymin=-0.0025,
      ymax=0.5,
      xmin=-3,
      xmax=3,
      tick label style={font=\scriptsize, scale = 1/0.8},
      xtick={-3,-2,-1,0,1,2,3},
      ytick=\empty,
      legend pos=north east,
      domain=-3:3,
      samples=200,
      thick
    ]
    \addplot[blue,opacity=0.6] {0.3989*pow(e,-x^2/2)};
    \addplot[blue, opacity=0, fill=blue, fill opacity=0.3, domain=-1.75:1.75] 
      {0.3989*pow(e,-x^2/2)}\closedcycle;
    \node at (axis cs: 0,0.2) {\large{$92\%$}}; 
  \end{axis}
\end{tikzpicture}
\end{center}
\begin{enumerate}[label=\textnormal{\Roman*}.]
\item Consulting the $Z$-table, we find $P(-z^* < Z < z^*) = 0.92$ when $z^* = 1.75$.
\item We can then compute the margin of error $E = z^* \sqrt{\frac{\widehat{p}(1-\widehat{p})}{n}} = 1.75 \sqrt{\frac{0.278 \cdot 0.722}{817}} = 0.027$.
\item With $92\%$ confidence,
$$\begin{aligned}0.278-0.027 < &p < 0.278+0.027  \\ 
0.251 < \ &p < 0.305.\end{aligned}$$
\end{enumerate}
Thus, with $92\%$ confidence, the proportion of voters in the riding who plan to vote NDP is between $25.1\%$ and $30.5\%$.
\end{example}

\begin{keypoint}
Regardless of whether we're constructing a confidence interval for a mean or proportion, and in the former case, regardless of whether we are given the population standard deviation $\sigma$ or not, we follow the same three-step procedure. It's just the margin of error formula that changes.
\begin{center}
\begin{tabular}{c|c|c}
Mean with $\sigma$ & Mean with $s$ & Proportion  \\ \hline & & \\[-2ex]
$E = z^*\frac{\sigma}{\sqrt{n}}$ & $E = t^*\frac{s}{\sqrt{n}}$ & $E = z^*\sqrt{\frac{\widehat{p}(1-\widehat{p})}{n}}$
\end{tabular}
\end{center}
Note that the margin of error formula indicates which distribution to use to find the critical value in the first step of the procedure.
\end{keypoint}


We obtained the margin of error formula for a proportion by estimating the parameter $\sigma = p(1-p)$ by the statistic $\widehat{p}(1-\widehat{p})$. This introduces a new source of variability, but it can be shown that regardless, as the sample size grows larger, the proportion of intervals that contain the population proportion $p$ will approach the confidence level $C$. Unfortunately, the tools required to show this are well beyond the scope of this course \cite{vanderVaart}.

\begin{remark}
The confidence interval for a proportion introduced in this section is commonly known as the Wald interval. There are other methods of constructing confidence intervals for proportions which actually perform better (the proportion of intervals containing $p$ will approach $C$ faster as the sample size increases). The Wald interval is usually covered in a first statistics course since the procedure for constructing it is essentially the same as the procedure for a mean from the previous section.
\end{remark}

\section{Fundamentals of Hypothesis Tests}\index{Hypothesis Test}

\subsection*{Tea Tasting}

Alice and Bob are having tea, and Alice is not happy. She says her tea tastes bad, and claims Bob must have poured the milk into her cup after pouring the tea, instead of first pouring the milk, then adding the tea, which heats the milk slower, resulting in better tasting tea.

Bob thinks this is ridiculous, and there's no way she can tell the difference, but Alice insists she can. Bob decides to challenge her on this claim. He pours five cups of tea one way, and another five cups the other way, puts them in front of Alice in a random order, and asks Alice which five cups were poured milk first.

Alice picks out the five cups correctly. Bob can't believe it. He thinks she must have just been lucky, so he calculates the probability of Alice selecting correctly purely by chance.

There are $\binom{10}{5} = 252$ ways to pick a subset of five teacups. Only one of these subsets is the correct one. This means that if Alice cannot tell the difference at all (so her correct selection happened purely by chance), the probability she selects the correct five cups is $\frac{1}{252} \approx 0.004 = 0.4\%$. This probability is so small that Bob has to concede that his experiment provided very strong evidence she can actually tell the difference.

What Bob has just done is a \emph{statistical hypothesis test}, and the probability he computed (the \emph{conditional probability} that Alice was able to make the correct selection of five cups, \emph{given} that she cannot actually tell the difference at all) is known as the \emph{P-value} of the test. If this value is sufficiently small, we conclude that it's too unlikely the evidence we've observed happened by chance, so there must be some underlying effect.

\subsection*{Basketball Big Shot}

The next day, Bob is playing basketball, and he meets a friend, Charles, who claims he makes 90\% of his free throws. Bob is skeptical, so he asks Charles to shoot twenty free throws, and of those twenty, sixteen go in the basket.

Bob calculates the probability that this happened purely by chance, assuming that Charles is actually as good as he claims. If Charles is indeed a 90\% free throw shooter, then in any sample of twenty free throws, the number he makes should be $X \sim \Binomial(20,0.9)$. Bob computes $P(X \leq 16) = 0.133$. This means that in a sample of twenty free throws, a 90\% free throw shooter will make sixteen or fewer baskets about $13\%$ of the time. So Bob has some evidence against Charles' claim, but it's not that strong. Bob is skeptical, but not yet totally convinced Charles is lying about his basketball skills.

Why did Bob compute $P(X \leq 16)$, and not $P(X = 16)$? As we take a sample of more and more free throws, the probability of any specific number of successful shots decreases, since there are simply more possible values for the number of baskets. Consider that in a sample of one hundred free throws, there are one hundred and one possibilities for the number of successful shots, so any individual outcome will have a very small probability of occurring.

When Bob is checking how strong the experimental evidence against Charles' claim is, what he wants to know is how far into the left tail of a 90\% free thrower's distribution the result lies. Is what we observed a routine occurrence for a 90\% free throw shooter, or should it almost never happen?

\begin{center}
\begin{tikzpicture}[scale = 0.6]
\begin{axis}[
    ybar,
    bar width=8pt,
    enlargelimits=0.05,
    xtick={0,...,20},
    ytick = \empty,
    tick label style={font=\scriptsize, scale = 1/0.6},
    ymin=0,
    ymax=0.3,
    legend style={at={(0.5,-0.15)}, anchor=north, legend columns=-1},
    width=15cm,
    height=8cm
]

% Highlighted bars: X <= 16
\addplot[
    ybar,
    fill=red!70,
    draw=black
] coordinates {
    (0.25, 1.024e-19)
    (1.25, 2.048e-18)
    (2.25, 1.943e-16)
    (3.25, 1.166e-14)
    (4.25, 4.872e-13)
    (5.25, 1.4606e-11)
    (6.25, 3.2475e-10)
    (7.25, 5.6315e-09)
    (8.25, 7.6149e-08)
    (9.25, 8.4604e-07)
    (10.25, 7.4864e-06)
    (11.25, 5.4512e-05)
    (12.25, 0.0003269)
    (13.25, 0.0016275)
    (14.25, 0.0067456)
    (15.25, 0.0226)
    (16.25, 0.0609)
};

% Non-highlighted bars: X > 16
\addplot[
    ybar,
    fill=gray!0,
    draw=black
] coordinates {
    (16.7, 0.1432)
    (17.7, 0.2511)
    (18.7, 0.2702)
    (19.7, 0.1216)
};

\end{axis}
\end{tikzpicture}
\end{center}

%Pictures of binomial dist with tail shaded

%If the number of successful shots is small enough that someone who makes 90\% of their free throws should almost always exceed it, we have strong evidence against Charles' claim.

The value of $P(X \leq 16)$ measures the cumulative probability in the tail, shown above in red, which tells us how far into the tail we've gone. To assess the strength of our experimental evidence, instead of using the probability of observing \emph{precisely} the evidence we have seen, we use the probability of observing evidence \emph{at least as strong} as the evidence we've seen.

\begin{note}
In the earlier tea tasting example, we also computed the probability of observing evidence \emph{at least as strong} as what we observed. It happened that in this case, Alice's selection was perfect, so she produced the strongest possible evidence she can taste the difference. If she had only selected four of the five cups that were poured milk first, and also made one incorrect selection, we would have computed the probability of selecting four or more correct cups by chance.
\end{note}

\subsection*{Statistical Hypotheses}

In the language of statistical tests, each test has a \newterm{null hypothesis}\index{Null Hypothesis}, denoted $H_0$, and an \newterm{alternative hypothesis}\index{Alternative Hypothesis}, denoted $H_a$.

In our tea tasting example, the null hypothesis, $H_0$, is that Alice cannot tell the difference between the two methods of pouring tea by tasting, and $H_a$ is that she can, and moreover, can even determine which of the two methods was used.

In our basketball example, the null hypothesis, $H_0$, is that Charles is a 90\% free throw shooter, and $H_a$ is that his free throw percentage is lower. Note that in this context, we interpret Charles' claim as a boast that he is a very good free throw shooter. This means if Charles were to make twenty out of twenty free throws, we would interpret this as evidence for his claim, not against it. 

The null hypothesis is often a statement that a certain procedure has no effect, or that a parameter is equal to a certain claimed value. In a statistical test, we do an experiment which may produce evidence for the alternative hypothesis, and decrease our confidence that the null hypothesis is correct. If this evidence is strong enough, we reject the null hypothesis, and conclude the alternative. The strength of the evidence is quantified as follows.

\begin{definition}
The \newterm{P-value}\index{P-value} in a test of $H_0$ against $H_a$ is the probability of observing evidence for $H_a$ which is at least as strong as the evidence we saw in the experiment, given that $H_0$ is true.
\end{definition}

How small does this probability need to be to convince us the null hypothesis is no longer believable in light of the evidence? This can depend on the context, but the most typical value used is 0.05, or 5\%, and this cutoff is referred to as the \newterm{significance level}\index{Significance Level} of the test. We'll discuss the details of this choice later, in Section \ref{TestPower}.

\begin{example}\label{MagicCoinEx}
You meet someone who claims to be a wizard. He says he can cast a spell that will cause a fair coin to land on heads more often than tails. You pull a coin from your pocket and ask him to cast the spell. After he does, you flip the coin thirty times, resulting in eighteen heads and twelve tails. Is this enough evidence, at the 5\% significance level, to conclude his spell actually works?

Here the null hypothesis is that the spell had no effect, so the coin is still fair. The alternative is that the probability of heads has increased. If we let $p$ denote the probability of heads after the spell has been cast, then we can abbreviate the hypotheses as $H_0: p = \frac{1}{2}$ and $H_a: p > \frac{1}{2}$. 

Under the null hypothesis, the number of heads we observe in thirty flips is $X \sim \Binomial(30,0.5)$. We observed a sample with eighteen heads, so the $P$-value for the test is $P(X \geq 18) \approx 0.181$, which was computed by summing up the binomial probabilities $P(X = 18) + P(X = 19) + \,\cdots\, + P(X = 30)$.

\begin{center}
\begin{tikzpicture}[scale = 0.6]
\begin{axis}[
    tick label style={font=\scriptsize, scale = 1/0.6},
    ybar,
    bar width=8pt,
    enlargelimits=0.05,
    xtick={5,...,25},
    ytick = \empty,
    ymin=0,
    ymax=0.165,
    legend style={at={(0.5,-0.15)}, anchor=north, legend columns=-1},
    width=15cm,
    height=8cm
]

% Highlighted bars: X <= 16
\addplot[
    ybar,
    fill=gray!0,
    draw=black
] coordinates {
(5.25, 0.000137090682983398)
(6.25, 0.000549554824829102)
(7.25, 0.0018312931060791)
(8.25, 0.00514936447143555)
(9.25, 0.0128737688064575)
(10.25, 0.0283775329589844)
(11.25, 0.0542540550231934)
(12.25, 0.08966064453125)
(13.25, 0.12516975402832)
(14.25, 0.144524574279785)
(15.25, 0.144524574279785)
(16.25, 0.12516975402832)
(17.25, 0.08966064453125)
};

% Non-highlighted bars: X > 16
\addplot[
    ybar,
    fill=red!50,
    draw=black
] coordinates {
(17.7, 0.0542540550231934)
(18.7, 0.0283775329589844)
(19.7, 0.0128737688064575)
(20.7, 0.00514936447143555)
(21.7, 0.0018312931060791)
(22.7, 0.000549554824829102)
(23.7, 0.000137090682983398)
(24.7, 2.64549255371094e-05)
};

\end{axis}
\end{tikzpicture}
\end{center}

Since $P >0.05$, we conclude that we have not seen enough evidence to reject the null hypothesis at the 5\% significance level. In other words, the coin has not shown sufficient bias in this sample of flips for us to believe the spell actually had an effect.
\end{example}

In summary, a hypothesis test to assess the strength of experimental evidence for an alternative hypothesis $H_a$ against a null hypothesis $H_0$, at a given significance level, proceeds as follows:
\begin{enumerate}[label=\Roman*.]
\item State the null hypothesis $H_0$ and the alternative hypothesis $H_a$.
\item Under the assumption that $H_0$ is true, compute the probability of observing evidence for $H_a$ which is at least as strong as the evidence observed in the experiment (the $P$-value of the test).
\item If the $P$-value is below the significance level, we have seen sufficient evidence to reject $H_0$ and conclude $H_a$, if not, our experiment did not produce strong enough evidence to reject $H_0$.
\end{enumerate}

\begin{keypoint}
If we observe strong enough evidence for the alternative, we \emph{reject} the null hypothesis. If not, then we \emph{do not reject} the null hypothesis. We never conclude the null hypothesis is true.
\end{keypoint}

This asymmetry is important to note. In the basketball example above, there is no way to convince ourselves that Charles' free throw shooting percentage is \emph{exactly} 90\% by using experimental data. We could produce a confidence interval for his shooting percentage, but no matter how small the interval, the exact value 0.9 is one of infinitely many other possible values in the interval.

One useful analogue is to imagine that someone in the room claims the temperature is $21.7^\circ$\,C, but the only thermometer available could be off by up to $2^\circ$\,C. If the thermometer reads $22.3^\circ$\,C (or even if it reads $21.7^\circ$\,C), you can't tell whether the claim is true or false, but if the thermometer reads $25.2^\circ$\,C, you know the claim is false.

\section{Hypothesis Test on a Population Mean}\index{Hypothesis Test! on a Mean}

In an old textbook, it's written that during the month of July, the mean daily high in Montreal is $26.2^{\circ}$\,C, with a standard deviation of $3.2^{\circ}$\,C. You wonder if this could still be true, so you take a sample of fifty days in July from the last decade, look up the daily high in Montreal for each, and find the the mean daily high in your sample is $27.3^{\circ}$\,C. Is this enough evidence to conclude the mean daily high in July over the last decade is above $26.2^{\circ}$\,C, at the 5\% significance level?

To answer this question, we can test the null hypothesis that the mean daily high last decade is still $26.2^{\circ}$\,C against the alternative that it's higher. These are both claims about the mean daily high for all days in July over the last decade, which is our population in this scenario. Therefore, letting $\mu$ denote the mean daily high for all days in July over the last decade, we can write the hypotheses as
$$H_0: \mu = 26.2 \text{\ \ and \ \ }H_a: \mu > 26.2.$$

Next, note that our sample mean, $\littlexbar = 27.3$, can be regarded as a realization of a Gaussian random variable. Why? We can appeal to the central limit theorem because our sample is a random sample and sufficiently large. Therefore, we can compute the probability of observing a value of $\xbar$ at least as high as the one in our sample under the null hypothesis that the mean temperature has not changed (the $P$-value of the test) by standardizing our sample mean
$$z = \frac{\littlexbar - \mu_{0}}{\frac{\sigma}{\sqrt{n}}}$$
and then using the $Z$-table. Since the test is done using the mean of a single sample, and the $P$-value is computed using the $Z$-table, this hypothesis test is known as the \newterm{one sample Z-test}.

\begin{remark}
In order the compute the test statistic $Z$, we'll need to know the population standard deviation $\sigma$. We would essentially never have this information in practice. In this example, we'll take $\sigma = 3.2$, which is not strictly correct (even if we believe the values in the old textbook were computed from a complete set of data at the time it was published, the standard deviation of daily highs in July has likely changed since then). First let's see how the test works, then we'll deal with this issue.
\end{remark}

\begin{example}
In an old textbook, it's written that during the month of July, the mean daily high in Montreal is $26.2^{\circ}$\,C, with a standard deviation of $3.2^{\circ}$\,C. You wonder if this could still be true, so you take a sample of fifty days in July from the last decade, look up the daily high in Montreal for each, and find the the mean daily high in your sample is $27.3^{\circ}$\,C. Is this enough evidence to conclude the mean daily high in July over the last decade is above $26.2^{\circ}$\,C, at the 5\% significance level?

\begin{enumerate}[label=\textnormal{\Roman*}.]
\item $H_0: \mu = 26.2$, $H_a: \mu > 26.2$.

\item Our test statistic is $z = \frac{\littlexbar - \mu_{0}}{\frac{\sigma}{\sqrt{n}}} = \frac{27.3 - 26.2}{\frac{3.2}{\sqrt{50}}} = 2.43$.

The P-value is the probability of obtaining a sample whose test statistic is at least as large as the one we observed, that is, the area under the standard normal distribution to the right of $2.43$.
\begin{center}
\begin{tikzpicture}[scale = 0.9]
  \begin{axis}[
  	  scale only axis,
      unit vector ratio=1 5 1,
      ymin=-0.0025,
      ymax=0.5,
      xmin=-3,
      xmax=3,
      tick label style={font=\scriptsize, scale = 1/1},
      xtick={-3,-2,-1,0,1,2,3},
      ytick=\empty,
      legend pos=north east,
      domain=-3:3,
      samples=200,
      thick
    ]

    \addplot[blue]  {(1/2.50663)*exp(-x*x/2)};
    
    \addplot[blue, opacity=0, fill=red, fill opacity=0.3, domain=2.43:3] 
      {(1/2.50663)*exp(-x*x/2)} \closedcycle;

  \end{axis}
\end{tikzpicture}
\end{center}
Thus, the P-value is $P(Z \geq 2.43) = 1 - P(Z \leq 2.43) = 1 - 0.9925 = 0.0075$.

\item Since $P < 0.05$, we have seen sufficient evidence to conclude that the mean daily high in July over the last decade in Montreal is above $26.2^{\circ}$\,C, at the 5\% significance level.
\end{enumerate}
\end{example}

\begin{example}
A bicycle components manufacturer claims the mean diameter of crank axles they produce is $24.98$\,mm with a standard deviation of $0.05$\,mm. A random sample of thirty-eight axles were measured very accurately, the mean diameter was found to be $24.968$\,mm. Is this enough evidence to conclude the mean diameters of axles the manufacturer is producing is not $24.98\,mm$, at the 5\% significance level?

Note that in this example, we have a \newterm{two-sided alternative hypothesis}. We're looking for evidence the mean axle diameter \textbf{is not} $24.98$\,mm. Contrast this with the last example, where we had a \newterm{one-sided alternative hypothesis}. We were looking for evidence the mean temperature was \textbf{above} $26.2^{\circ}$\,C.

\begin{keypoint}
The alternative hypothesis is chosen based on what the researchers want to be able to conclude from the test. Is the goal of the experiment to show that the population mean is higher than a certain claimed value, lower than a certain claimed value, or not equal to that value?
\end{keypoint}

\begin{enumerate}[label=\textnormal{\Roman*}.]
\item $H_0: \mu = 24.98$, $H_a: \mu \neq 24.98$.

\item Our test statistic is $z = \frac{\littlexbar - \mu_{0}}{\frac{\sigma}{\sqrt{n}}} = \frac{24.968 - 24.98}{\frac{0.05}{\sqrt{38}}} = -1.48$.

The P-value is the probability of obtaining a sample whose test statistic is at least as far from zero as the one we observed, that is, the area under the standard normal distribution the left of $-1.48$ and to the right of $1.48$.
\begin{center}
\begin{tikzpicture}[scale = 0.9]
  \begin{axis}[
  	  scale only axis,
      unit vector ratio=1 5 1,
      ymin=-0.0025,
      ymax=0.5,
      xmin=-3,
      xmax=3,
      tick label style={font=\scriptsize, scale = 1/0.9},
      xtick={-3,-2,-1,0,1,2,3},
      ytick=\empty,
      legend pos=north east,
      domain=-3:3,
      samples=200,
      thick
    ]

    \addplot[blue]  {(1/2.50663)*exp(-x*x/2)};
    
    \addplot[blue, opacity=0, fill=red, fill opacity=0.3, domain=1.48:3] 
      {(1/2.50663)*exp(-x*x/2)} \closedcycle;
      
      \addplot[blue, opacity=0, fill=red, fill opacity=0.3, domain=-3:-1.48] 
      {(1/2.50663)*exp(-x*x/2)} \closedcycle;

  \end{axis}
\end{tikzpicture}
\end{center}
Thus, the P-value is $P(Z \geq 1.48) + P(Z \leq -1.48) = 2\,P(Z \leq -1.48) = 0.1388$.

\item Since $P > 0.05$, we have not seen sufficient evidence to conclude the mean diameters of axles the manufacturer is producing is different from $24.98\,mm$, at the 5\% significance level
\end{enumerate}
\end{example}

Note that in the last two examples, we tested claims about the population mean $\mu$, and in our test, used the population standard deviation $\sigma$ provided by the same authority that gave us the value of $\mu$ we had doubts about (in the first example, a textbook, and in the second, a bicycle components manufacturer), which seems problematic. Indeed, the conclusion of the test is only properly justified if the value of $\sigma$ used in the test is the true population standard deviation, which is essentially always unknown in practice.

\subsection*{What if we don't know $\sigma$?}

We've dealt with this problem when constructing confidence intervals, and fortunately, the same approach extends to hypothesis testing. Using the sample standard deviation $s$ in place of the parameter $\sigma$, we can calculate the test statistic
$$t = \frac{\littlexbar - \mu_{0}}{\frac{s}{\sqrt{n}}}.$$

The hypothesis test will proceed in the same way, but to calculate the $P$-value we'll need to use the $T$-distribution with $n-1$ degrees of freedom in place of the standard normal distribution. Since the test is done using data from a single sample, and the $P$-value is computed using the $T$-table, this hypothesis test is known as the \newterm{one sample T-test}.

\begin{remark}
Notice that the $T$-distribution is used to account for the additional variability introduced by using the sample standard deviation $s$ as an estimate for the population standard deviation $\sigma$. This is done when constructing confidence intervals and also when performing hypothesis tests.
\end{remark}

\begin{example}
An insurance company claims that customers who switch to their insurance policies save an average of \$15 on their monthly car insurance payment. Monthly payments in a random sample of 12 customers that switched to this company are given below, before and after the switch, each column representing a single customer.
\begin{center}\begin{tabular}{rcccccccccccccc}
Pre-Switch & 70 & 115 & 96 & 84 & 97 & 72 & 90 & 83 & 77 & 121 & 101 & 93 \\
Post-Switch & 67 & 92 & 82 & 86 & 74 & 65 & 75 & 74 & 71 & 89 & 91 & 78 \\
\end{tabular}\end{center}

Is this enough evidence to conclude that, on average, customers who switch do not save as much as the company claims, at the 5\% significance level?

To evaluate a claim about savings, first we'll need to compute the savings for each customer in the sample. Doing so gives the results below. The mean amount saved in the sample is $\littlexbar = 12.92$ with a standard deviation of $s = 9.56$. The null hypothesis of the test is that the mean savings is \$15, and the alternative is that it's smaller.
\begin{center}\begin{tabular}{rcccccccccccccc}
Pre-Switch & 70 & 115 & 96 & 84 & 97 & 72 & 90 & 83 & 77 & 121 & 101 & 93 \\
Post-Switch & 67 & 92 & 82 & 86 & 74 & 65 & 75 & 74 & 71 & 89 & 91 & 78 \\
Savings & 3 & 23 & 14 & -2 & 23 & 7 & 15 & 9 & 6 & 32 & 10 & 15 \\
\end{tabular}\end{center}

\begin{enumerate}[label=\textnormal{\Roman*}.]
\item $H_0: \mu = 15$, $H_a: \mu < 15$.

\item Our test statistic is $t = \frac{\littlexbar - \mu_{0}}{\frac{s}{\sqrt{n}}} = \frac{12.92 - 15}{\frac{9.56}{\sqrt{12}}} = -0.75$.

The P-value is the probability of obtaining a sample whose test statistic is at least as small as the one we observed, that is, the area to the left of $-0.75$ under the T-distribution with 11 degrees of freedom.
\begin{center}
\begin{tikzpicture}[scale = 0.9]
  \begin{axis}[
  	  scale only axis,
      unit vector ratio=1 5 1,
      ymin=-0.0025,
      ymax=0.5,
      xmin=-3,
      xmax=3,
      tick label style={font=\scriptsize, scale = 1/0.9},
      xtick={-3,-2,-1,0,1,2,3},
      ytick=\empty,
      legend pos=north east,
      domain=-3:3,
      samples=200,
      thick
    ]

    \addplot[blue]  {0.39*(1+(x*x)/11)^(-6)};
      
      \addplot[blue, opacity=0, fill=red, fill opacity=0.3, domain=-3:-0.75] 
      {0.39*(1+(x*x)/11)^(-6)} \closedcycle;

  \end{axis}
\end{tikzpicture}
\end{center}
Consulting a table of values for the T-distribution, we can find the critical value which produces a tail area of 5\% is $-1.796$. Therefore, the P-value in our test is (much) larger than 5\%. Using a computer to explicitly compute the area gives $P = 0.234$.

\item Since $P > 0.05$, we have not seen sufficient evidence to conclude the mean monthly savings is less than \$15, at the 5\% significance level.
\end{enumerate}
\end{example}

\begin{warning}
The hypothesis test above is not properly justified! The sample consisted of only twelve individuals. Moreover, we are in a context where we have no reason to believe the population distribution under consideration (the distribution of the amount saved by switching) is Gaussian.
\end{warning}

It happened that the conclusion of the test was that there is not enough evidence in our sample to conclude the claim is false, which was the status quo before we did the test, so there's no harm done. But it's important to recognize that the $P$-value, which determines the conclusion of the test, is only an accurate assessment of the strength of evidence in our sample when the approximation given by the central limit theorem is accurate. In situations where this is not the case, one should simply refuse to use the methods of this section (and the next), and appeal to other methods instead.

\section{Hypothesis Test on a Population Proportion}\index{Hypothesis Test! on a Proportion}

In Example \ref{MagicCoinEx}, we tested whether a coin which had a magic spell cast on it had become biased towards heads, which was the supposed effect of the spell, by flipping it thirty times. In the sample of thirty flips, eighteen were heads. Letting $p$ denote the actual proportion of heads for this coin, our hypotheses were $H_0: p = \frac{1}{2}$ (the coin is fair) and $H_a: p > \frac{1}{2}$ (the coin is biased towards heads). 

We calculated the $P$-value directly from the binomial distribution, but calculating a tail probability of a binomially distributed random variable is tedious. Let's do the same test, but this time we'll use the central limit theorem to make the $P$-value calculation much more routine.

For a Bernoulli distributed random variable (which counts the number of heads in a single flip of a coin), the mean is $\mu = p$ and the standard deviation is $\sigma = \sqrt{p(1-p)}$. This means the null hypothesis $H_0: p = \frac{1}{2}$ is not just a claim about the population mean $\mu$, it's also a claim about the population standard deviation $\sigma$.

\begin{keypoint}
If in fact $p = \frac{1}{2}$, we are sampling from a population distribution with $\mu = \frac{1}{2}$ and $\sigma = \sqrt{\frac{1}{2}(1-\frac{1}{2})}$.
\end{keypoint}

The $P$-value is the probability of observing evidence at least as strong as what was observed in our sample, \emph{assuming $H_0$ is true}, so for the purposes of calculating the $P$-value, the population standard deviation $\sigma$ is known, and we can use the $Z$-test, exactly as in the last section. 

If we let $p_0$ denote the claimed proportion in the null hypothesis ($\frac{1}{2}$ in our example), $\widehat{p}$ denote the proportion observed in our sample ($\frac{18}{30}$ in our example), and make the replacements $\mu = p_0$ and $\sigma = \sqrt{p_0(1-p_0)}$, our test statistic becomes

$$z = \frac{\widehat{p} - p_0}{\sqrt{\frac{p_0(1-p_0)}{n}}}.$$

\begin{example}
You meet someone who claims to be a wizard. He says he can cast a spell that will cause a fair coin to land on heads more often than tails. You pull a coin from your pocket and ask him to cast the spell. After he does, you flip the coin thirty times, resulting in eighteen heads and twelve tails. Is this enough evidence, at the 5\% significance level, to conclude his spell actually works?

\begin{enumerate}[label=\textnormal{\Roman*}.]
\item $H_0: p = \frac{1}{2}$, $H_a: p > \frac{1}{2}$.

\item Our test statistic is $z = \frac{\widehat{p} - p_0}{\sqrt{\frac{p_0(1-p_0)}{n}}} = \frac{\frac{18}{30} - \frac{1}{2}}{\sqrt{\frac{\frac{1}{2}(1-\frac{1}{2})}{30}}} = 1.10$.

The P-value is the probability of obtaining a sample whose test statistic is at least as large as the one we observed, that is, the area under the standard normal distribution to the right of $1.10$.
\begin{center}
\begin{tikzpicture}[scale = 0.9]
  \begin{axis}[
  	  scale only axis,
      unit vector ratio=1 5 1,
      ymin=-0.0025,
      ymax=0.5,
      xmin=-3,
      xmax=3,
      tick label style={font=\scriptsize, scale = 1/0.9},
      xtick={-3,-2,-1,0,1,2,3},
      ytick=\empty,
      legend pos=north east,
      domain=-3:3,
      samples=200,
      thick
    ]

    \addplot[blue]  {(1/2.50663)*exp(-x*x/2)};
    
    \addplot[blue, opacity=0, fill=red, fill opacity=0.3, domain=1.10:3] 
      {(1/2.50663)*exp(-x*x/2)} \closedcycle;

  \end{axis}
\end{tikzpicture}
\end{center}
Thus, the P-value is $P(Z \geq 1.10) = 1 - P(Z \leq 1.10) = 1 - 0.8643 = 0.1357$.

\item Since $P > 0.05$, we have not seen sufficient evidence to conclude that the coin is biased towards heads, at the 5\% significance level.
\end{enumerate}
\end{example}

Note that even though the conclusion is the same as it was in Example \ref{MagicCoinEx}, the $P$-value is not. The majority of the discrepancy is due to the continuous approximation given by the central limit theorem. In Example \ref{MagicCoinEx}, the $P$-value included the probability exactly 18 heads are observed, but not the probability 19 heads are observed. To mimic this in our continuous approximation, we can use $\widehat{p} = \frac{18.5}{30}$ instead of $\widehat{p} = \frac{18}{30}$ in our test, to have an even split between the two borderline cases. This is known as a continuity correction. We won't cover this in the course, but it's a good exercise to run through the test again with this change, and see that the $P$-value comes out very close to the value we obtained by working directly with the binomial distribution. 

As the sample size increases, the continuity correction will have a smaller and smaller effect on the $P$-value of the test, so as long as the sample size is large enough, the discrepancy between the $P$-values will be insignificant.

\begin{example}
A botanist estimates that $73$\% of the mature trees in a forest are maples. In a random sample of 200 mature trees, 161 of them were maples. Is this enough evidence to conclude the botanist's estimate is not correct, at the 5\% significance level?

We're assessing whether our sample data is enough to convince us the botanist's estimate of $73$\% is incorrect, so the test will use a two-sided alternative. The botanist's estimate could be incorrect by being either too low or too high.

\begin{enumerate}[label=\textnormal{\Roman*}.]
\item $H_0: p = 0.73$, $H_a: p \neq 0.73$.

\item Our test statistic is $z = \frac{\widehat{p} - p_0}{\sqrt{\frac{p_0(1-p_0)}{n}}} = \frac{\frac{161}{200} - 0.73}{\sqrt{\frac{0.73(1-0.73)}{200}}} = 2.39$.

The P-value is the probability of obtaining a sample whose test statistic is at least as far from zero as the one we observed, that is, the area under the standard normal distribution to the left of $-2.39$ and to the right of $2.39$.
\begin{center}
\begin{tikzpicture}[scale = 0.9]
  \begin{axis}[
  	  scale only axis,
      unit vector ratio=1 5 1,
      ymin=-0.0025,
      ymax=0.5,
      xmin=-3,
      xmax=3,
      tick label style={font=\scriptsize, scale = 1/0.9},
      xtick={-3,-2,-1,0,1,2,3},
      ytick=\empty,
      legend pos=north east,
      domain=-3:3,
      samples=200,
      thick
    ]

    \addplot[blue]  {(1/2.50663)*exp(-x*x/2)};
    
    \addplot[blue, opacity=0, fill=red, fill opacity=0.3, domain=2.39:3] 
      {(1/2.50663)*exp(-x*x/2)} \closedcycle;
      
      \addplot[blue, opacity=0, fill=red, fill opacity=0.3, domain=-3:-2.39] 
      {(1/2.50663)*exp(-x*x/2)} \closedcycle;

  \end{axis}
\end{tikzpicture}
\end{center}
Thus, the P-value is $P(Z \geq 2.39) + P(Z \leq -2.39) = 2\,P(Z \leq -2.39) = 0.0164$.

\item Since $P < 0.05$, we have seen sufficient evidence to conclude the botanist's estimate of $73$\% maples is incorrect, at the 5\% significance level.
\end{enumerate}
\end{example}

\begin{keypoint}
Regardless of whether we're doing a hypothesis test on a mean or proportion, and in the former case, regardless of whether we are given the population standard deviation $\sigma$ or not, we follow the same three-step procedure. It's just the test statistic that changes.
\begin{center}
\begin{tabular}{c|c|c}
Mean with $\sigma$ & Mean with $s$ & Proportion  \\ \hline & & \\[-2ex]
$z = \frac{\littlexbar - \mu_{0}}{\frac{\sigma}{\sqrt{n}}}$ & $t = \frac{\littlexbar - \mu_{0}}{\frac{s}{\sqrt{n}}}$ & $z = \frac{\widehat{p} - p_0}{\sqrt{\frac{p_0(1-p_0)}{n}}}$
\end{tabular}
\end{center}
Note that the name of the test statistic indicates which distribution to use when computing the $P$-value of the test, and in each case, the test can be done with a one-sided or two-sided alternative hypothesis.
\end{keypoint}

\section{Significance, Confidence, \& Power}\label{TestPower}

Each hypothesis test ends with a decision to either reject the null hypothesis, and conclude the alternative is correct, or refuse to reject the null hypothesis because we have not observed strong enough evidence against it in our sample.

This means there are only two ways to come to the \emph{incorrect} conclusion, either by rejecting the null hypothesis when, in fact, it's true, or by not rejecting the null hypothesis when, in fact, the alternative is true. These are unfortunately known by the uninspiring names \newterm{Type I error}\index{Type I Error} and \newterm{Type II error}\index{Type II Error}.
\begin{center}
\begin{tabular}{|c|c|c|}
\hline
\textbf{Decision} & \textbf{$H_0$ is True} & \textbf{$H_a$ is True} \\
\hline
Do not reject $H_0$ & Correct & Type II Error \\
\hline
Reject $H_0$ & Type I Error & Correct \\
\hline
\end{tabular}
\end{center}

\begin{remark}
A useful analogy here is a jury trial. The defendant is only deemed guilty if the there's sufficient evidence, so $H_0$ is innocence and $H_a$ is guilt. In this context, convicting an innocent defendant is a Type I error, and acquitting a guilty defendant is a Type II error.
\end{remark}

To understand how these occur, consider a one-sample $Z$-test of $H_0: \mu = \mu_0$ against $H_a: \mu > \mu_0$, done at the 5\% significance level, as usual.

We've seen that the distribution of the sample mean $\xbar$ is centred at the population mean $\mu$, and if the hypotheses of the central limit theorem hold, is approximately Gaussian. In the figures below, we'll draw this distribution in blue. The null hypothesis, which may or may not be true, claims the population mean is $\mu_0$.

\begin{center}
\begin{tikzpicture}[scale = 0.9]
  \begin{axis}[
  	  scale only axis,
      width=10cm,
      height=4cm,
      ymin=-0.0025,
      ymax=0.5,
      xmin=-4,
      xmax=4,
      tick label style={font=\scriptsize, scale = 1/0.9},
      xtick={-1,1},
      xticklabels={$\mu_0$,$\mu$},
      ytick=\empty,
      legend pos=north east,
      domain=-7:7,
      samples=200,
      thick
    ]

    \addplot[black, dashed]  {(1/2.50663)*exp(-(x+1)*(x+1)/2)};
    %\addplot[black, thin] (-1,0) -- (-1,1/2.50663);
   
    \addplot[blue]  {(1/2.50663)*exp(-(x-1)*(x-1)/2)};

  \end{axis}
\end{tikzpicture}
\end{center}

We'll call these two distributions the \newterm{true distribution}\index{True Distribution} of $\xbar$ (centred at the true mean $\mu$), and the \newterm{null distribution}\index{Null Distribution} of $\xbar$ (centred at at the hypothesized value $\mu_0$) respectively. Note that when we calculate the test statistic, we use the hypothesized mean $\mu_0$.

Now if the null hypothesis is, in fact, true, then the two distributions are identical, and the correct decision is to not reject $H_0$. We'll incorrectly reject $H_0$ (a Type I error) when the area to the right of the test statistic (the $P$-value) is under $5\%$, which happens when our sample is among the $5\%$ highest samples, in terms of the sample mean.

\begin{center}
\begin{tikzpicture}[scale = 0.9]
  \begin{axis}[
  	  scale only axis,
      width=8cm,
      height=4cm,
      ymin=-0.0025,
      ymax=0.5,
      xmin=-3.5,
      xmax=3.5,
      tick label style={font=\scriptsize, scale = 1/0.9},
      xtick={0},
      xticklabels={$\mu_0 = \mu$},
      ytick=\empty,
      legend pos=north east,
      domain=-7:7,
      samples=200,
      thick
    ]
   
    \addplot[blue]  {(1/2.50663)*exp(-(x)*(x)/2)};
    
    \addplot[blue, opacity=0, fill=red, fill opacity=0.3, domain=1.96:3.5] 
      {(1/2.50663)*exp(-x*x/2)} \closedcycle;
      
      \addplot[blue, opacity=0, fill=green, fill opacity=0.3, domain=-3.5:1.96] 
      {(1/2.50663)*exp(-x*x/2)} \closedcycle;
      
      \addplot[black, dashed]  {(1/2.50663)*exp(-(x)*(x)/2)};
      
      \node at (axis cs:0,0.165) {$1-\alpha$};
    
     \addplot[blue, opacity=0, fill=none, fill opacity=1, domain=1.96:4, postaction={pattern=north east lines}] 
      {(1/2.50663)*exp(-(x)*(x)/2)} \closedcycle;

  \end{axis}
\end{tikzpicture}
\end{center}

Thus, when we choose the significance level of the test, we're setting the probability of a Type I error. Why not always take a very small significance level then, since making a Type I error is something we would ideally never do?

The problem comes when the null hypothesis is false. If the population mean $\mu$ is higher than $\mu_0$, the correct decision is to reject $H_0$. In this situation, a Type II error will occur if the $P$-value is above $5\%$. When we select a random sample, its mean is drawn from the true distribution of $\xbar$ (the blue distribution), then we take that realization $\littlexbar$ and compute the $P$-value using the null distribution of $\xbar$ (the dashed black distribution). 

The green area below is the probability of correctly rejecting $H_0$ (obtaining a sample whose test statistic is far enough into the tail of the null distribution to have $P < 0.05$), and the red area is the probability of not rejecting $H_0$ (obtaining a sample with $P > 0.05$).

\begin{center}
\begin{tikzpicture}[scale = 0.9]
  \begin{axis}[
  	  scale only axis,
      width=10cm,
      height=4cm,
      ymin=-0.0025,
      ymax=0.5,
      xmin=-2.5,
      xmax=5.5,
      tick label style={font=\scriptsize, scale = 1/0.9},
      xtick={0,3},
      xticklabels={$\mu_0$,$\mu$},
      ytick=\empty,
      legend pos=north east,
      domain=-2.5:5.5,
      samples=200,
      thick
    ]
    
    \addplot[blue, opacity=0, fill=green, fill opacity=0.3, domain=1.96:5.5] 
      {(1/2.50663)*exp(-(x-3)*(x-3)/2)} \closedcycle;
      
      \addplot[blue, opacity=0, fill=red, fill opacity=0.3, domain=-3:1.96] 
      {(1/2.50663)*exp(-(x-3)*(x-3)/2)} \closedcycle;

    \addplot[black, dashed]  {(1/2.50663)*exp(-(x)*(x)/2)};
    %\addplot[black, thin] (-1,0) -- (-1,1/2.50663);
   
    \addplot[blue]  {(1/2.50663)*exp(-(x-3)*(x-3)/2)};
    
    \node at (axis cs:3,0.165) {$1-\beta$};
    
    \addplot[blue, opacity=0, fill=none, fill opacity=1, domain=1.96:4, postaction={pattern=north east lines}] 
      {(1/2.50663)*exp(-(x)*(x)/2)} \closedcycle;
    
  \end{axis}
\end{tikzpicture}
\end{center}

The probability of not rejecting $H_0$ when it's true is the \newterm{confidence}\index{Confidence Level! of a Hypothesis Test} level of the test, denoted $1-\alpha$, and the probability of rejecting $H_0$ when it's false is called the \newterm{power}\index{Power! of a Hypothesis Test} of the test, denoted $1- \beta$. As we increase the confidence level (that is, decrease the significance level) of the test, the Type I error rate ($\alpha$) shrinks, but the Type II error rate ($\beta$) grows.
\begin{center}
\begin{tabular}{|c|c|c|}
\hline
\textbf{Decision} & \textbf{$H_0$ is True} & \textbf{$H_a$ is True} \\
\hline
Do not reject $H_0$ & $1-\alpha$ & $\beta$ \\
\hline
Reject $H_0$ & $\alpha$ & $1-\beta$ \\
\hline
\end{tabular}
\end{center}

\begin{keypoint}
There is a tradeoff between confidence and power. A test done at the 1\% significance level has a higher confidence level, but lower power, than a test done at the 5\% significance level.
\end{keypoint}

\subsection*{Effect Size, Sample Size, and Significance}

The difference between the true mean $\mu$ and the hypothesized mean $\mu_0$ is often referred to as the \newterm{effect size}\index{Effect Size}. Consider for example a new cold medication on the market. If those who have a cold and take no medication have symptoms for a mean of 5.7 days, and those who have a cold and take the new medication have symptoms for a mean of 5.2 days, one could say the effect of the medication was to decrease the mean symptom length by 0.5 days.

A hypothesis test is an assessment of the strength of evidence for an effect (potentially in a certain direction depending on the alternative hypothesis chosen) in sample data, and the power of the test, $\beta$, is the probability of detecting an effect when an effect actually exists. When we reject $H_0$ and conclude that we've seen sufficient evidence for an effect, it's common to say that our study or experiment has produced a \newterm{statistically significant}\index{Statistical Significance} result.

\begin{remark}
Notice that as the effect size increases, and $\mu$ moves farther away from $\mu_0$, the power $\beta$ grows. Also, as the sample size increases, and the variance of both distributions decreases, $\beta$ grows. The power of the test is an increasing function of both the effect size and the sample size.
\end{remark}

To detect a very small effect, we would typically need a very large sample, and a large effect can typically be detected with a small sample. Consider repeatedly flipping a coin with a $90\%$ heads bias, and a coin with a $55\%$ heads bias. We should detect the bias of the former coin with far fewer flips.

\begin{example}
Suppose we perform a one-sample $Z$-test of $H_0: \mu = 5$ against $H_a: \mu < 5$, given that $\sigma = 3$. In fact, the true value of $\mu$ is $4$. If we use a sample of size $50$, and the test is done at the $5\%$ significance level, what is the power of the test?

\begin{center}
\begin{tikzpicture}[scale = 0.9]
  \begin{axis}[
  	  scale only axis,
      width=10cm,
      height=4cm,
      ymin=-0.0025,
      ymax=1.2,
      xmin=2.8,
      xmax=6.2,
      tick label style={font=\scriptsize, scale = 1/0.9},
      xtick={4,5},
      xticklabels={$4$,$5$},
      ytick=\empty,
      legend pos=north east,
      domain=2.8:6.2,
      samples=200,
      thick
    ]
    
    \addplot[blue, opacity=0, fill=red, fill opacity=0.3, domain=4.17:7.5] 
      {((0.9403)*exp(-24*(x-4)*(x-4)/9)} \closedcycle;
      
      \addplot[blue, opacity=0, fill=green, fill opacity=0.3, domain=0:4.17] 
      {(0.9403)*exp(-24*(x-4)*(x-4)/9)} \closedcycle;

    \addplot[black, dashed]  {(0.9403)*exp(-24*(x-5)*(x-5)/9)};
    %\addplot[black, thin] (-1,0) -- (-1,1/2.50663);
   
    \addplot[blue]  {(0.9403)*exp(-24*(x-4)*(x-4)/9)};
    
    \addplot[black, dashed, opacity=0, fill=none, fill opacity=1, domain=1:4.17, postaction={pattern=north east lines}] 
      {(0.9403)*exp(-24*(x-5)*(x-5)/9)} \closedcycle;

  \end{axis}
\end{tikzpicture}
\end{center}

The test statistic $z$ for a sample with mean $\littlexbar$ is given by
$$z = \frac{\littlexbar - \mu_0}{\frac{\sigma}{\sqrt{n}}} = \frac{\littlexbar - 5}{\frac{3}{\sqrt{50}}}=2.36(\littlexbar - 5)$$
and since $P(Z < -1.65) = 0.05$, we will (correctly) reject $H_0$ when
$$\begin{aligned}2.36(\littlexbar - 5) &< -1.65 \\
\littlexbar - 5 &< -0.70 \\
\littlexbar &< 4.30.\end{aligned}$$
This means the power of the test is the probability a random sample has a mean under $4.30$. The true distribution of $\xbar$ has $\mu = 4$, so by the central limit theorem $\xbar \sim AG(4,\frac{3}{\sqrt{50}})$, and we can calculate
$$\begin{aligned}1-\beta &= P(\xbar < 4.30) \\ &= P(Z < \textstyle\frac{4.30 - 4}{\frac{3}{\sqrt{50}}}) \\ &= P (Z < 0.40) = 0.7071\end{aligned}$$
\end{example}

It's instructive to re-run the calculation after changing the sample size, population standard deviation, or significance level to see the effect. Notice that although the power of the test depends on all of these, the confidence depends only on the significance level. In the scenario where $H_0$ is true, a test done at the $5\%$ significance level will not reject it precisely $95\%$ of the time.

\begin{keypoint}
When we test for the presence of some effect, in a case where there really is none, we will incorrectly conclude there is an effect $5\%$ of the time. This is not a error, it is a deliberate choice. We accept the 5\% error rate in order to have a reasonable chance of detecting an effect when there is one.

Similarly, 5\% of confidence intervals constructed at the 95\% confidence level will not contain the population mean $\mu$. This is, again, a deliberate choice we make to be able to have reasonably close bounds on the value of $\mu$.
\end{keypoint}

This is why the scientific method is so important. If we incorrectly conclude there is an effect at the 5\% significance level when none actually exists, but publish the result and the experimental method, then when others try to replicate it, the false positive will quickly be detected, since each trial done with a new random sample only has a 5\% chance of coming to the same incorrect conclusion.

\begin{remark}
Conversely, we can undermine the scientific process by repeatedly replicating an experiment until achieving significance (when there is no effect, this will happen by chance in 5\% of experiments done at the 5\% significance level), and deny the experiments that did not achieve significance ever happened.

This practice is known as \emph{$P$-hacking}, and is a real problem in modern scientific practice, where often there is only an incentive to publish experiments that are novel (never been done before), and have achieved significance. Attempts have been made to address this problem by giving researchers incentives to replicate more experiments, and ways to publish their findings regardless of whether they achieved significance or not.
\end{remark}

% Testing 90% coin for unfairness vs 55% coin, sample size and effect siuze discussion

\section{Pearson's Chi-Square Test}\index{Pearson's Chi-Square Test}

\subsection*{The Unfair Die}

%If a die is rolled ten times, and the result is six every time, this does not imply the die is not fair, but if we're willing to consider the hypothesis that the die is biased towards six, then rolling ten sixes in ten rolls provides some very strong evidence for that hypothesis. Unless we have reason to believe with absolute certainty that the die is fair, we should not be willing to accept that possibility once we've seen enough evidence against it.

Suppose we roll a die fifty times and tabulate the number of rolls in our sample where each outcome occurred, as well as the number we would expect to find on a fair die. This way we can compare the distribution of results in our sample to the expected distribution for a fair die.

\renewcommand{\arraystretch}{1.25}
\begin{center}
\begin{tabular}{|c|c|c|c|c|c|c|}
\hline
Outcome & $1$ & $2$ & $3$ & $4$ & $5$ & $6$ \\
\hline
Observed Count & $8$ & $4$ & $11$ & $6$ & $9$ & $12$ \\
\hline
Expected Count & $50\cdot\frac{1}{6}$ & $50\cdot\frac{1}{6}$ & $50\cdot\frac{1}{6}$ & $50\cdot\frac{1}{6}$ & $50\cdot\frac{1}{6}$ & $50\cdot\frac{1}{6}$ \\
\hline
\end{tabular}
\end{center}

Regardless of the results in our sample, it's always possible the die is fair, and we take this as our null hypothesis, $H_0$, with the alternative that the die is not fair, that is, the probabilities of the six outcomes are not exactly the same.

The larger the differences between the counts in the sample and the expected counts from a fair die, the stronger the evidence for the alternative $H_a$. Note that although the expected counts are not actually realizable in any sample (they're not integers), the bigger the difference between the observed and expected counts, the stronger the evidence that the die is not fair.

To measure the extent to which the observed counts in our sample (denoted $\mathcal{O}_i$) differ from the expected counts on a fair die (denoted $E_i$), we can sum the squared differences between the sample counts and the expected counts, and weight the terms with their expected counts as follows.
$$\sum_{i=1}^6 \frac{(\mathcal{O}_i - E_i)^2}{E_i} = \frac{(8-\frac{50}{6})^2}{\frac{50}{6}} + \frac{(4-\frac{50}{6})^2}{\frac{50}{6}} + \dots + \frac{(12-\frac{50}{6})^2}{\frac{50}{6}} = 5.44 $$

When the null hypothesis is true, that is, if the die is in fact fair, the values of this statistic across all random samples have a particular distribution, called the chi-square distribution.

\begin{definition}\index{Distribution! Chi-Square}
The \newterm{chi-square distribution}\index{Chi-Square Distribution} with $n$ degrees of freedom is defined as the distribution of $\chi^2 = {Z_1}^2 + {Z_2}^2 + \dots + {Z_n}^2$ where all $Z_i \sim \Gaussian(0,1)$ and all are independent. Graphs of the densities of chi-square random variables with one, five, and ten degrees of freedom are shown in red, blue, and green respectively.
\end{definition}
\begin{center}
\begin{tikzpicture}[scale = 1]
  \begin{axis}[
  	  scale only axis,
      unit vector ratio=1 20 1,
      ymin=-0.0025,
      ymax=0.3,
      xmin=-0.1,
      xmax=15,
      tick label style={font=\scriptsize, scale = 1/1},
      xtick={0,1,2,3,4,5,6,7,8,9,10,11,12,13,14,15},
      ytick=\empty,
      legend pos=north east,
      domain=0:15,
      samples=200,
      thick
    ]
    % Chi-square with 1 degree of freedom
    \addplot[red] {1/(sqrt(2*pi*x))*exp(-x/2)};

    % Chi-square with 5 degrees of freedom
    \addplot[blue]  {1/(pow(2,2.5)*1.32934)*pow(x,1.5)*exp(-x/2)};

    % Chi-square with 10 degrees of freedom
    \addplot[green!50!black]  {1/(pow(2,5)*24)*pow(x,4)*exp(-x/2)};
  \end{axis}
\end{tikzpicture}
\end{center}

\begin{proposition}\label{ChiSquareNullDist}
Consider a random sample of $n$ values from a given discrete distribution which takes $k$ possible values. Let $\mathcal{O}_i$ be the number of times the $i^{th}$ value occurs in the sample, and let $E_i$ be $n$ times the probability of observing the $i^{th}$ value. Then as $n \to \infty$ the distribution of the statistic 
$$\sum_{i=1}^k \frac{(\mathcal{O}_i - E_i)^2}{E_i}$$
approaches the chi-square distribution with $k-1$ degrees of freedom.
\end{proposition}

The proof of this fact is beyond the scope of this course, and appears in more advanced texts \cite{vanderVaart}. It was first shown in 1900 by the statistician Karl Pearson, which is why the test we're about to perform bears his name. Now that we know the distribution of the statistic we computed, which we'll now call $\chi^2 = 5.44$, under the hypothesis that the die is fair, we can carry out a hypothesis test, as we did in the last few sections.

\begin{example}
A die, which may or may not be fair, is rolled fifty times. The results are 8 ones, 4 twos, 11 threes, 6 fours, 9 fives, and 12 sixes (the observed values in the table on the last page). Can we conclude the die is not fair, at the 5\% significance level?

\begin{enumerate}[label=\textnormal{\Roman*}.]
\item $H_0: \text{The die is fair}$, $H_a: \text{The die is not fair}$.

\item Our test statistic is $\chi^2 = \sum_{i=1}^6 \frac{(\mathcal{O}_i - E_i)^2}{E_i} = \frac{(8-\frac{50}{6})^2}{\frac{50}{6}} + \frac{(4-\frac{50}{6})^2}{\frac{50}{6}} + \dots + \frac{(12-\frac{50}{6})^2}{\frac{50}{6}} = 5.44.$ \\

The P-value is the probability of obtaining a sample whose test statistic is at least as large as the one we observed (the larger the $\chi^2$ value, the farther the observed counts are from the expected counts). Thus, the P-value is $P(\chi^2 \geq 5.44)$, i.e., the probability a chi-square random variable with 5 degrees of freedom takes a value greater than or equal to $5.44$.
\begin{center}
\begin{tikzpicture}[scale = 1]
  \begin{axis}[
  	  scale only axis,
      unit vector ratio=1 20 1,
      ymin=-0.0025,
      ymax=0.3,
      xmin=-0.1,
      xmax=15,
      tick label style={font=\scriptsize, scale = 1/1},
      xtick={0,1,2,3,4,5,6,7,8,9,10,11,12,13,14,15},
      ytick=\empty,
      legend pos=north east,
      domain=0:15,
      samples=200,
      thick
    ]

    % Chi-square with 5 degrees of freedom
    \addplot[blue]  {1/(pow(2,2.5)*1.32934)*pow(x,1.5)*exp(-x/2)};
    
    \addplot[blue, opacity=0, fill=red, fill opacity=0.3, domain=5.44:15] 
      {1/(pow(2,2.5)*1.32934)*pow(x,1.5)*exp(-x/2)} \closedcycle;

  \end{axis}
\end{tikzpicture}
\end{center}
Consulting a table of values for the chi-square distribution, we can find the critical value which produces a tail area of 5\% is about 11. Therefore, the P-value in our test is (much) larger than 5\%. Using a computer to explicitly compute the area gives $P = 0.366$.

\item Since $P > 0.05$, we have not seen sufficient evidence to conclude that the die is not fair, at the 5\% significance level.
\end{enumerate}
\end{example}

This test can be used in many scenarios where we want to know whether some treatment results in a change from a known distribution, and is usually known as the \newterm{chi-square goodness-of-fit test}\index{Pearson's Chi-Square Test! for Goodness of Fit}. It attempts to answer the very general question: could this sample have come from that distribution?

For example, from past data, we may know the distribution of the number of weeks it takes patients to fully recover from a certain surgery. If adding physiotherapy sessions has no effect on recovery time, the recovery times in a sample of patients who undergo physiotherapy is a sample from this same distribution (the null hypothesis of the test), and if the physiotherapy is actually having some kind of effect, it's a sample from some other distribution (the alternative hypothesis of the test).

\subsection*{Testing Independence}

We can use a variation of this test, the \newterm{chi-square test of independence}\index{Pearson's Chi-Square Test! of Independence}, to check for evidence that two variables are not independent in a given population. Suppose we ask a random sample of 109 people at the college, all of whom are in their twenties, thirties, or fourties, whether they prefer to drink tea or coffee. The number of individuals in the sample who fall into each possible category of age and drink preference is given in the table below.
\renewcommand{\arraystretch}{1.25}
\begin{center}
\begin{tabular}{|c|c|c|c|c|c|c|}
\hline
 & Twenties & Thirties & Fourties \\
\hline
Prefer Tea & $17$ & $7$ & $10$  \\
\hline
Prefer Coffee & $23$ & $25$ & $27$ \\
\hline
\end{tabular}
\end{center}

Notice that there are finitely many categories into which values of each variable can belong. We'll add the totals along each row and column, as well as the grand total of all six values in the bottom right. The result is known as a \newterm{contingency table}\index{Contingency Table}.
\renewcommand{\arraystretch}{1.25}
\begin{center}
\begin{tabular}{|c|c|c|c|c|c|c|}
\hline
 & Twenties & Thirties & Fourties & Total \\
\hline
Prefer Tea & $17$ & $7$ & $10$ & $34$  \\
\hline
Prefer Coffee & $23$ & $25$ & $27$ & $75$ \\
\hline
Total & $40$ & $32$ & $37$  & $109$ \\
\hline
\end{tabular}
\end{center}

Now suppose that we only had the row and column totals, but we knew that age and drink preference were independent variables.
\renewcommand{\arraystretch}{1.25}
\begin{center}
\begin{tabular}{|c|c|c|c|c|c|c|}
\hline
 & Twenties & Thirties & Fourties & Total \\
\hline
Prefer Tea & $?$ & $?$ & $?$ & $34$  \\
\hline
Prefer Coffee & $?$ & $?$ & $?$ & $75$ \\
\hline
Total & $40$ & $32$ & $37$  & $109$  \\
\hline
\end{tabular}
\end{center}

Since the probability a randomly selected individual prefers tea and is in their twenties is the product of the probability they prefer tea and the probability they are in their twenties (that is the definition of independence), we can compute the number of individuals who prefer tea and are in their twenties we would expect to see in our sample as $P(\text{Tea} \cap \text{Twenties}) = \frac{34}{109}\frac{40}{109}$, so the expected number in that cell is $N(\text{Tea} \cap \text{Twenties}) = 109\cdot\frac{34}{109}\cdot\frac{40}{109} = \frac{34\cdot 40}{109} = 12.48$.

In general, if the two variables we measured are independent, the expected count in each cell in the table is the product of the corresponding row and column totals, divided by the grand total, briefly, $E_{i,j} = \frac{R_i \cdot C_j}{n}$. Filling in each entry in this manner gives us the expected counts in our sample under the hypothesis that the two variables are independent.
\renewcommand{\arraystretch}{1.25}
\begin{center}
\begin{tabular}{|c|c|c|c|c|c|c|}
\hline
 & Twenties & Thirties & Fourties & Total \\
\hline
Prefer Tea & $12.75$ & $9.98$ & $11.54$ & $34$  \\
\hline
Prefer Coffee & $27.52$ & $22.02$ & $25.46$ & $75$ \\
\hline
Total & $40$ & $32$ & $37$  & $109$  \\
\hline
\end{tabular}
\end{center}

Computing the chi-square statistic (the weighted sum of the squared differences between the observed and expected counts, each term weighted with its expected count, as before) gives
$$\sum_{i=1}^2\sum_{j=1}^3 \frac{(\mathcal{O}_{i,j} - E_{i,j})^2}{E_{i,j}} = \frac{(17-12.75)^2}{12.75} + \frac{(7-9.98)^2}{9.98} + \dots + \frac{(27-25.46)^2}{25.46} = 3.75.$$

\begin{proposition}
Consider a random sample of $n$ individuals, and two categorical variables $X$ and $Y$ which can take $r$ and $c$ possible values respectively in the population. Let $\mathcal{O}_{i,j}$ be the number of times the $i^{th}$ value of $X$ occurs together with the $j^{th}$ value of $Y$ in the sample, and let $E_{i,j} = \frac{R_i \cdot C_j}{n}$, where $R_i$ is the total number of times the $i^{th}$ value of $X$ occurs in the sample, and $C_j$ is the total number of times the $j^{th}$ value of $Y$ occurs in the sample. If the variables $X$ and $Y$ are independent in the population, then as $n \to \infty$ the distribution of the statistic 
$$\sum_{i=1}^r\sum_{j=1}^c \frac{(\mathcal{O}_{i,j} - E_{i,j})^2}{E_{i,j}}$$
approaches the chi-square distribution with $(r-1)(c-1)$ degrees of freedom.
\end{proposition}

As with Proposition \ref{ChiSquareNullDist}, the proof is beyond the scope of this course, but it's possible to give an intuitive interpretation of the number of degrees of freedom. Given fixed row and column totals in a contingency table with $r$ rows and $c$ columns, $(r-1)(c-1)$ is precisely the number of entries that need to be specified to determine the table completely. 

Consider a $2 \times 2$ contingency table, for example, with fixed row and column totals. Once any single value is specified, the other entry in the same row can be obtained by subtracting that value from the row total. At that point, each column has one entry fixed, and the other can be obtained by subtracting it from the respective column total. This means that only a single entry in the table can be freely chosen, and once that's done, the rest are completely determined, hence $(2-1)(2-1) = 1$ degree of freedom. This reasoning generalizes to larger tables as well.

With the distribution of the chi-square statistic under the hypothesis that the variables are independent established, we can perform the hypothesis test in the usual framework of this chapter.

\begin{example}
A random sample of 109 people at a college, all of whom are in their twenties, thirties, or fourties, are asked whether they prefer to drink tea or coffee. The number of individuals in the sample who fall into each possible category of age and drink preference is given in the table below. Can we conclude that age and drink preference are not independent variables, at the 10\% significance level?
\renewcommand{\arraystretch}{1.25}
\begin{center}
\begin{tabular}{|c|c|c|c|c|c|c|}
\hline
 & Twenties & Thirties & Fourties \\
\hline
Prefer Tea & $17$ & $7$ & $10$  \\
\hline
Prefer Coffee & $23$ & $25$ & $27$ \\
\hline
\end{tabular}
\end{center}

We begin by constructing the contingency table, summing along each row and column, as above, and then compute the expected count in each cell. The results are given below.
\renewcommand{\arraystretch}{1.25}
\begin{center}
\begin{tabular}{|c|c|c|c|c|c|c|}
\hline
 & Twenties & Thirties & Fourties & Total \\
\hline
Prefer Tea & $12.75$ & $9.98$ & $11.54$ & $34$  \\
\hline
Prefer Coffee & $27.52$ & $22.02$ & $25.46$ & $75$ \\
\hline
Total & $40$ & $32$ & $37$  & $109$  \\
\hline
\end{tabular}
\end{center}

\begin{enumerate}[label=\textnormal{\Roman*}.]
\item $H_0: \text{Age and drink preference are independent}$, $H_a: \text{Age and drink preference are not independent}$.

\item Our test statistic is $\chi^2 = \sum_{i=1}^2\sum_{j=1}^3 \frac{(\mathcal{O}_{i,j} - E_{i,j})^2}{E_{i,j}} = \frac{(17-12.75)^2}{12.75} + \frac{(7-9.98)^2}{9.98} + \dots + \frac{(27-25.46)^2}{25.46} = 3.75.$ \\

The P-value is the probability of obtaining a sample whose test statistic is at least as large as the one we observed (the larger the $\chi^2$ value, the farther the observed counts are from the expected counts). Thus, the P-value is $P(\chi^2 \geq 3.75)$, i.e., the probability a chi-square random variable with $(2-1)(3-1) = 2$ degrees of freedom takes a value greater than or equal to $3.75$.
\begin{center}
\begin{tikzpicture}[scale = 1]
  \begin{axis}[
  	  scale only axis,
      unit vector ratio=1 20 1,
      ymin=-0.0025,
      ymax=0.3,
      xmin=-0.1,
      xmax=15,
      tick label style={font=\scriptsize, scale = 1/1},
      xtick={0,1,2,3,4,5,6,7,8,9,10,11,12,13,14,15},
      ytick=\empty,
      legend pos=north east,
      domain=0:15,
      samples=200,
      thick
    ]

    % Chi-square with 5 degrees of freedom
    \addplot[blue]  {(1/2)*exp(-x/2)};
    
    \addplot[blue, opacity=0, fill=red, fill opacity=0.3, domain=3.75:15] 
      {(1/2)*exp(-x/2)} \closedcycle;

  \end{axis}
\end{tikzpicture}
\end{center}
Consulting a table of values for the chi-square distribution, we can find the critical value which produces a tail area of 10\% is about $4.605$. Therefore, the P-value in our test is (much) larger than 10\%. Using a computer to explicitly compute the area gives $P = 0.153$.

\item Since $P > 0.10$, we have not seen sufficient evidence to conclude that age and drink preference are not independent, at the 10\% significance level.
\end{enumerate}
\end{example}

\begin{remark}
We're testing how far our sample data is from what we would expect to see if the variables were independent. Either we have seen enough evidence to convince us they are not, or we haven't seen enough evidence to convince us they are not. We can never conclude the variables are independent from sample data.
\end{remark}

\begin{example}
A biologist crosses pea plants to examine the distribution of two traits in the offspring, seed shape (round or wrinkled) and seed colour (yellow or green). The resulting number of plants with each combination of traits is given below.
\renewcommand{\arraystretch}{1.25}
\begin{center}
\begin{tabular}{|c|c|c|c|c|c|c|}
\hline
 & Yellow & Green \\
\hline
Round & $29$ & $9$  \\
\hline
Wrinkled & $11$ & $11$ \\
\hline
\end{tabular}
\end{center}

If there is no genetic linkage between the genes that code for the two traits, we would expect the two variables to be independent (whether one is inherited does not change the probability of inheriting the other). Does this sample provide enough evidence, at the 5\% significance level, to conclude otherwise? 

We begin by finding the row and column totals, and computing the expected count in each cell.
\renewcommand{\arraystretch}{1.25}
\begin{center}
\begin{tabular}{|c|c|c|c|c|c|c|}
\hline
 & Yellow & Green & Total \\
\hline
Round & $25.33$ & $12.67$ & $38$ \\
\hline
Wrinkled & $14.67$ & $7.33$ & $22$ \\
\hline
Total & $40$ & $20$ & $60$ \\
\hline
\end{tabular}
\end{center}

\begin{enumerate}[label=\textnormal{\Roman*}.]
\item $H_0: \text{Seed shape and colour and are inherited independently (i.e. not linked)}$ \\
$H_a: \text{Seed shape and colour and are not inherited independently (i.e. linked)}$.

\item Our test statistic is $\chi^2 = \sum_{i=1}^2\sum_{j=1}^2 \frac{(\mathcal{O}_{i,j} - E_{i,j})^2}{E_{i,j}} = \frac{(29-25.33)^2}{25.33} + \frac{(9-12.67)^2}{12.67} + \dots + \frac{(11-7.33)^2}{7.33} = 4.35.$ \\

The P-value is the probability of obtaining a sample whose test statistic is at least as large as the one we observed, i.e., the probability a chi-square random variable with $(2-1)(2-1) = 1$ degree of freedom takes a value greater than or equal to $4.35$.
\begin{center}
\begin{tikzpicture}[scale = 1]
  \begin{axis}[
  	  scale only axis,
      unit vector ratio=1 20 1,
      ymin=-0.0025,
      ymax=0.3,
      xmin=-0.1,
      xmax=15,
      tick label style={font=\scriptsize, scale = 1/1},
      xtick={0,1,2,3,4,5,6,7,8,9,10,11,12,13,14,15},
      ytick=\empty,
      legend pos=north east,
      domain=0:15,
      samples=200,
      thick
    ]

    % Chi-square with 5 degrees of freedom
    \addplot[blue]  {(1/pow(2*pi*x,0.5))*exp(-x/2)};
    
    \addplot[blue, opacity=0, fill=red, fill opacity=0.3, domain=4.35:15] 
      {(1/pow(2*pi*x,0.5))*exp(-x/2)} \closedcycle;

  \end{axis}
\end{tikzpicture}
\end{center}
Consulting a table of values for the chi-square distribution, we can find the critical value which produces a tail area of 5\% is $3.841$. Therefore, the P-value in our test is smaller than 5\%. Using a computer to explicitly compute the area gives $P = 0.037$.

\item Since $P < 0.05$, we can conclude that seed shape and colour are not inherited independently, i.e., are linked, at the 5\% significance level.
\end{enumerate}
\end{example}

\begin{remark}
In fact, seed shape and colour in pea plants are inherited independently (this is a classic example in genetics), so in the test above a Type I error occurred. We rejected a true null hypothesis, which is what will occur in 5\% of samples.

The point to remember is that 95\% of the time, when the experiment is performed, the test will not achieve significance, so the balance of evidence will lead us to the correct conclusion.
\end{remark}

%\begin{example} In the United States, 8\% of families do not have a car, 33\% have one car, 38\% have two, 14\% have three, and 7\% have four or more. In a survey of a random sample of 210 Canadian families, 23 did not have a car, 81 had one car, 75 had two, 24 had three, and 7 had four or more.

%Is this sufficient evidence to conclude, at the 10\% significance level, that the distribution of the number of cars per family differs between the two countries?

%\renewcommand{\arraystretch}{1.25}
%\begin{center}
%\begin{tabular}{|c|c|c|c|c|c|c|}
%\hline
%Number of Cars & $0$ & $1$ & $2$ & $3$ & $> 3$ \\
%\hline
%Observed Count & $23$ & $81$ & $75$ & $24$ & $7$ \\
%\hline
%Expected Count & $210 \cdot 0.08$ & $210 \cdot 0.33$ & $210 \cdot 0.38$ & $210 \cdot 0.14$  & $210 \cdot 0.07$   \\
%\hline
%\end{tabular}
%\end{center}

%\begin{enumerate}[label=\textnormal{\Roman*}.]
%\item $H_0: \text{The distributions are identical}$, $H_a: \text{The distributions differ between countries}$.

%\item Our test statistic is $\chi^2 = \sum_{i=1}^5 \frac{(\mathcal{O}_i - E_i)^2}{E_i} = \frac{(23-16.8)^2}{16.8} + \frac{(81-69.3)^2}{69.3} + \dots + \frac{(7-14.7)^2}{14.7} = 9.58$

%The P-value is the probability a $\chi^2$ random variable with 4 degrees of freedom takes a value greater than 9.58, i.e., $P(\chi^2 \geq 9.58)$.
%\begin{center}
%\begin{tikzpicture}[scale = 1]
%  \begin{axis}[
%  	  scale only axis,
%      unit vector ratio=1 20 1,
%      ymin=-0.0025,
%      ymax=0.3,
%      xmin=-0.1,
%      xmax=15,
%      tick label style={font=\scriptsize, scale = 1/1},
%      xtick={0,1,2,3,4,5,6,7,8,9,10,11,12,13,14,15},
%      ytick=\empty,
%      legend pos=north east,
%      domain=0:15,
%      samples=200,
%      thick
%    ]
%
%    % Chi-square with 4 degrees of freedom
%    \addplot[blue]  {1/(pow(2,2)*1)*pow(x,1)*exp(-x/2)};
%    
%    \addplot[blue, opacity=0, fill=red, fill opacity=0.3, domain=9.58:15] 
%      {1/(pow(2,2)*1)*pow(x,1)*exp(-x/2)} \closedcycle;
%
%  \end{axis}
%\end{tikzpicture}
%\end{center}
%Consulting a table of values for the chi-square distribution, we can find the critical value which produces a tail area of 10\% is 7.78. Therefore, the P-value in our test is smaller than 10\%. Using a computer to explicitly compute the area gives $P = 0.048$.

%\item Since $P < 0.10$, we have seen sufficient evidence to conclude that the distribution of number of cars per family is different in the two countries, at the 10\% significance level.

%\end{enumerate}
%\end{example}

%\begin{remark}
%There's a variation of this test used to assess whether enough evidence is present in a set of sample data to conclude that two variables are not independent. That variation is usually called the chi-square test of independence, while the test developed in this section is known as the chi-square goodness-of-fit test.
%\end{remark}






% Modified 31 Oct 2005:  Conditioning fallacy alluded to.
% This chapter has been modified on 6-4-05.
% There are two \choice
\pagestyle{headings}
\chapter{Regression} \label{chp 6}

\section{Scatterplots \& the Coefficient of Correlation}

Suppose we have a data set with two quantitative variables, at or above the interval level of measurement. If we measure the values of the two variables on perpendicular axes, and add a point for each individual in our data, the resulting cloud of points is known as a \newterm{scatterplot}\index{Scatterplot}.

\begin{center}
\begin{tikzpicture}
    \begin{axis}[
        xlabel={},
        ylabel={},
        xmin=0, xmax=10,
        ymin=2, ymax=8,
        grid=major,
        title={}
    ]

    % Define the data points for the scatter plot
    \addplot[only marks, mark=*, color=blue] table {
         x y
 6.895 3.95
 2.475 6.242
 3.283 6.08
 6.374 4.667
 5.566 4.574
 4.504 4.416
 2.494 5.775
 3.036 5.958
 1.82 6.086
 5.688 4.869
 8.957 3.382
 2.156 6.88
 3.469 5.614
 2.662 6.624
 4.22 5.355
 6.697 4.516
 7.794 3.571
 5.45 4.636
 4.316 5.424
 7.286 3.911
 3.639 6.033
 3.39 6.02
 4.624 5.033
 3.083 6.428
 4.724 5.27
 3.651 5.933
 3.577 6.099
 7.262 3.731
 4.077 5.348
 4.879 4.789
 5.088 4.224
 2.268 6.182
 6.114 4.482
 4.322 5.074
 5.493 4.97
 3.405 6.524
 1.56 6.421
 6.014 4.595
 4.599 4.792
 6.396 4.8
 1.688 6.375
 6.233 4.293
 6.926 3.986
 4.143 5.344
 2.573 6.175
 4.013 5.367
 2.139 5.644
 5.371 4.211
 3.524 6.116
    };

    %\addplot[color=red, thick, domain=1:9] expression {1.1*x + 1};

    \end{axis}
\end{tikzpicture}
\end{center}

The scatterplot gives a visual representation of the relationship between the two variables, and often if there's a strong association between the two variables, it's clearly visible. In this case, we can see that larger values of the variable on the horizontal axis are associated with smaller values of the variable on the vertical axis. The following definition gives a quantitative measure of this association.

\begin{definition}
The \newterm{correlation coefficient}\index{Correlation Coefficient} is defined, for populations and samples respectively, as

$$\rho = \frac{1}{n}\sum_{i=1}^{n}\left(\frac{x_i - \mu_x}{\sigma_x}\right)\left(\frac{y_i - \mu_y}{\sigma_y}\right) \qquad \text{and} \ \ \qquad r = \frac{1}{n-1}\sum_{i=1}^{n}\left(\frac{x_i - \overline{x}}{s_x}\right)\left(\frac{y_i - \overline{y}}{s_y}\right).$$

In both cases, it is the mean product of the $z$-scores, across all individuals in the dataset. Note that in the case of a sample, the sample standard deviation is used in place of the population standard deviation, and the result is not truly a mean, since we divide by $n-1$ instead of $n$, as when computing the sample variance.
\end{definition}

\begin{example}\label{CorrelationEx}
The math and reading grades in a random sample of five students from an elementary school are given below. Compute the coefficient of correlation, $r$.

\begin{center}
\begin{minipage}{0.4\textwidth}
\renewcommand{\arraystretch}{1.25}
\begin{center}
\begin{tabular}{|c|c|c|}
\hline
Name & Math Gr. & Read Gr. \\
\hline
Alice & $72$ & $84$ \\
\hline
Bob & $93$ & $82$  \\
\hline
Celia & $88$ & $91$  \\
\hline
Dave & $68$ & $75$  \\
\hline
Evan & $77$ & $73$  \\
\hline
\end{tabular}
\vspace*{0.1in}
\end{center}
\end{minipage}\begin{minipage}{0.4\textwidth}
\begin{center}
\begin{tikzpicture}[scale=0.65]
    \begin{axis}[
        xlabel={Math Grade},
        ylabel={Reading Grade},
        tick label style={font=\scriptsize, scale = 1/0.65},
        label style={font=\scriptsize, scale = 1/0.65},
        xmin=60, xmax=100,
        ymin=60, ymax=100,
        grid=major,
        title={}
    ]

    % Define the data points for the scatter plot
    \addplot[only marks, mark=*, color=blue] table {
        x y
        72 84
        93 82
        88 91
        68 75
        77 73
    };

    \end{axis}
\end{tikzpicture}
\end{center}
\end{minipage}
\end{center}

Note that we will compute the sample correlation, $r$, which implies we're actually interested in the association between the two variables in the population we're sampling from (all students at the school), not just the association observed in this particular group of five students.

The mean and standard deviation for math grades are $\overline{x} = 79.60$ and $s_x = 10.60$, and the mean and standard deviation for reading grades are $\overline{y} = 81.00$ and $s_y = 7.25$. Standardizing each of the values, then computing the product across each row, summing the results, and dividing by $n-1 = 4$,

\begin{center}
\begin{minipage}{0.32\textwidth}
\renewcommand{\arraystretch}{1.25}
\begin{center}
\begin{tabular}{|c|c|c|}
\hline
Name & Math Z & Read Z \\
\hline
Alice & $-0.72$ & $0.41$ \\
\hline
Bob & $1.26$ & $0.14$  \\
\hline
Celia & $0.79$ & $1.38$  \\
\hline
Dave & $-1.09$ & $-0.83$  \\
\hline
Evan & $-0.25$ & $-1.10$  \\
\hline
\end{tabular}
\vspace*{0.1in}
\end{center}
\end{minipage}\ \ \begin{minipage}{0.55\textwidth}
$\begin{aligned}r &= \textstyle\frac{1}{4}\sum_{i=1}^{5}z_{x_i}z_{y_i} \\
&= \textstyle\frac{1}{4}\left((-0.72)(0.41)+(1.26)(0.14)+ \dots + (-0.25)(-1.10)\right) \\
&= \textstyle\frac{1}{4} \cdot 2.68 \\ &= 0.54 \end{aligned}$
\end{minipage}
\end{center}
Thus, the sample correlation coefficient between math grades and reading grades is $r = 0.54$.
\end{example}

\begin{keypoint}
The coefficient of correlation is a measure of how well the points in the scatterplot line up along a straight line. There could be other clear patterns in the scatterplot, and the value of one variable can even be completely determined by the other, but we can still have $r = 0$, which is the case for each scatterplot below.
\end{keypoint}

\begin{center}
\begin{minipage}{0.33\textwidth}
\begin{center}
\begin{tikzpicture}[scale=0.5]
    \begin{axis}[
        tick label style={font=\scriptsize, scale = 1/0.65},
        label style={font=\scriptsize, scale = 1/0.65},
        xticklabel=\empty,
        yticklabel=\empty,
        xmin=0, xmax=100,
        ymin=0, ymax=100,
        grid=major,
        title={}
    ]

    % Define the data points for the scatter plot
    \addplot[only marks, mark=*, color=blue] table {
        x y
        50 10
        45 12
        55 12
        60 18
        40 18
        35 30
        65 30
        70 50
        30 50
        75 80
        25 80
            };

    \end{axis}
\end{tikzpicture}
\end{center}
\end{minipage}\begin{minipage}{0.33\textwidth}
\begin{center}
\begin{tikzpicture}[scale=0.5]
    \begin{axis}[
        tick label style={font=\scriptsize, scale = 1/0.65},
        label style={font=\scriptsize, scale = 1/0.65},
        xticklabel=\empty,
        yticklabel=\empty,
        xmin=0, xmax=100,
        ymin=0, ymax=100,
        grid=major,
        title={}
    ]

    % Define the data points for the scatter plot
    \addplot[only marks, mark=*, color=blue] table {
        x y
        15 50
        85 50
        20 55
        80 55
        20 45
        80 45
        25 60
        75 60
        25 40
        75 40
        30 35
        30 65
        70 35
        70 65
        35 70
        35 30
        65 70
        65 30
        40 75
        60 75
        40 25
        60 25
        45 80
        45 20
        55 80
        55 20
        50 85
        50 15
    };

    \end{axis}
\end{tikzpicture}
\end{center}
\end{minipage}\begin{minipage}{0.33\textwidth}
\begin{center}
\begin{tikzpicture}[scale=0.5]
    \begin{axis}[
        tick label style={font=\scriptsize, scale = 1/0.65},
        label style={font=\scriptsize, scale = 1/0.65},
        xticklabel=\empty,
        yticklabel=\empty,
        xmin=0, xmax=100,
        ymin=0, ymax=100,
        grid=major,
        title={}
    ]

    % Define the data points for the scatter plot
    \addplot[only marks, mark=*, color=blue] table {
        x y
        10 70
        20 70
        30 70
        40 70
        50 70
        60 70
        70 70
        80 70
        90 70
        10 30
        20 30
        30 30
        40 30
        50 30
        60 30
        70 30
        80 30
        90 30
    };

    \end{axis}
\end{tikzpicture}
\end{center}
\end{minipage}
\end{center}

%\begin{proposition}
%The coefficient of correlation always takes a value in the interval $[-1,1]$, and takes the value $\pm 1$ if and only if all points in the scatterplot lie on a line of nonzero slope.
%\end{proposition}
%\begin{proof}

%\end{proof}

The coefficient of correlation measures the strength and direction of the linear relationship between the variables. The result is typically summarized with the terms strong, weak, positive, and negative. Two variables with $r = 0.32$ are weakly positively correlated, while two variables with $r=-0.83$ are strongly negatively correlated.

\begin{proposition}
The coefficient of correlation always takes a value in the interval $[-1,1]$, and takes the value $\pm 1$ if and only if all points in the scatterplot lie on a line (excepting the cases where the scatterplot contains only a single point, or the line is vertical or horizontal, in which case $r$ is undefined).
\end{proposition}

\begin{remark}
It's important to realize the presence of a strong correlation between variables does not imply any causal link between them, and even when there is such a link, what the nature of that link might be. A common cautionary tale involves a town where an analyst finds a strong positive correlation between the number of active fire trucks in an area, and the number of fires in that area. The town decides to take advantage of this result and prevent fires by decommissioning all of their fire trucks.
\end{remark}

Suppose we'd like to predict the value of one variable (which we'll the \newterm{response variable}\index{Response Variable}, $y$) using the value of the other (which we'll call the \newterm{explanatory variable}\index{Explanatory Variable}, $x$). In other words, we would like to find a function $f$ whose inputs are possible values of $x$ and whose outputs we can interpret as predicted values of $y$. This very general problem of constructing a function which models the relationship between a quantitative explanatory and response variable from a sample of points is known as \newterm{regression}. %In the next section, we'll study this problem in the special case where the function $f$ is restricted to being linear.

\section{The Least-Squares Regression Line}

Given a scatterplot, with the explanatory variable on the horizontal axis, and the response variable on the vertical axis, our goal is to find a function $f$ which best describes the relationship between the two variables. 

If we restrict the form of $f$ to a linear function $f(x) = ax+b$, and measure the degree to which $f$ deviates from the data by summing the squared differences between the values of the response variable $y_i$ and the value of $f(x_i)$ on the line (these differences are known as \newterm{residuals}\index{Residuals}, denoted $r_i$), we obtain the \newterm{least squares regression line}\index{Regression Line}.

\begin{center}
\begin{minipage}{0.35\textwidth}
\begin{center}
\begin{tikzpicture}[scale=0.7]
    \begin{axis}[
        xlabel={},
        ylabel={},
        tick label style={font=\scriptsize, scale = 1/0.65},
        label style={font=\scriptsize, scale = 1/0.65},
        xmin=60, xmax=100,
        ymin=60, ymax=100,
        grid=major,
        title={}
    ]

    % Define the data points for the scatter plot
    \addplot[only marks, mark=*, color=blue] table {
        x y
        72 84
        93 82
        88 91
        68 75
        77 73
    };
    
    \addplot[color=red, thick, domain=68:93] expression {0.8*x+17};
    \addplot [dashed] coordinates {(72,84) (72,0.8*72+17)} node[midway, right] {$r_2$};
    \addplot [dashed] coordinates {(93,82) (93,0.8*93+17)} node[midway, right] {$r_5$};
    \addplot [dashed] coordinates {(88,91) (88,0.8*88+17)} node[midway, left] {$r_4$};
    \addplot [dashed] coordinates {(68,75) (68,0.8*68+17)} node[midway, left] {$r_1$};
    \addplot [dashed] coordinates {(77,73) (77,0.8*77+17)} node[midway, right] {$r_3$};

    \end{axis}
\end{tikzpicture}

\end{center}
\end{minipage}\qquad\begin{minipage}{0.6\textwidth}
The least squares regression line $f(x) = ax + b$ is the line that results in the smallest possible sum of squared residuals
$$RSS = \sum_{i=1}^{n} r_i = \sum_{i=1}^{n} (y_i - f(x_i))^2$$
across all $a, b \in \mathbb{R}$. Using techniques from linear algebra or multivariable calculus, it's possible to solve for the values of $a$ and $b$ in general. Doing so yields the result below.
\end{minipage}
\end{center}

\begin{proposition}\label{RegressionLineEq}
Given a sample of points $(x_1, y_1)$, $(x_2, y_2)$, \dots , $(x_n, y_n)$, the least squares regression line $f(x) = ax+b$ has slope $a = r \cdot \frac{s_y}{s_x}$ and y-intercept $b = \overline{y} - a\cdot\overline{x}$.
\end{proposition}

\begin{remark}
To write the equation of the regression line requires calculating $\overline{x}$, $\overline{y}$, $s_x$, $s_y$, and $r$. This means that if we've gone through the trouble of computing the coefficient of correlation, the equation of the regression line comes essentially for free.
\end{remark}

\begin{example}\label{GradesRegressionEx} Find the equation of the least squares regression line for the data below (this is the data from Example \ref{CorrelationEx}), using math grade as the explanatory variable, and reading grade as the response variable. Use the line to find the predicted reading grade for a student whose math grade is 91.

\begin{center}
\begin{minipage}{0.4\textwidth}
\renewcommand{\arraystretch}{1.25}
\begin{center}
\begin{tabular}{|c|c|c|}
\hline
Name & Math Gr. & Read Gr. \\
\hline
Alice & $72$ & $84$ \\
\hline
Bob & $93$ & $82$  \\
\hline
Celia & $88$ & $91$  \\
\hline
Dave & $68$ & $75$  \\
\hline
Evan & $77$ & $73$  \\
\hline
\end{tabular}
\vspace*{0.1in}
\end{center}
\end{minipage}\begin{minipage}{0.4\textwidth}
\begin{center}
\begin{tikzpicture}[scale=0.65]
    \begin{axis}[
        xlabel={Math Grade},
        ylabel={Reading Grade},
        tick label style={font=\scriptsize, scale = 1/0.65},
        label style={font=\scriptsize, scale = 1/0.65},
        xmin=60, xmax=100,
        ymin=60, ymax=100,
        grid=major,
        title={}
    ]

    % Define the data points for the scatter plot
    \addplot[only marks, mark=*, color=blue] table {
        x y
        72 84
        93 82
        88 91
        68 75
        77 73
    };
    
    \addplot[color=red, thick, domain=68:93] expression {0.37*x+51.54};

    \end{axis}
\end{tikzpicture}
\end{center}
\end{minipage}
\end{center}

In Example \ref{CorrelationEx}, we computed $\overline{x} = 79.60$, $\overline{y} = 81.00$, $s_x = 10.60$, $s_y = 7.25$, and $r = 0.64$, so we can calculate $a = 0.54 \cdot \frac{7.25}{10.60} = 0.37$ and $b = 81.00 - 0.37 \cdot 79.60 = 51.54$. Thus, the equation of the regression line is $f(x) = 0.37x+51.54$, which is shown on the scatterplot above.

Using the regression line, we can compute a predicted reading grade of $\widehat{y} = 0.38\cdot 91 + 51.54 \approx 86$ for a student whose math grade is 91. 
\end{example}

We use the notation $\widehat{y}$ to indicate a predicted value of the response variable from the regression line. In particular, if $(x_i, y_i)$ is one of the points in our sample data, then it's important to distinguish the observed value $y_i$ in the sample data from the predicted value $\widehat{y}_i = f(x_i)$, and with this notation we can write the residual $r_i$ as the difference $y_i - \widehat{y}_i$.

\begin{remark}
The domain of the regression line is restricted to values of the explanatory variable inside the range of the sample data. Using the regression line to make predictions outside the range of the sample data is known as \newterm{extrapolation}\index{Extrapolation}, and should always be avoided.
\end{remark}

Note that the slope of the regression line can be interpreted as the expected change in the response variable when the explanatory variable's value increases by one, though one has to be careful to avoid assuming any causal link between the values of the two. 

In some cases, like Example \ref{PlantsRegressionEx} below, we would expect that we can exert some degree of control over the value of the response variable by altering the explanatory variable. In others instances, like Example \ref{GradesRegressionEx} above, this may not be the case (despite the positive correlation between the grades, spending extra study time on math in order to obtain a higher grade could cause a student's reading grade to suffer).

\begin{example}\label{PlantsRegressionEx}
Twelve plants are grown under identical conditions, but with the amount of fertilizer used in the soil each grows in carefully controlled. The amount of fertilizer mixed into the soil when the plants are potted and the height of the plants after four weeks of growth are recorded in the table below.

\renewcommand{\arraystretch}{1.25}
\begin{center}
\begin{tabular}{|c|c|c|c|c|c|c|c|c|c|c|c|c|}
\hline
Fertilizer\,(g) & $15$ & $25$ & $20$ & $25$ & $40$ & $35$ & $30$ & $40$ & $45$ & $20$ & $30$  \\
\hline
Height\,(cm) & $14$ & $17$ & $18$ & $22$ & $27$ & $24$ & $24$ & $28$ & $28$ & $23$ & $26$   \\
\hline
\end{tabular}
\end{center}

Find the equation of the least squares regression line, using the amount of fertilizer as the explanatory variable and the height as the response variable, and interpret its slope in this context. What does the regression line predict the height of a plant grown in soil with 32\,g of fertilizer will be?

After computing $\overline{x} = 29.55$, $s_x = 9.61$, $\overline{y} = 22.82$, $s_y = 4.69$, and $r = 0.86$, we can then calculate $a = 0.86\cdot\frac{4.69}{9.61} = 0.42$ and $b = 22.82-0.42\cdot29.55 = 10.41$, and arrive at the equation $\widehat{y} = 0.42x+10.41$.

The slope of the line, 0.42, corresponds to the additional height predicted to result from using one extra gram of fertilizer. Of course, in practice one extra gram of fertilizer has a larger effect in under-fertilized soil than in well-fertilized soil, and at some point adding more fertilizer will do more harm than good. Nonetheless, the slope provides a summary of the effect of fertilizer in our linear model, within the range of the sample data.

If we let $x = 32$, the model predicts the height will be $\widehat{y} = 0.42\cdot 32 + 10.41 = 23.85\, cm$.
\end{example}

%There is a more efficient formula for the slope of the regression line which is useful in the case where the coefficient of correlation is not already known, and will be useful in the next section.

%\begin{proposition}\label{RegressionSlopeFormula}
%The slope of the regression line is given by $a = \displaystyle\frac{\sum_{i=1}^{n}(x_i-\overline{x})(y_i-\overline{y})}{\sum_{i=1}^{n}(x_i-\overline{x})^2}$.
%\end{proposition}

%\begin{proof}
%In Proposition \ref{RegressionLineEq} earlier, we have the expression $a = r\cdot\frac{s_y}{s_x}$ for the slope, and
%$$\begin{aligned}r\cdot\frac{s_y}{s_x} &= \frac{1}{n-1}\left(\sum_{i=1}^{n}\frac{(x_i-\overline{x})}{s_x}\frac{(y_i-\overline{y})}{s_y}\right)\cdot \frac{s_y}{s_x} \\ 
%&= \frac{s_y}{s_x}\cdot\frac{1}{n-1}\cdot\frac{1}{s_x s_y}\sum_{i=1}^{n}(x_i-\overline{x})(y_i-\overline{y}) \\ 
%&= \frac{1}{(n-1)s^2_x}\cdot\sum_{i=1}^{n}(x_i-\overline{x})(y_i-\overline{y}) \\ 
%&= \frac{1}{(n-1)\frac{1}{n-1}\sum_{i=1}^{n}(x_i-\overline{x})^2}\cdot\sum_{i=1}^{n}(x_i-\overline{x})(y_i-\overline{y}) \\[1ex]
%&= \frac{\sum_{i=1}^{n}(x_i-\overline{x})(y_i-\overline{y})}{\sum_{i=1}^{n}(x_i-\overline{x})^2} \\ \end{aligned}$$
%\end{proof}

\section{The Coefficient of Determination}

Now that we can find the equation of the regression line and use it to make predictions, our next concern is how to measure how useful the line actually is. One natural way to quantify this is to ask how much better the line is than a simpler alternative, always predicting the mean value of $y$. This should sound familiar, since what we're really doing here is comparing the regression line to a simpler model where the explanatory variable has \emph{no effect} on the value of the response variable.

\begin{definition}
The \newterm{total sum of squares}\index{TSS} is the sum of the squared differences between the observed values of the response variable and the mean value, and the \newterm{explained sum of squares}\index{ESS} is the sum of the squared differences between the predicted values of the response variable and the mean value.
$$TSS = \sum_{i=1}^{n} (y_i - \overline{y})^2 \qquad ESS = \sum_{i=1}^{n} (\widehat{y_i} - \overline{y})^2$$
\end{definition}
\begin{center}
\begin{minipage}{0.45\textwidth}
\begin{center}
\begin{tikzpicture}[scale=0.7]
    \begin{axis}[
        xlabel={},
        ylabel={},
        tick label style={font=\scriptsize, scale = 1/0.65},
        label style={font=\scriptsize, scale = 1/0.65},
        xmin=60, xmax=100,
        ymin=60, ymax=100,
        grid=major,
        title={}
    ]

    % Define the data points for the scatter plot
    \addplot[only marks, mark=*, color=blue] table {
        x y
        72 84
        93 82
        88 91
        68 75
        77 73
    };
    
    \addplot[color=black, thick, domain=68:93] expression {81};
    \addplot [dashed,thick] coordinates {(72,84) (72,81)} node[midway, right] {};
    \addplot [dashed,thick] coordinates {(93,82) (93,81)} node[midway, right] {};
    \addplot [dashed,thick] coordinates {(88,91) (88,81)} node[midway, left] {};
    \addplot [dashed,thick] coordinates {(68,75) (68,81)} node[midway, left] {};
    \addplot [dashed,thick] coordinates {(77,73) (77,81)} node[midway, right] {};

    \end{axis}
\end{tikzpicture}

\end{center}
\end{minipage}\qquad\begin{minipage}{0.45\textwidth}
\begin{tikzpicture}[scale=0.7]
    \begin{axis}[
        xlabel={},
        ylabel={},
        tick label style={font=\scriptsize, scale = 1/0.65},
        label style={font=\scriptsize, scale = 1/0.65},
        xmin=60, xmax=100,
        ymin=60, ymax=100,
        grid=major,
        title={}
    ]

    % Define the data points for the scatter plot
    %\addplot[only marks, mark=*, color=blue] table {
    %    x y
    %    72 84
    %    93 82
    %    88 91
    %    68 75
    %    77 73
    %};
    
    \addplot[color=black, thick, domain=68:93] expression {81};
    \addplot[color=red, thick, domain=68:93] expression {0.37*x+51.54};
    \addplot [dashed,thick] coordinates {(72,81) (72,0.37*72+51.54)} node[midway, right] {};
    \addplot [dashed,thick] coordinates {(93,81) (93,0.37*93+51.54)} node[midway, right] {};
    \addplot [dashed,thick] coordinates {(88,81) (88,0.37*88+51.54)} node[midway, left] {};
    \addplot [dashed,thick] coordinates {(68,81) (68,0.37*68+51.54)} node[midway, left] {};
    \addplot [dashed,thick] coordinates {(77,81) (77,0.37*77+51.54)} node[midway, right] {};

    \end{axis}
\end{tikzpicture}
\end{minipage}
\end{center}

\begin{keypoint}
The $TSS$ measures of the total amount of variation of the response variable around its mean. Notice the similarity between the $TSS$ and the sample variance (they differ only by factor of $n-1$). The $ESS$ measures the amount of variation from the mean which is accounted for by the regression line.
\end{keypoint}

If the sample points fall almost exactly along a straight line, then the $TSS$ and $ESS$ will be almost the same. In general, the higher the proportion of variation from the mean the regression line accounts for, the better it fits the data.  This motivates the definition below.

\begin{center}
\begin{minipage}{0.45\textwidth}
\begin{center}
\begin{tikzpicture}[scale=0.7]
    \begin{axis}[
        xlabel={},
        ylabel={},
        tick label style={font=\scriptsize, scale = 1/0.65},
        label style={font=\scriptsize, scale = 1/0.65},
        xmin=60, xmax=100,
        ymin=60, ymax=100,
        grid=major,
        title={}
    ]

    % Define the data points for the scatter plot
    \addplot[only marks, mark=*, color=blue] table {
        x y
        68 68
        70 72
        75 74
        84 84
        93 94
    };
    
    \addplot[color=black, thick, domain=68:93] expression {78.4};
    \addplot [dashed,thick] coordinates {(68,68) (68,78.4)} node[midway, right] {};
    \addplot [dashed,thick] coordinates {(70,72) (70,78.4)} node[midway, right] {};
    \addplot [dashed,thick] coordinates {(75,74) (75,78.4)} node[midway, left] {};
    \addplot [dashed,thick] coordinates {(84,84) (84,78.4)} node[midway, left] {};
    \addplot [dashed,thick] coordinates {(93,94) (93,78.4)} node[midway, right] {};

    \end{axis}
\end{tikzpicture}

\end{center}
\end{minipage}\qquad\begin{minipage}{0.45\textwidth}
\begin{tikzpicture}[scale=0.7]
    \begin{axis}[
        xlabel={},
        ylabel={},
        tick label style={font=\scriptsize, scale = 1/0.65},
        label style={font=\scriptsize, scale = 1/0.65},
        xmin=60, xmax=100,
        ymin=60, ymax=100,
        grid=major,
        title={}
    ]

    % Define the data points for the scatter plot
    % Define the data points for the scatter plot
    %\addplot[only marks, mark=*, color=blue] table {
    %    x y
    %    68 68
    %    70 72
    %    75 74
    %    84 84
    %    93 94
    %};
    
    \addplot[color=black, thick, domain=68:93] expression {78.4};
    \addplot[color=red, thick, domain=68:93] expression {x+0.04055};
    \addplot [dashed,thick] coordinates {(68,68+0.04055) (68,78.4)} node[midway, right] {};
    \addplot [dashed,thick] coordinates {(70,70+0.04055) (70,78.4)} node[midway, right] {};
    \addplot [dashed,thick] coordinates {(75,75+0.04055) (75,78.4)} node[midway, left] {};
    \addplot [dashed,thick] coordinates {(84,84+0.04055) (84,78.4)} node[midway, left] {};
    \addplot [dashed,thick] coordinates {(93,93+0.04055) (93,78.4)} node[midway, right] {};

    \end{axis}
\end{tikzpicture}
\end{minipage}
\end{center}

\begin{definition} 
The \newterm{coefficient of determination}\index{Coefficient of Determination} is given by $R^2 = \frac{ESS}{TSS}$. It measures the proportion of the variance of the response variable which is captured by the regression line.
\end{definition}

\begin{theorem}
The coefficient of determination is the square of the coefficient of correlation, $R^2 = r^2$.
\end{theorem}
\begin{proof}
$$\begin{aligned} R^2 &= \frac{ESS}{TSS}
= \frac{\sum_{i=1}^{n}(\widehat{y}_i - \overline{y})^2}{\sum_{i=1}^{n}(y_i - \overline{y})^2} 
= \frac{\sum_{i=1}^{n}(ax_i+b - \overline{y})^2}{\sum_{i=1}^{n}(y_i - \overline{y})^2}
= \frac{\sum_{i=1}^{n}(r\frac{s_y}{s_x} x_i + \overline{y} - r\frac{s_y}{s_x}\overline{x} - \overline{y})^2}{\sum_{i=1}^{n}(y_i - \overline{y})^2} \\[2ex] 
&= \frac{\sum_{i=1}^{n}(r\frac{s_y}{s_x}(x_i - \overline{x}))^2}{\sum_{i=1}^{n}(y_i - \overline{y})^2} = r^2\frac{{s_y}^2}{{s_x}^2} \frac{\sum_{i=1}^{n}(x_i - \overline{x})^2}{\sum_{i=1}^{n}(y_i - \overline{y})^2} = r^2\frac{{s_y}^2}{{s_x}^2} \frac{(n-1){s_x}^2}{(n-1){s_y}^2} = r^2 \end{aligned}$$
\end{proof}

This means that if we have done the work of calculating the coefficient of correlation $r$, not only can we easily write down the equation of the regression line, we also get a quantitative measure of how well it fits the data.

\begin{example}
Calculate the coefficient of determination in Examples \ref{GradesRegressionEx} and \ref{PlantsRegressionEx}. In which case is the regression line a more informative model?

In Example \ref{GradesRegressionEx}, the regression line was $\widehat{y} = 0.37x+51.54$ with $r = 0.54$. Here, $r^2 = 0.29$, so only 29\% of the variance in reading grades is explained by the regression line.

In Example \ref{PlantsRegressionEx}, the regression line was $\widehat{y} = 0.42x+10.41$ with $r = 0.86$. Here, $r^2 = 0.74$, so here 74\% of the variance in height is explained by the regression line, and the regression line is a much better predictive model.
\end{example}

\begin{example}
In a study of user preferences, the number of comedy movies and horror movies watched by a sample of twelve users of a streaming service in the past year was recorded. The results are given in the table below.

\renewcommand{\arraystretch}{1.25}
\begin{center}
\begin{tabular}{|c|c|c|c|c|c|c|c|c|c|c|c|c|}
\hline
Comedy & $4$ & $12$ & $15$ & $1$ & $8$ & $5$ & $10$ & $9$ & $14$ & $1$ & $8$ & $8$  \\
\hline
Horror & $6$ & $9$ & $8$ & $9$ & $2$ & $7$ & $4$ & $8$ & $3$ & $13$ & $9$ & $2$   \\
\hline
\end{tabular}
\end{center}

Does it seem like the least squares regression line would be a useful model for predicting the number of horror movies users view from the number of comedy movies?

If we compute $r = -0.39$ and $r^2 = 0.15$, we can see there is a weak negative correlation between the two variables, but only 15\% of the variance in the number of comedies would be captured by the regression line, so it seems that the regression line would not be a very useful predictive model in this scenario.
\end{example}

%\begin{theorem} $SST = SSE + SSR$ \end{theorem}

%\begin{proof}
%By adding and subtracting the predicted values $\widehat{y}_i$ in the expression for the $SST$ and expanding, we can decompose the $SST$ into three terms as follows.
%$$\begin{aligned}SST &= \sum_{i=1}^{n}(y_i - \overline{y})^2 \\
%&= \sum_{i=1}^{n}(y_i - \widehat{y}_i + \widehat{y}_i - \overline{y})^2 \\
%&= \sum_{i=1}^{n}(y_i - \widehat{y}_i)^2 + 2(y_i - \widehat{y}_i)(\widehat{y}_i - \overline{y}) + (\widehat{y}_i - \overline{y})^2 \\
%&= \sum_{i=1}^{n}(y_i - \widehat{y}_i)^2 + \sum_{i=1}^{n} (\widehat{y}_i - \overline{y})^2 + 2\sum_{i=1}^{n}(y_i - \widehat{y}_i)(\widehat{y}_i - \overline{y}) \\ 
%& = SSR + SSE + 2\sum_{i=1}^{n}(y_i - \widehat{y}_i)(\widehat{y}_i - \overline{y}) \end{aligned}$$
%It now suffices to show that the last term is zero. To do this, recall that the equation of the least squares regression line is $\widehat{y} = ax + (\overline{y} - a\overline{x})$, which we can rearrange to $\widehat{y} - \overline{y} = a(x - \overline{x})$. Thus, for any $(x_i,y_i)$ in our sample, we have $\widehat{y}_i - \overline{y} = a(x_i - \overline{x})$ and
%$$(y_i - \widehat{y}_i) = (y_i - \widehat{y}_i) - \overline{y} + \overline{y} = (y_i-\overline{y}) - (\widehat{y}_i - \overline{y}) = (y_i - \overline{y})-a(x_i - \overline{x})$$
%We can use these relationships to eliminate all the predicted values of $y$ in the last term in our sum,
%$$\begin{aligned}\sum_{i=1}^{n}(y_i - \widehat{y}_i)(\widehat{y}_i - \overline{y}) &= \sum_{i=1}^{n}((y_i - \overline{y}) - a(x_i - \overline{x}))a(x_i - \overline{x}) \\
%&= a \sum_{i=1}^{n} (x_i - \overline{x})(y_i - \overline{y}) -a(x_i-\overline{x})^2 \\
%&= a \left(\sum_{i=1}^{n} (x_i - \overline{x})(y_i - \overline{y}) - a\sum_{i=1}^{n}(x_i-\overline{x})^2\right) \\
%&= a \left(\sum_{i=1}^{n} (x_i - \overline{x})(y_i - \overline{y}) - \frac{\sum_{i=1}^{n}(x_i-\overline{x})(y_i-\overline{y})}{\sum_{i=1}^{n}(x_i-\overline{x})^2}\sum_{i=1}^{n}(x_i-\overline{x})^2\right) \\ 
%&= a \left(\sum_{i=1}^{n} (x_i - \overline{x})(y_i - \overline{y}) - \sum_{i=1}^{n}(x_i-\overline{x})(y_i-\overline{y})\right) = 0\\ \end{aligned} $$
%Note the use of Proposition \ref{RegressionSlopeFormula} in the second last line.

%\end{proof}






\backmatter

% --- Bibliography ---

\phantomsection  % ensures correct hyperlink from TOC
\addcontentsline{toc}{chapter}{Bibliography}
\bibliographystyle{siam}
\bibliography{201-SN1_Notes}{}


% --- Index ---

\cleardoublepage
\phantomsection  % ensures correct hyperlink from TOC
\addcontentsline{toc}{chapter}{Index}
\printindex


\end{document}
